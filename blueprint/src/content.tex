% In this file you should put the actual content of the blueprint.
% It will be used both by the web and the print version.
% It should *not* include the \begin{document}
%
% If you want to split the blueprint content into several files then
% the current file can be a simple sequence of \input. Otherwise It
% can start with a \section or \chapter for instance.

\chapter{Introduction}

The goal of this project is to formalize Succinct Non-Interactive Arguments of Knowledge (SNARKs) in
Lean. Our focus is on SNARKs based on Interactive Oracle Proofs (IOPs) and variants thereof (i.e.
Polynomial IOPs). We aim to develop a general framework for IOP-based SNARKs with verified, modular
building blocks and transformations. This modular approach enables us to construct complex protocols
from simpler components while ensuring correctness and soundness by construction.

\chapter{Oracle Reductions}\label{chap:oracle_reductions}

\section{Definitions}\label{sec:oracle_reductions_defs}

In this section, we give the basic definitions of a public-coin interactive oracle reduction
(henceforth called an oracle reduction or IOR). We will define its building blocks, and various
security properties.

\subsection{Format}\label{sec:oracle_reductions_defs_format}

An \textbf{(interactive) oracle reduction (IOR)} is an interactive protocol between two parties, a
\emph{prover} $\mathcal{P}$ and a \emph{verifier} $\mathcal{V}$. In ArkLib, IORs are defined in the
following setting:
\begin{enumerate}
    \item We work in an ambient dependent type theory (in our case, Lean).

    \item The protocol flow is fixed and defined by a given \emph{type signature}, which
    describes in each round which party sends a message to the other, and the type of that message.

    \item The prover and verifier has access to some inputs (called the \emph{(oracle) context}) at
    the beginning of the protocol. These inputs are classified as follows:
    \begin{itemize}
        \item \emph{Public inputs} (or \emph{statement}) $\mathbbm{x}$: available to both parties;
        \item \emph{Private inputs} (or \emph{witness}) $\mathbbm{w}$: available only to the prover;
        \item \emph{Oracle inputs} (or \emph{oracle statement}) $\mathbbm{ox}$: the underlying data
        is available to the prover, but it's only exposed as an oracle to the verifier. See~\Cref{def:oracle_interface} for more information.
        \item \emph{Shared oracle} $\mathcal{O}$: the oracle is available to both parties via an
        interface; in most cases, it is either empty, a probabilistic sampling oracle, a random
        oracle, or a group oracle (for the Algebraic Group Model). See~\Cref{sec:vcvio} for more
        information on oracle computations.
    \end{itemize}

    \item The messages sent from the prover may either: 1) be seen directly by the verifier, or 2)
    only available to a verifier through an \emph{oracle interface} (which specifies the type for
    the query and response, and the oracle's behavior given the underlying message).

    Currently, in the oracle reduction setting, we \emph{only} allow messages sent to be available
    through oracle interfaces. In the (non-oracle) reduction setting, all messages are available
    directly. Future extensions may allow for mixed visibility for prover's messages.

    \item $\mathcal{V}$ is assumed to be \emph{public-coin}, meaning that its challenges are chosen
    uniformly at random from the finite type corresponding to that round, and it uses no randomness
    otherwise, except from those coming from the shared oracle.

    \item At the end of the protocol, the prover and verifier outputs a new (oracle) context, which consists of:
    \begin{itemize}
        \item The verifier takes in the input statement and the challenges, performs an \emph{oracle} computation on the input oracle statements and the oracle messages, and outputs a new output statement.

        The verifier also outputs the new oracle statement in an implicit manner, by specifying a
        subset of the input oracle statements \& the oracle messages. Future extensions may allow for more flexibility in specifying output oracle statements (i.e. not just a subset, but a linear combination, or any other function).
        \item The prover takes in some final private state (maintained during protocol execution), and outputs a new output statement, new output oracle statement, and new output witness.
    \end{itemize}
\end{enumerate}

\begin{remark}[Literature Comparison]
In the literature, our definition corresponds to the notion of \emph{functional} IORs. Historically,
(vector) IOPs were the first notion to be introduced by~\cite{IOPs}; these are IORs where the output
statement is true/false, all oracle statements and messages are vectors over some alphabet $\Sigma$,
and the oracle interfaces are for querying specific positions in the vector. More recent works have
considered other oracle interfaces, e.g., polynomial oracles~\cite{Marlin, DARK}, generalized proofs
to reductions~\cite{ARoK, WARP, Arc, fics-facs}, and considered general oracle
interfaces~\cite{WHIR}. Most of the IOP theory has been distilled in the
textbook~\cite{ChiesaYogev2024}.

We have not seen any work that considers our most general setting, of IORs with arbitrary oracle interfaces.
\end{remark}

We now go into more details on these objects, and how they are represented in Lean. Our description will aim to be as close as possible to the Lean code, and hence may differ somewhat from ``mainstream'' mathematical \& cryptographic notation.

\begin{definition}[Oracle Interface]
    \label{def:oracle_interface}
    An oracle interface for an underlying data type $\mathsf{D}$ consists of the following:
    \begin{itemize}
        \item A type $\mathsf{Q}$ for queries to the oracle,
        \item A type $\mathsf{R}$ for responses from the oracle,
        \item A function $\mathsf{oracle} : \mathsf{D} \to \mathsf{Q} \to \mathsf{R}$ that specifies
        the oracle's behavior given the underlying data and a query.
    \end{itemize}
    \lean{OracleInterface}
\end{definition}

See \texttt{OracleInterface.lean} for common instances of $\mathsf{OracleInterface}$.


\begin{definition}[Context]
    \label{def:context}
    In an (oracle) reduction, its \emph{(oracle) context} consists of a statement type, a witness
    type, and (in the oracle case) an indexed list of oracle statement types.

    Currently, we do not abstract out / bundle the context as a separate structure, but rather
    specifies the types explicitly. This may change in the future.
\end{definition}

\begin{definition}[Protocol Specification]
    \label{def:protocol_spec}
    A protocol specification for an $n$-message (oracle) reduction, is an element of the following type:
    \begin{align*}
        \ProtocolSpec\ n &:= \Fin\ n \to \Direction \times \Type.
    \end{align*}
    In the above, $\Direction := \{ \PtoVdir, \VtoPdir \}$ is the type of possible directions of messages, and $\Fin\ n := \{ i : \bbN \quotient i < n \}$ is the type of all natural numbers less than $n$.

    In other words, for each step $i$ of interaction, the protocol specification describes the \emph{direction} of the message sent in that step, i.e., whether it is from the prover or from the verifier. It also describes the \emph{type} of that message.

    In the oracle setting, we also expect an oracle interface for each message from the prover to the verifier.
    \lean{ProtocolSpec}
\end{definition}

We define some supporting definitions for a protocol specification.

\begin{definition}[Protocol Specification Components]
    \label{def:protocol_spec_components}
    Given a protocol spec $\pSpec : \ProtocolSpec\ n$, we define:
    \begin{itemize}
        \item $\pSpec.\Dir\ i := (\pSpec\ i).\mathsf{fst}$ extracts the direction of the $i$-th message.
        \item $\pSpec.\Type\ i := (\pSpec\ i).\mathsf{snd}$ extracts the type of the $i$-th message.
        \item $\pSpec.\MessageIdx := \{i : \Fin\ n \quotient \pSpec.\Dir\ i = \PtoVdir\}$ is the subtype of indices corresponding to prover messages.
        \item $\pSpec.\ChallengeIdx := \{i : \Fin\ n \quotient \pSpec.\Dir\ i = \VtoPdir\}$ is the subtype of indices corresponding to verifier challenges.
        \item $\pSpec.\mathsf{Message}\ i := (i : \pSpec.\MessageIdx) \to \pSpec.\Type\ i.\mathsf{val}$ is an indexed family of message types in the protocol.
        \item $\pSpec.\mathsf{Challenge}\ i := (i : \pSpec.\ChallengeIdx) \to \pSpec.\Type\ i.\mathsf{val}$ is an indexed family of challenge types in the protocol.
    \end{itemize}
    \lean{ProtocolSpec.dir, ProtocolSpec.Type, ProtocolSpec.MessageIdx, ProtocolSpec.ChallengeIdx, ProtocolSpec.Message, ProtocolSpec.Challenge}
    \uses{def:protocol_spec}
\end{definition}

\begin{definition}[Protocol Transcript]
    \label{def:transcript}
        Given protocol specification $\pSpec : \ProtocolSpec\ n$, we define:
    \begin{itemize}
        \item A \emph{transcript} up to round $k : \Fin\ (n + 1)$ is an element of type
                \[ \Transcript\ k\ \pSpec := (i : \Fin\ k) \to \pSpec.\Type\ (\uparrow i : \Fin\ n) \]
        where $\uparrow i : \Fin\ n$ denotes casting $i : \Fin\ k$ to $\Fin\ n$ (valid since $k \leq n + 1$).

        \item A \emph{full transcript} is $\FullTranscript\ \pSpec := (i : \Fin\ n) \to \pSpec.\Type\ i$.

        \item The type of all \emph{messages} from prover to verifier is
        \[ \pSpec.\Messages := \prod_{i : \pSpec.\MessageIdx} \pSpec.\Message\ i \]

        \item The type of all \emph{challenges} from verifier to prover is
        \[ \pSpec.\Challenges := \prod_{i : \pSpec.\ChallengeIdx} \pSpec.\Challenge\ i \]
    \end{itemize}
    \lean{ProtocolSpec.Transcript, ProtocolSpec.Message, ProtocolSpec.Challenge}
    \uses{def:protocol_spec, def:protocol_spec_components}
\end{definition}

% In the interactive protocols we consider, both parties $P$ and $V$ may have access to a shared
% oracle $O$. An interactive protocol becomes an \emph{interactive (oracle) reduction} if its
% execution reduces an input relation $R_{\mathsf{in}}$ to an output relation $R_{\mathsf{out}}$. Here
% a relation is just a function $\mathsf{IsValid}: \mathsf{Statement} \times \mathsf{Witness} \to
% \mathsf{Bool}$, for some types \verb|Statement| and \verb|Witness|. We do not concern ourselves with
% the running time of $\mathsf{IsValid}$ in this project (though future extensions may prove that
% relations can be decided in polynomial time, for a suitable model of computation).

\begin{remark}[Design Decision]
    We do not enforce a particular interaction flow in the definition of an interactive (oracle) reduction. This is done so that we can capture all protocols in the most generality. Also, we want to allow the prover to send multiple messages in a row, since each message may have a different oracle representation (for instance, in the Plonk protocol, the prover's first message is a 3-tuple of polynomial commitments.)
\end{remark}

\begin{definition}[Type Signature of a Prover]
    \label{def:prover}
    A prover $\mathcal{P}$ in a reduction consists of the following components:

    \begin{itemize}
        \item \textbf{Prover State}: A family of types $\mathsf{PrvState} : \Fin(n+1) \to \Type$ representing the prover's internal state at each round of the protocol.

        \item \textbf{Input Processing}: A function
        \[ \mathsf{input} : \StmtIn \to \WitIn \to \mathsf{PrvState}(0) \]
        that initializes the prover's state from the input statement and witness.

        \item \textbf{Message Sending}: For each message index $i : \pSpec.\MessageIdx$, a function
        \[ \mathsf{sendMessage}_i : \mathsf{PrvState}(i.\mathsf{val}.\mathsf{castSucc}) \to \OracleComp(\oSpec, \pSpec.\Message(i) \times \mathsf{PrvState}(i.\mathsf{val}.\mathsf{succ})) \]
        that generates the message and updates the prover's state.

        \item \textbf{Challenge Processing}: For each challenge index $i : \pSpec.\ChallengeIdx$, a function
        \[ \mathsf{receiveChallenge}_i : \mathsf{PrvState}(i.\mathsf{val}.\mathsf{castSucc}) \to \pSpec.\Challenge(i) \to \mathsf{PrvState}(i.\mathsf{val}.\mathsf{succ}) \]
        that updates the prover's state upon receiving a challenge.

        \item \textbf{Output Generation}: A function
        \[ \mathsf{output} : \mathsf{PrvState}(\Fin.\mathsf{last}(n)) \to \StmtOut \times \WitOut \]
        that produces the final output statement and witness from the prover's final state.
    \end{itemize}
    \lean{Prover, ProverState, ProverInput, ProverRound, ProverOutput}
\end{definition}

\begin{definition}[Type Signature of an Oracle Prover]
    \label{def:oracle_prover}
    An oracle prover is a prover whose input statement includes the underlying data for oracle statements, and whose output includes oracle statements. Formally, it is a prover with input statement type $\StmtIn \times (\forall i : \iota_{\mathsf{si}}, \OStmtIn(i))$ and output statement type $\StmtOut \times (\forall i : \iota_{\mathsf{so}}, \OStmtOut(i))$, where:
    \begin{itemize}
        \item $\OStmtIn : \iota_{\mathsf{si}} \to \Type$ are the input oracle statement types
        \item $\OStmtOut : \iota_{\mathsf{so}} \to \Type$ are the output oracle statement types
    \end{itemize}
    \lean{OracleProver}
\end{definition}

% Our modeling of oracle reductions only consider \emph{public-coin} verifiers; that is, verifiers who
% only outputs uniformly random challenges drawn from the (finite) types, and uses no other
% randomness. Because of this fixed functionality, we can bake the verifier's behavior in the
% interaction phase directly into the protocol execution semantics.

After the interaction phase, the verifier may then run some verification procedure to check the
validity of the prover's responses. In this procedure, the verifier gets access to the public part
of the context, and oracle access to either the shared oracle, or the oracle inputs.
% This procedure differs depending on whether the verifier has
% full access, or only oracle access, to the prover's messages. Note that there is no difference on
% the prover side whether the protocol is an \emph{interactive oracle reduction (IOR)} or simply an
% \emph{interactive reduction (IR)}.

\begin{definition}[Type Signature of a Verifier]
    \label{def:verifier}
    A verifier $\mathcal{V}$ in a reduction is specified by a single function:
    \[ \mathsf{verify} : \StmtIn \to \FullTranscript(\pSpec) \to \OracleComp(\oSpec, \StmtOut) \]

    This function takes the input statement and the complete transcript of the protocol interaction, and performs an oracle computation (potentially querying the shared oracle $\oSpec$) to produce an output statement.

    The verifier is assumed to be \emph{public-coin}, meaning it only sends uniformly random challenges and uses no other randomness beyond what is provided by the shared oracle.
    \lean{Verifier}
\end{definition}

\begin{definition}[Type Signature of an Oracle Verifier]
    \label{def:oracle_verifier}
    An oracle verifier $\mathcal{V}$ consists of the following components:

    \begin{itemize}
        \item \textbf{Verification Logic}: A function
        \[ \mathsf{verify} : \StmtIn \to \pSpec.\Challenges \to \OracleComp(\oSpec \mathrel{++_\mathsf{o}} ([\OStmtIn]_\mathsf{o} \mathrel{++_\mathsf{o}} [\pSpec.\Message]_\mathsf{o}), \StmtOut) \]
        that takes the input statement and verifier challenges, and performs oracle queries to the shared oracle, input oracle statements, and prover messages to produce an output statement.

        \item \textbf{Output Oracle Embedding}: An injective function
        \[ \mathsf{embed} : \iota_{\mathsf{so}} \hookrightarrow \iota_{\mathsf{si}} \oplus \pSpec.\MessageIdx \]
        that specifies how each output oracle statement is derived from either an input oracle statement or a prover message.

        \item \textbf{Type Compatibility}: A proof term
        \[ \mathsf{hEq} : \forall i : \iota_{\mathsf{so}}, \OStmtOut(i) = \begin{cases}
            \OStmtIn(j) & \text{if } \mathsf{embed}(i) = \mathsf{inl}(j) \\
            \pSpec.\Message(k) & \text{if } \mathsf{embed}(i) = \mathsf{inr}(k)
        \end{cases} \]
        ensuring that output oracle statement types match their sources.
    \end{itemize}

    This design ensures that output oracle statements are always a subset of the available input oracle statements and prover messages.
    \lean{OracleVerifier}
\end{definition}

\begin{definition}[Oracle Verifier to Verifier Conversion]
    \label{def:oracle_verifier_to_verifier}
    An oracle verifier can be converted to a standard verifier through a natural simulation process. The key insight is that while an oracle verifier only has oracle access to certain data (input oracle statements and prover messages), a standard verifier can be given the actual underlying data directly.

    The conversion works as follows: when the oracle verifier needs to make an oracle query to some data, the converted verifier can respond to this query immediately using the actual underlying data it possesses. This is accomplished through the \texttt{OracleInterface} type class, which specifies for each data type how to respond to queries given the underlying data.

    Specifically, given an oracle verifier $\mathcal{V}_{\text{oracle}}$:
    \begin{itemize}
        \item The converted verifier $\mathcal{V}_{\text{oracle}}.\mathsf{toVerifier}$ takes as input both the statement \emph{and} the actual underlying data for all oracle statements
        \item When $\mathcal{V}_{\text{oracle}}$ attempts to query an oracle statement or prover message, the converted verifier uses the corresponding \texttt{OracleInterface} instance to compute the response from the actual data
        \item The output oracle statements are constructed according to the embedding specification, selecting the appropriate subset of input oracle statements and prover messages
    \end{itemize}
    \lean{OracleVerifier.toVerifier}
    \uses{def:oracle_verifier}
\end{definition}

An oracle reduction then consists of a type signature for the interaction, and a pair of prover and
verifier for that type signature.

\begin{definition}[Interactive Reduction]
    \label{def:reduction}
    An interactive reduction for protocol specification $\pSpec : \ProtocolSpec(n)$ and oracle specification $\oSpec$ consists of:
    \begin{itemize}
        \item A \textbf{prover} $\mathcal{P} : \Prover(\pSpec, \oSpec, \StmtIn, \WitIn, \StmtOut, \WitOut)$
        \item A \textbf{verifier} $\mathcal{V} : \Verifier(\pSpec, \oSpec, \StmtIn, \StmtOut)$
    \end{itemize}

    The reduction establishes a relationship between input relations on $(\StmtIn, \WitIn)$ and output relations on $(\StmtOut, \WitOut)$ through the interactive protocol defined by $\pSpec$.
    \lean{Reduction}
    \uses{def:prover, def:verifier}
\end{definition}

\begin{definition}[Interactive Oracle Reduction]
    \label{def:oracle_reduction}
    An interactive oracle reduction for protocol specification $\pSpec : \ProtocolSpec(n)$ with oracle interfaces for all prover messages, and oracle specification $\oSpec$, consists of:
    \begin{itemize}
        \item An \textbf{oracle prover} $\mathcal{P} : \OracleProver(\pSpec, \oSpec, \StmtIn, \WitIn, \StmtOut, \WitOut, \OStmtIn, \OStmtOut)$
        \item An \textbf{oracle verifier} $\mathcal{V} : \OracleVerifier(\pSpec, \oSpec, \StmtIn, \StmtOut, \OStmtIn, \OStmtOut)$
    \end{itemize}

    where:
    \begin{itemize}
        \item $\OStmtIn : \iota_{\mathsf{si}} \to \Type$ are the input oracle statement types with oracle interfaces
        \item $\OStmtOut : \iota_{\mathsf{so}} \to \Type$ are the output oracle statement types
    \end{itemize}

    The oracle reduction allows the verifier to access prover messages and oracle statements only through specified oracle interfaces, enabling more flexible and composable protocol designs.
    \lean{OracleReduction}
    \uses{def:oracle_prover, def:oracle_verifier}
\end{definition}

\subsection{Execution Semantics}\label{sec:execution_semantics}

We now define what it means to execute an oracle reduction. This is essentially achieved by first
executing the prover, interspersed with oracle queries to get the verifier's challenges (these will
be given uniform random probability semantics later on), and then executing the verifier's checks.
Any message exchanged in the protocol will be added to the context. We may also log information
about the execution, such as the log of oracle queries for the shared oracles, for analysis purposes
(i.e. feeding information into the extractor).

\begin{definition}[Prover Execution to Round]
    \label{def:prover_run_to_round}
    The execution of a prover up to round $i : \Fin(n+1)$ is defined inductively:

    \[ \mathsf{Prover}.\mathsf{runToRound}(i, \mathsf{stmt}, \mathsf{wit}) := \]
    \[ \mathsf{Fin}.\mathsf{induction}( \]
    \[ \quad \mathsf{pure}(\langle \mathsf{default}, \mathsf{prover}.\mathsf{input}(\mathsf{stmt}, \mathsf{wit}) \rangle), \]
    \[ \quad \mathsf{prover}.\mathsf{processRound}, \]
    \[ \quad i \]
    \[ ) \]

    where $\mathsf{processRound}$ handles individual rounds by either:
    \begin{itemize}
        \item \textbf{Verifier Challenge} ($\pSpec.\mathsf{dir}(j) = \mathsf{V\_to\_P}$): Query for a challenge and update prover state
        \item \textbf{Prover Message} ($\pSpec.\mathsf{dir}(j) = \mathsf{P\_to\_V}$): Generate message via $\mathsf{sendMessage}$ and update state
    \end{itemize}

    Returns the transcript up to round $i$ and the prover's state after round $i$.
    \lean{Prover.runToRound, Prover.processRound}
    \uses{def:prover, def:protocol_spec, def:transcript}
\end{definition}

\begin{definition}[Complete Prover Execution]
    \label{def:prover_run}
    The complete execution of a prover is defined as:

    \[ \mathsf{Prover}.\mathsf{run}(\mathsf{stmt}, \mathsf{wit}) := \mathsf{do} \; \{ \]
    \[ \quad \langle \mathsf{transcript}, \mathsf{state} \rangle \leftarrow \mathsf{prover}.\mathsf{runToRound}(\Fin.\mathsf{last}(n), \mathsf{stmt}, \mathsf{wit}) \]
    \[ \quad \langle \mathsf{stmtOut}, \mathsf{witOut} \rangle := \mathsf{prover}.\mathsf{output}(\mathsf{state}) \]
    \[ \quad \mathsf{return} \; \langle \mathsf{stmtOut}, \mathsf{witOut}, \mathsf{transcript} \rangle \]
    \[ \} \]

    Returns the output statement, output witness, and complete transcript.
    \lean{Prover.run}
    \uses{def:prover, def:prover_run_to_round}
\end{definition}

\begin{definition}[Verifier Execution]
    \label{def:verifier_run}
    The execution of a verifier is simply the application of its verification function:

    \[ \mathsf{Verifier}.\mathsf{run}(\mathsf{stmt}, \mathsf{transcript}) := \mathsf{verifier}.\mathsf{verify}(\mathsf{stmt}, \mathsf{transcript}) \]

    This takes the input statement and full transcript, and returns the output statement via an oracle computation.
    \lean{Verifier.run}
    \uses{def:verifier}
\end{definition}

\begin{definition}[Oracle Verifier Execution]
    \label{def:oracle_verifier_run}
    The execution of an oracle verifier is defined as:

    \[ \mathsf{OracleVerifier}.\mathsf{run}(\mathsf{stmt}, \mathsf{oStmtIn}, \mathsf{transcript}) := \mathsf{do} \; \{ \]
    \[ \quad \mathsf{f} := \mathsf{simOracle2}(\oSpec, \mathsf{oStmtIn}, \mathsf{transcript}.\mathsf{messages}) \]
    \[ \quad \mathsf{stmtOut} \leftarrow \mathsf{simulateQ}(\mathsf{f}, \mathsf{verifier}.\mathsf{verify}(\mathsf{stmt}, \mathsf{transcript}.\mathsf{challenges})) \]
    \[ \quad \mathsf{return} \; \mathsf{stmtOut} \]
    \[ \} \]

    This simulates the oracle access to input oracle statements and prover messages, then executes the verification logic.
    \lean{OracleVerifier.run}
    \uses{def:oracle_verifier, def:oracle_interface}
\end{definition}

\begin{definition}[Interactive Reduction Execution]
    \label{def:reduction_run}
    The execution of an interactive reduction consists of running the prover followed by the verifier:

    \[ \mathsf{Reduction}.\mathsf{run}(\mathsf{stmt}, \mathsf{wit}) := \mathsf{do} \; \{ \]
    \[ \quad \langle \mathsf{prvStmtOut}, \mathsf{witOut}, \mathsf{transcript} \rangle \leftarrow \mathsf{reduction}.\mathsf{prover}.\mathsf{run}(\mathsf{stmt}, \mathsf{wit}) \]
    \[ \quad \mathsf{stmtOut} \leftarrow \mathsf{reduction}.\mathsf{verifier}.\mathsf{run}(\mathsf{stmt}, \mathsf{transcript}) \]
    \[ \quad \mathsf{return} \; ((\mathsf{prvStmtOut}, \mathsf{witOut}), \mathsf{stmtOut}, \mathsf{transcript}) \]
    \[ \} \]

    Returns both the prover's output (statement and witness) and the verifier's output statement, along with the complete transcript.
    \lean{Reduction.run}
    \uses{def:reduction, def:prover_run, def:verifier_run}
\end{definition}

\begin{definition}[Oracle Reduction Execution]
    \label{def:oracle_reduction_run}
    The execution of an interactive oracle reduction is similar to a standard reduction but includes logging of oracle queries:

    \[ \mathsf{OracleReduction}.\mathsf{run}(\mathsf{stmt}, \mathsf{wit}, \mathsf{oStmt}) := \mathsf{do} \; \{ \]
    \[ \quad \langle \langle \mathsf{prvStmtOut}, \mathsf{witOut}, \mathsf{transcript} \rangle, \mathsf{proveQueryLog} \rangle \leftarrow \]
    \[ \qquad (\mathsf{simulateQ}(\mathsf{loggingOracle}, \mathsf{reduction}.\mathsf{prover}.\mathsf{run}(\langle \mathsf{stmt}, \mathsf{oStmt} \rangle, \mathsf{wit}))).\mathsf{run} \]
    \[ \quad \langle \mathsf{stmtOut}, \mathsf{verifyQueryLog} \rangle \leftarrow \]
    \[ \qquad (\mathsf{simulateQ}(\mathsf{loggingOracle}, \mathsf{reduction}.\mathsf{verifier}.\mathsf{run}(\mathsf{stmt}, \mathsf{oStmt}, \mathsf{transcript}))).\mathsf{run} \]
    \[ \quad \mathsf{return} \; ((\mathsf{prvStmtOut}, \mathsf{witOut}), \mathsf{stmtOut}, \mathsf{transcript}, \mathsf{proveQueryLog}, \mathsf{verifyQueryLog}) \]
    \[ \} \]

    Returns the same outputs as a standard reduction, plus logs of all oracle queries made by both the prover and verifier.
    \lean{OracleReduction.run}
    \uses{def:oracle_reduction, def:prover_run, def:oracle_verifier_run}
\end{definition}


\subsection{Security Properties}\label{sec:security}

We can now define properties of interactive reductions. The two main properties we consider in this
project are completeness and various notions of soundness. We will cover zero-knowledge at a later
stage.

First, for completeness, this is essentially probabilistic Hoare-style conditions on the execution
of the oracle reduction (with the honest prover and verifier). In other words, given a predicate on
the initial context, and a predicate on the final context, we require that if the initial predicate
holds, then the final predicate holds with high probability (except for some \emph{completeness}
error).

\begin{definition}[Completeness]
    \label{def:completeness}
    A reduction satisfies \textbf{completeness} with error $\epsilon \geq 0$ and with respect to
    input relation $R_{\text{in}}$ and output relation $R_{\text{out}}$, if for all valid statement-witness pair
    $(x_{\text{in}}, w_{\text{in}})$ for $R_{\text{in}}$, the execution between the honest prover and the honest verifier
    will result in a tuple $((x_{\text{out}}^P, w_{\text{out}}), x_{\text{out}}^V)$ such that:
    \begin{itemize}
        \item $R_{\text{out}}(x_{\text{out}}^V, w_{\text{out}}) = \text{True}$ (the output statement-witness pair is valid), and
        \item $x_{\text{out}}^P = x_{\text{out}}^V$ (the output statements are the same from both prover and verifier)
    \end{itemize}
    except with probability $\epsilon$.
    \lean{Reduction.completeness}
    \uses{def:reduction, def:reduction_run}
\end{definition}

\begin{definition}[Perfect Completeness]
    \label{def:perfect_completeness}
    A reduction satisfies \textbf{perfect completeness} if it satisfies completeness with error $0$.
    This means that the probability of the reduction outputting a valid statement-witness pair is
    \emph{exactly} 1 (instead of at least $1 - 0$).
    \lean{Reduction.perfectCompleteness}
    \uses{def:completeness}
\end{definition}

Almost all oracle reductions we consider actually satisfy \emph{perfect completeness}, which
simplifies the proof obligation. In particular, this means we only need to show that no matter what challenges are chosen, the verifier will always accept given messages from the honest prover.

\subsubsection{Extractors}

For knowledge soundness, we need to consider different types of extractors that can recover witnesses from malicious provers.

\begin{definition}[Straightline Extractor]
    \label{def:straightline_extractor}
    A \textbf{straightline, deterministic, non-oracle-querying extractor} takes in:
    \begin{itemize}
        \item the output witness $w_{\text{out}}$,
        \item the initial statement $x_{\text{in}}$,
        \item the IOR transcript $\tau$,
        \item the query logs from the prover and verifier
    \end{itemize}
    and returns a corresponding initial witness $w_{\text{in}}$.

    Note that the extractor does not need to take in the output statement, since it can be derived
    via re-running the verifier on the initial statement, the transcript, and the verifier's query
    log.

    This form of extractor suffices for proving knowledge soundness of most hash-based IOPs.
    \lean{Extractor.Straightline}
    \uses{def:transcript}
\end{definition}

\begin{definition}[Round-by-Round Extractor]
    \label{def:rbr_extractor}
    A \textbf{round-by-round extractor} with index $m$ is given:
    \begin{itemize}
        \item the input statement $x_{\text{in}}$,
        \item a partial transcript of length $m$,
        \item the prover's query log
    \end{itemize}
    and returns a witness to the statement.

    Note that the RBR extractor does not need to take in the output statement or witness.
    \lean{Extractor.RoundByRound}
    \uses{def:transcript}
\end{definition}

\begin{definition}[Rewinding Extractor]
    \label{def:rewinding_extractor}
    A \textbf{rewinding extractor} consists of:
    \begin{itemize}
        \item An extractor state type
        \item Simulation oracles for challenges and oracle queries for the prover
        \item A function that runs the extractor with the prover's oracle interface, allowing for calling the prover multiple times
    \end{itemize}
    This allows the extractor to rewind the prover to earlier states and try different challenges.
    \lean{Extractor.Rewinding}
    \uses{def:prover, def:oracle_interface}
\end{definition}

\subsubsection{Adversarial Provers}

% \begin{definition}[Adaptive Prover]
%     \label{def:adaptive_prover}
%     An \textbf{adaptive prover} extends the basic prover type with the ability to choose the input statement adaptively based on oracle access. This models stronger adversaries that can choose their statements after seeing some oracle responses.
%     \lean{Prover.Adaptive}
%     \uses{def:prover}
% \end{definition}

\begin{definition}[State-Restoration Prover]
    \label{def:sr_prover}
    A \textbf{state-restoration prover} is a modified prover that has query access to challenge oracles that can return the $i$-th challenge, for all $i$, given the input statement and the transcript up to that point.

    It takes in the input statement and witness, and outputs a full transcript of interaction,
    along with the output statement and witness.

    This models adversaries in the state-restoration setting where challenges can be queried programmably.
    \lean{Prover.StateRestoration.KnowledgeSoundness}
    \uses{def:transcript}
\end{definition}

\subsubsection{Soundness Definitions}

For soundness, we need to consider different notions. These notions differ in two main aspects:
\begin{itemize}
    \item Whether we consider the plain soundness, or knowledge soundness. The latter relies on the
    notion of an \emph{extractor}.
    \item Whether we consider plain, state-restoration, round-by-round, or rewinding notion of
    soundness.
\end{itemize}

We note that state-restoration knowledge soundness is necessary for the security of the SNARK
protocol obtained from the oracle reduction after composing with a commitment scheme and applying
the Fiat-Shamir transform. It in turn is implied by either round-by-round knowledge soundness, or
special soundness (via rewinding). At the moment, we only care about non-rewinding soundness, so mostly we will care about round-by-round knowledge soundness.

\begin{definition}[Soundness]
    \label{def:soundness}
    A reduction satisfies \textbf{soundness} with error $\epsilon \geq 0$ and with respect to input
    language $L_{\text{in}} \subseteq \text{Statement}_{\text{in}}$ and output language $L_{\text{out}} \subseteq \text{Statement}_{\text{out}}$ if:
    \begin{itemize}
        \item for all (malicious) provers with arbitrary types for witness types,
        \item for all arbitrary input witness,
        \item for all input statement $x_{\text{in}} \notin L_{\text{in}}$,
    \end{itemize}
    the execution between the prover and the honest verifier will result in an output statement
    $x_{\text{out}} \in L_{\text{out}}$ with probability at most $\epsilon$.
    \lean{Verifier.soundness}
    \uses{def:verifier, def:prover_run}
\end{definition}

\begin{definition}[Knowledge Soundness]
    \label{def:knowledge_soundness}
    A reduction satisfies \textbf{(straightline) knowledge soundness} with error $\epsilon \geq 0$ and
    with respect to input relation $R_{\text{in}}$ and output relation $R_{\text{out}}$ if:
    \begin{itemize}
        \item there exists a straightline extractor $E$, such that
        \item for all input statement $x_{\text{in}}$, witness $w_{\text{in}}$, and (malicious) prover,
        \item if the execution with the honest verifier results in a pair $(x_{\text{out}}, w_{\text{out}})$,
        \item and the extractor produces some $w'_{\text{in}}$,
    \end{itemize}
    then the probability that $(x_{\text{in}}, w'_{\text{in}})$ is not valid for $R_{\text{in}}$ and yet $(x_{\text{out}}, w_{\text{out}})$ is valid for $R_{\text{out}}$ is at most $\epsilon$.

    A (straightline) extractor for knowledge soundness is a deterministic algorithm that takes in the output public context after executing the oracle reduction, the side information (i.e. log of oracle queries from the malicious prover) observed during execution, and outputs the witness for the input context.

    Note that since we assume the context is append-only, and we append only the public (or oracle)
    messages obtained during protocol execution, it follows that the witness stays the same throughout
    the execution.
    \lean{Verifier.knowledgeSoundness}
    \uses{def:verifier, def:reduction_run, def:straightline_extractor}
\end{definition}

\subsubsection{Round-by-Round Security}

To define round-by-round (knowledge) soundness, we need to define the notion of a \emph{state function}. This is a (possibly inefficient) function $\mathsf{StateF}$ that, for every challenge sent by the verifier, takes in the transcript of the protocol so far and outputs whether the state is doomed or not. Roughly speaking, the requirement of round-by-round soundness is that, for any (possibly malicious) prover $P$, if the state function outputs that the state is doomed on some partial transcript of the protocol, then the verifier will reject with high probability.

\begin{definition}[State Function]
    \label{def:state_function}
    A \textbf{(deterministic) state function} for a verifier, with respect to input language $L_{\text{in}}$ and
    output language $L_{\text{out}}$, consists of a function that maps partial transcripts to boolean values, satisfying:
    \begin{itemize}
        \item For all input statements not in the language, the state function is false for the empty transcript
        \item If the state function is false for a partial transcript, and the next message is from the
        prover to the verifier, then the state function is also false for the new partial transcript
        regardless of the message
        \item If the state function is false for a full transcript, the verifier will not output a statement
        in the output language
    \end{itemize}
    \lean{Verifier.StateFunction}
\end{definition}

\begin{definition}[Knowledge State Function]
    \label{def:knowledge_state_function}
    A \textbf{knowledge state function} for a verifier, with respect to input relation $R_{\text{in}}$, output
    relation $R_{\text{out}}$, and intermediate witness types, extends the basic state function to track
    witness validity throughout the protocol execution. This is used to define round-by-round knowledge soundness.
    \lean{Verifier.KnowledgeStateFunction}
\end{definition}

\begin{definition}[Round-by-Round Soundness]
    \label{def:round_by_round_soundness}
    A protocol with verifier $\mathcal{V}$ satisfies \textbf{round-by-round soundness} with respect to input language
    $L_{\text{in}}$, output language $L_{\text{out}}$, and error function $\epsilon: \text{ChallengeIdx} \to \mathbb{R}_{\geq 0}$ if:
    \begin{itemize}
        \item there exists a state function for the verifier and the input/output languages, such that
        \item for all initial statements $x_{\text{in}} \notin L_{\text{in}}$,
        \item for all initial witnesses,
        \item for all provers,
        \item for all challenge rounds $i$,
    \end{itemize}
    the probability that:
    \begin{itemize}
        \item the state function is false for the partial transcript output by the prover
        \item the state function is true for the partial transcript appended by next challenge (chosen randomly)
    \end{itemize}
    is at most $\epsilon(i)$.
    \lean{Verifier.rbrSoundness}
    \uses{def:verifier, def:state_function, def:prover_run_to_round}
\end{definition}

\begin{definition}[Round-by-Round Knowledge Soundness]
    \label{def:round_by_round_knowledge_soundness}
    A protocol with verifier $\mathcal{V}$ satisfies \textbf{round-by-round knowledge soundness} with respect to input
    relation $R_{\text{in}}$, output relation $R_{\text{out}}$, and error function $\epsilon: \text{ChallengeIdx} \to \mathbb{R}_{\geq 0}$ if:
    \begin{itemize}
        \item there exists a knowledge state function for the verifier and the languages of the input/output relations,
        \item there exists a round-by-round extractor,
        \item for all initial statements,
        \item for all initial witnesses,
        \item for all provers,
        \item for all challenge rounds $i$,
    \end{itemize}
    the probability that:
    \begin{itemize}
        \item the extracted witness does not satisfy the input relation
        \item the state function is false for the partial transcript output by the prover
        \item the state function is true for the partial transcript appended by next challenge (chosen randomly)
    \end{itemize}
    is at most $\epsilon(i)$.
    \lean{Verifier.rbrKnowledgeSoundness}
    \uses{def:verifier, def:knowledge_state_function, def:rbr_extractor, def:prover_run_to_round}
\end{definition}

% \begin{remark}[Alternative Formulations of RBR Knowledge Soundness]
%     There are different ways to formulate round-by-round knowledge soundness, differing in whether
%     the extractor's failure to produce a valid witness is included as part of the security condition.
%     Some formulations condition on the extractor producing an invalid witness while the state function
%     transitions from false to true, while others may condition on the state function transition
%     regardless of extractor success. The current formalization includes the extractor failure as
%     part of the security condition.
% \end{remark}

% \subsubsection{Extractor Properties}

% These definitions are highly experimental and may change in the future. The goal is to put some conditions on the extractor in order for prove sequential composition preserves knowledge soundness.

% \begin{definition}[Monotone Straightline Extractor]
%     \label{def:monotone_straightline_extractor}
%     An extractor is \textbf{monotone} if its success probability on a given query log is the same as
%     the success probability on any extension of that query log. This property ensures that the extractor's
%     performance does not degrade when given more information.
%     \lean{Verifier.Extractor.Straightline.IsMonotone}
%     \uses{def:straightline_extractor}
% \end{definition}

% \begin{definition}[Monotone RBR Extractor]
%     \label{def:monotone_rbr_extractor}
%     A round-by-round extractor is \textbf{monotone} if its success probability on a given query log
%     is the same as the success probability on any extension of that query log.
%     \lean{Extractor.RoundByRoundOneShot.IsMonotone}
%     \uses{def:rbr_extractor}
% \end{definition}

\subsubsection{Implications Between Security Notions}

We have a lattice of security notions, with knowledge and round-by-round being two strengthenings of soundness.

\begin{theorem}[Knowledge Soundness Implies Soundness]
    \label{thm:knowledge_soundness_implies_soundness}
    Knowledge soundness with knowledge error $\epsilon < 1$ implies soundness with the same
    soundness error $\epsilon$, and for the corresponding input and output languages.
    \lean{Verifier.knowledgeSoundness_implies_soundness}
    \uses{def:knowledge_soundness, def:soundness}
\end{theorem}

\begin{theorem}[RBR Soundness Implies Soundness]
    \label{thm:rbr_soundness_implies_soundness}
    Round-by-round soundness with error function $\epsilon$ implies soundness with error
    $\sum_i \epsilon(i)$, where the sum is over all challenge rounds $i$.
    \lean{Verifier.rbrSoundness_implies_soundness}
    \uses{def:round_by_round_soundness, def:soundness}
\end{theorem}

\begin{theorem}[RBR Knowledge Soundness Implies RBR Soundness]
    \label{thm:rbr_knowledge_soundness_implies_rbr_soundness}
    Round-by-round knowledge soundness with error function $\epsilon$ implies round-by-round
    soundness with the same error function $\epsilon$.
    \lean{Verifier.rbrKnowledgeSoundness_implies_rbrSoundness}
    \uses{def:round_by_round_knowledge_soundness, def:round_by_round_soundness}
\end{theorem}

\begin{theorem}[RBR Knowledge Soundness Implies Knowledge Soundness]
    \label{thm:rbr_knowledge_soundness_implies_knowledge_soundness}
    Round-by-round knowledge soundness with error function $\epsilon$ implies knowledge soundness
    with error $\sum_i \epsilon(i)$, where the sum is over all challenge rounds $i$.
    \lean{Verifier.rbrKnowledgeSoundness_implies_knowledgeSoundness}
    \uses{def:round_by_round_knowledge_soundness, def:knowledge_soundness}
\end{theorem}

\subsubsection{Zero-Knowledge}

\begin{definition}[Simulator]
    \label{def:simulator}
    A \textbf{simulator} consists of:
    \begin{itemize}
        \item Oracle simulation capabilities for the shared oracles
        \item A prover simulation function that takes an input statement and produces a transcript
    \end{itemize}
    The simulator should have programming access to the shared oracles and be able to generate
    transcripts that are indistinguishable from real protocol executions.
    \lean{Reduction.Simulator}
\end{definition}

\begin{remark}[Zero-Knowledge Definition]
    We define honest-verifier zero-knowledge as follows: There exists a simulator such that for all
    (malicious) verifiers, the distributions of transcripts generated by the simulator and the
    interaction between the verifier and the prover are (statistically) indistinguishable.
    A full definition will be provided in future versions.
\end{remark}

\subsubsection{Oracle-Specific Security}

For oracle reductions, the security definitions are analogous to those for standard reductions, but adapted to work with oracle interfaces:

\begin{definition}[Oracle Reduction Completeness]
    \label{def:oracle_reduction_completeness}
    Completeness of an oracle reduction is the same as for non-oracle reductions, but applied to the
    converted reduction where oracle statements are handled through their interfaces.
    \lean{OracleReduction.completeness}
    \uses{def:oracle_reduction, def:completeness, def:oracle_verifier_to_verifier}
\end{definition}

\begin{definition}[Oracle Verifier Soundness]
    \label{def:oracle_verifier_soundness}
    Soundness of an oracle verifier is defined by converting it to a standard verifier and applying
    the standard soundness definition.
    \lean{OracleVerifier.soundness}
    \uses{def:oracle_verifier, def:soundness, def:oracle_verifier_to_verifier}
\end{definition}

\begin{definition}[Oracle Verifier Knowledge Soundness]
    \label{def:oracle_verifier_knowledge_soundness}
    Knowledge soundness of an oracle verifier is defined by converting it to a standard verifier
    and applying the standard knowledge soundness definition.
    \lean{OracleVerifier.knowledgeSoundness}
    \uses{def:oracle_verifier, def:knowledge_soundness, def:oracle_verifier_to_verifier}
\end{definition}

\begin{definition}[Oracle Verifier RBR Soundness]
    \label{def:oracle_verifier_rbr_soundness}
    Round-by-round soundness of an oracle verifier is defined by converting it to a standard verifier
    and applying the standard round-by-round soundness definition.
    \lean{OracleVerifier.rbrSoundness}
    \uses{def:oracle_verifier, def:round_by_round_soundness, def:oracle_verifier_to_verifier}
\end{definition}

\begin{definition}[Oracle Verifier RBR Knowledge Soundness]
    \label{def:oracle_verifier_rbr_knowledge_soundness}
    Round-by-round knowledge soundness of an oracle verifier is defined by converting it to a standard
    verifier and applying the standard round-by-round knowledge soundness definition.
    \lean{OracleVerifier.rbrKnowledgeSoundness}
    \uses{def:oracle_verifier, def:round_by_round_knowledge_soundness, def:oracle_verifier_to_verifier}
\end{definition}

By default, the properties we consider are perfect completeness and (straightline) round-by-round knowledge soundness. We can encapsulate these properties into the following typing judgement:

\[
    \Gamma := (\Psi; \Theta; \varSigma; \rho; \mathcal{O}) \vdash \{\mathcal{R}_1\} \quad \langle\mathcal{P}, \mathcal{V}, \mathcal{E}\rangle \quad \{\!\!\{\mathcal{R}_2; \mathsf{St}; \epsilon\}\!\!\}
\]

\subsubsection{State-Restoration Security}

\begin{definition}[State-Restoration Soundness]
    \label{def:sr_soundness}
    \textbf{State-restoration soundness} is a security notion where the adversarial prover has access to
    challenge oracles that can return the $i$-th challenge for any round $i$, given the input statement
    and the transcript up to that point. This models stronger adversaries in the programmable random
    oracle model or when challenges can be computed deterministically.

    A verifier satisfies state-restoration soundness if for all input statements not in the language,
    for all witnesses, and for all state-restoration provers, the probability that the verifier
    outputs a statement in the output language is bounded by the soundness error.

    \emph{Note: This definition is currently under development in the Lean formalization.}
    % \lean{Verifier.srSoundness}
\end{definition}

\begin{definition}[State-Restoration Knowledge Soundness]
    \label{def:sr_knowledge_soundness}
    \textbf{State-restoration knowledge soundness} extends state-restoration soundness with the
    requirement that there exists a straightline extractor that can recover valid witnesses from
    any state-restoration prover that convinces the verifier.

    \emph{Note: This definition is currently under development in the Lean formalization.}
    % \lean{Verifier.srKnowledgeSoundness}
\end{definition}


% We can now define properties of interactive reductions. The two main properties we consider in this
% project are completeness and various notions of soundness. We will cover zero-knowledge at a later
% stage.

% First, for completeness, this is essentially probabilistic Hoare-style conditions on the execution
% of the oracle reduction (with the honest prover and verifier). In other words, given a predicate on
% the initial context, and a predicate on the final context, we require that if the initial predicate
% holds, then the final predicate holds with high probability (except for some \emph{completeness}
% error).

% \begin{definition}[Completeness]
%     \label{def:completeness}
%     \lean{Reduction.completeness}
%     \uses{def:oracle_reduction}
% \end{definition}

% Almost all oracle reductions we consider actually satisfy \emph{perfect completeness}, which
% simplifies the proof obligation. In particular, this means we only need to show that no matter what challenges are chosen, the verifier will always accept given messages from the honest prover.

% For soundness, we need to consider different notions. These notions differ in two main aspects:
% \begin{itemize}
%     \item Whether we consider the plain soundness, or knowledge soundness. The latter relies on the
%     notion of an \emph{extractor}.
%     \item Whether we consider plain, state-restoration, round-by-round, or rewinding notion of
%     soundness.
% \end{itemize}

% We note that state-restoration knowledge soundness is necessary for the security of the SNARK
% protocol obtained from the oracle reduction after composing with a commitment scheme and applying
% the Fiat-Shamir transform. It in turn is implied by either round-by-round knowledge soundness, or
% special soundness (via rewinding). At the moment, we only care about non-rewinding soundness, so mostly we will care about round-by-round knowledge soundness.

% \begin{definition}[Soundness]
%     \label{def:soundness}
%     \lean{Verifier.soundness}
%     \uses{def:oracle_reduction}
% \end{definition}

% A (straightline) extractor for knowledge soundness is a deterministic algorithm that takes in the output public context after executing the oracle reduction, the side information (i.e. log of oracle queries from the malicious prover) observed during execution, and outputs the witness for the input context.

% Note that since we assume the context is append-only, and we append only the public (or oracle)
% messages obtained during protocol execution, it follows that the witness stays the same throughout
% the execution.

% \begin{definition}[Knowledge Soundness]
%     \label{def:knowledge_soundness}
%     \lean{Verifier.knowledgeSoundness}
%     \uses{def:oracle_reduction}
% \end{definition}

% To define round-by-round (knowledge) soundness, we need to define the notion of a \emph{state function}. This is a (possibly inefficient) function $\mathsf{StateF}$ that, for every challenge sent by the verifier, takes in the transcript of the protocol so far and outputs whether the state is doomed or not. Roughly speaking, the requirement of round-by-round soundness is that, for any (possibly malicious) prover $P$, if the state function outputs that the state is doomed on some partial transcript of the protocol, then the verifier will reject with high probability.

% \begin{definition}[State Function]
%     \label{def:state_function}
%     \lean{Verifier.StateFunction}
% \end{definition}

% \begin{definition}[Round-by-Round Soundness]
%     \label{def:round_by_round_soundness}
%     \lean{Verifier.rbrSoundness}
%     \uses{def:oracle_reduction}
% \end{definition}

% \begin{definition}[Round-by-Round Knowledge Soundness]
%     \label{def:round_by_round_knowledge_soundness}
%     \lean{Verifier.rbrKnowledgeSoundness}
%     \uses{def:oracle_reduction}
% \end{definition}

% \textbf{PL Formalization.} We write our definitions in PL notation in~\Cref{fig:type-defs}. The set of types $\Type$ is the same as Lean's dependent type theory (omitting universe levels); in particular, we care about basic dependent types (Pi and Sigma), finite natural numbers, finite fields, lists, vectors, and polynomials.

% \begin{figure}[t]
%     \[\begin{array}{rcl}
%         % Basic types
%         \mathsf{Type} &::=& \mathsf{Unit} \mid \mathsf{Bool} \mid \mathbb{N} \mid \mathsf{Fin}\; n \mid \mathbb{F}_q \mid \mathsf{List}\;(\alpha : \mathsf{Type}) \mid (i : \iota) \to \alpha\; i \mid (i : \iota) \times \alpha\; i \mid \dots \\[1em]
%         % Protocol message types
%         \mathsf{Dir} &::=& \mathsf{P2V.Pub} \mid \mathsf{P2V.Orac} \mid \mathsf{V2P} \\
%         \mathsf{OI}\; (\mathrm{M} : \Type) &::=& \langle \mathrm{Q}, \mathrm{R}, \mathrm{M} \to \mathrm{Q} \to \mathrm{R} \rangle \\
%         % Protocol type signature
%         \pSpec\; (n : \mathbb{N}) &::=& \mathsf{Fin}\; n \to (d : \mathsf{Dir}) \times (M : \Type) \times (\mathsf{if}\; d = \mathsf{P2V.Orac} \; \mathsf{then} \; \mathsf{OI}(M) \; \mathsf{else} \; \mathsf{Unit}) \\
%         % Oracle type signature
%         \oSpec \; (\iota : \mathsf{Type}) &::=& (i : \iota) \to \mathsf{dom}\; i \times \mathsf{range}\; i \\[1em]
%         % Contexts
%         \varSigma &::=& \emptyset \mid \varSigma \times \Type \\
%         \Omega &::=& \emptyset \mid \Omega \times \langle \mathrm{M} : \Type, \mathsf{OI}(\mathrm{M}) \rangle \\
%         \Psi &::=& \emptyset \mid \Psi \times \Type\\
%     \end{array}\]
%     \[\begin{array}{rcl}
%         \Gamma &::=& (\Psi; \Omega; \varSigma; \rho; \mathcal{O})\\
%         \mathsf{OComp}^{\mathcal{O}}\; (\alpha : \Type) &::=& \mid\; \mathsf{pure}\; (a : \alpha) \\
%         && \mid\; \mathsf{queryBind}\;(i : \iota)\; (q : \mathsf{dom}\; i)\; (k : \mathsf{range}\; i \to \mathsf{OComp}^{\mathcal{O}}\; \alpha) \\
%         && \mid\; \mathsf{fail} \\[1em]
%         \tau_{\mathsf{P}}(\Gamma) &::=& (i : \mathsf{Fin}\; n) \to (h : (\rho \; i).\mathsf{fst} = \mathsf{P2V}) \to \\
%         && \varSigma \to \Omega \to \Psi \to \rho_{[:i]} \to \mathsf{OComp}^{\mathcal{O}}\;\left( (\rho \; i).\mathsf{snd}\right) \\[1em]

%         \tau_{\mathsf{V}}(\Gamma) &::=& \varSigma \to (\rho.\mathsf{Chals}) \to \mathsf{OComp}^{\mathcal{O} :: \OI(\Omega) :: \OI(\rho.\mathsf{Msg.Orac})}\; \mathsf{Unit} \\[1em]
%         \tau_{\mathsf{E}}(\Gamma) &::=& \varSigma \to \Omega \to \rho.\mathsf{Transcript} \to \calO.\mathsf{QueryLog} \to \Psi
%     \end{array}\]
%     \caption{Type definitions for interactive oracle reductions}
%     \label{fig:type-defs}
% \end{figure}

% Using programming language notation, we can express an interactive oracle reduction as a typing judgment:
% \[
%     \Gamma := (\Psi; \Theta; \varSigma; \rho; \mathcal{O}) \vdash \mathcal{P} : \tau_{\mathsf{P}}(\Gamma), \; \mathcal{V} : \tau_{\mathsf{V}}(\Gamma)
% \]
% where:
% \begin{itemize}
%     \item $\Psi$ represents the witness (private) inputs
%     \item $\Theta$ represents the oracle inputs
%     \item $\varSigma$ represents the public inputs (i.e. statements)
%     \item $\mathcal{O} : \oSpec\; \iota$ represents the shared oracle
%     \item $\rho : \pSpec\; n$ represents the protocol type signature
%     \item $\mathcal{P}$ and $\mathcal{V}$ are the prover and verifier, respectively, being of the given types $\tau_{\mathsf{P}}(\Gamma)$ and $\tau_{\mathsf{V}}(\Gamma)$.
% \end{itemize}

% To exhibit valid elements for the prover and verifier types, we will use existing functions in the ambient programming language (e.g. Lean).

% By default, the properties we consider are perfect completeness and (straightline) round-by-round knowledge soundness. We can encapsulate these properties into the following typing judgement:

% \[
%     \Gamma := (\Psi; \Theta; \varSigma; \rho; \mathcal{O}) \vdash \{\mathcal{R}_1\} \quad \langle\mathcal{P}, \mathcal{V}, \mathcal{E}\rangle \quad \{\!\!\{\mathcal{R}_2; \mathsf{St}; \epsilon\}\!\!\}
% \]


\section{Composition of Oracle Reductions}\label{sec:composition_oracle_reductions}

In this section, we describe a suite of composition operators for building secure oracle reductions from simpler secure components. In other words, we define a number of definitions that govern how oracle reductions can be composed to form larger reductions, and how the resulting reduction inherits the security properties of the components.

% The first group of rules changes relations and shared oracles.

% \subsection{Changing Relations and Oracles}

% Here we express the consequence rule.  Namely, if we have an oracle reduction for
% \(\mathcal{R}_1 \implies \mathcal{R}_2\), along with
% \(\mathcal{R}_1' \implies \mathcal{R}_1\) and \(\mathcal{R}_2 \implies \mathcal{R}_2'\),
% then we obtain an oracle reduction for \(\mathcal{R}_1' \implies \mathcal{R}_2'\).

% \[
% \frac{%
%   \begin{array}{c}
%     \Psi; \Theta; \Sigma \vdash \{\mathcal{R}_1\}\; \langle \mathcal{P},\, \mathcal{V},\, \mathcal{E}\rangle^{\mathcal{O}} : \tau \;\{\!\!\{\mathcal{R}_2; \,\mathsf{St};\, \epsilon\}\!\!\} \\[1.5ex]
%     \mathcal{R}_1' \implies \mathcal{R}_1 \\[1.5ex]
%     \mathcal{R}_2 \implies \mathcal{R}_2'
%   \end{array}%
% }{%
%   \Psi; \Theta; \Sigma \vdash \{\mathcal{R}_1'\}\; \langle \mathcal{P},\, \mathcal{V},\, \mathcal{E}\rangle^{\mathcal{O}} : \tau \;\{\!\!\{\mathcal{R}_2'; \,\mathsf{St};\, \epsilon\}\!\!\}
% } \quad \text{(Conseq)}
% \]

% % Lifting shared oracles: if $\mathcal{O}_1 \subset \mathcal{O}_2$, then reductions with $\mathcal{O}_1$ lift to reductions with $\mathcal{O}_2$ with the same security properties

% % Frame rule
% \[
% \frac{%
%   \Psi; \Theta; \Sigma \vdash \{\mathcal{R}_1\} \; \langle\mathcal{P}, \mathcal{V}, \mathcal{E}\rangle^{\mathcal{O}} : \tau \; \{\!\!\{\mathcal{R}_2; \mathsf{St}; \epsilon\}\!\!\}
% }{%
%   \Psi; \Theta; \Sigma \vdash \{\mathcal{R} \times \mathcal{R}_1\} \; \langle\mathcal{P}, \mathcal{V}, \mathcal{E}\rangle^{\mathcal{O}} : \tau \; \{\!\!\{\mathcal{R} \times \mathcal{R}_2; \mathsf{St}; \epsilon\}\!\!\}
% } \quad \text{(Frame)}
% \]

% % Oracle lifting rule
% \[
% \frac{%
%   \begin{array}{c}
%     \Psi; \Theta; \Sigma \vdash \{\mathcal{R}_1\} \; \langle\mathcal{P}, \mathcal{V}, \mathcal{E}\rangle^{\mathcal{O}_1} : \tau \; \{\!\!\{\mathcal{R}_2; \mathsf{St}; \epsilon\}\!\!\} \\[1.5ex]
%     \mathcal{O}_1 \subset \mathcal{O}_2
%   \end{array}%
% }{%
%   \Psi; \Theta; \Sigma \vdash \{\mathcal{R}_1\} \; \langle\mathcal{P}, \mathcal{V}, \mathcal{E}\rangle^{\mathcal{O}_2} : \tau \; \{\!\!\{\mathcal{R}_2; \mathsf{St}; \epsilon\}\!\!\}
% } \quad \text{(Oracle-Lift)}
% \]

% TODO: figure out how the state function needs to change for these rules (they are basically the same, but not exactly)

% \subsection{Sequential Composition}

% The reason why we consider interactive (oracle) reductions at the core of our formalism is that we
% can \emph{compose} these reductions to form larger reductions. Equivalently, we can take a complex
% \emph{interactive (oracle) proof} (which differs only in that it reduces a relation to the
% \emph{trivial} relation that always outputs true) and break it down into a series of smaller
% reductions. The advantage of this approach is that we can prove security properties (completeness
% and soundness) for each of the smaller reductions, and these properties will automatically transfer
% to the larger reductions.

% This section is devoted to the composition of interactive (oracle) reductions, and proofs that the
% resulting reductions inherit the security properties of the two (or more) constituent reductions.

% Sequential composition can be expressed as the folowing rule:

% \[
% \frac{%
%   \begin{array}{c}
%     \Psi; \Theta; \varSigma \vdash \{\mathcal{R}_1\} \; \langle\mathcal{P}_1, \mathcal{V}_1, \mathcal{E}_1\rangle^{\mathcal{O}} : \tau_1 \; \{\!\!\{\mathcal{R}_2; \mathsf{St}_1; \epsilon_1\}\!\!\} \\[1.5ex]
%     \Psi; (\Theta :: \tau_1) ; \varSigma \vdash \{\mathcal{R}_2\} \; \langle\mathcal{P}_2, \mathcal{V}_2, \mathcal{E}_2\rangle^{\mathcal{O}} : \tau_2 \; \{\!\!\{\mathcal{R}_3; \mathsf{St}_2; \epsilon_2\}\!\!\}
%   \end{array}%
% }{%
%   \Psi; (\Theta :: \tau_1 :: \tau_2) \ ; \varSigma \vdash \{\mathcal{R}_1\} \; \langle\mathcal{P}_1 \circ \mathcal{P}_2, \mathcal{V}_1 \circ \mathcal{V}_2, \mathcal{E}_1 \circ_{\mathcal{V}_2} \mathcal{E}_2\rangle^{\mathcal{O}} : \tau_1 \oplus \tau_2 \; \{\!\!\{\mathcal{R}_3; \mathsf{St}_1 \oplus \mathsf{St}_2; \epsilon_1 \oplus \epsilon_2\}\!\!\}
% } \quad \text{(Seq-Comp)}
% \]

% % \begin{definition}[Sequential Composition of Protocol Type Signatures]
% %     \label{def:protocol_spec_composition}
% %     \lean{ProtocolSpec.append}
% % \end{definition}

% % \begin{definition}[Sequential Composition of Provers]
% %     \label{def:prover_composition}
% %     \lean{Prover.append}
% % \end{definition}

% % \begin{definition}[Sequential Composition of Oracle Verifiers]
% %     \label{def:oracle_verifier_composition}
% %     \lean{OracleVerifier.append}
% % \end{definition}

% % \begin{definition}[Sequential Composition of Oracle Reductions]
% %     \label{def:oracle_reduction_composition}
% %     \lean{Reduction.append}
% % \end{definition}

\subsection{Sequential Composition}\label{sec:sequential_composition}

Sequential composition allows us to chain together oracle reductions where the output context of one reduction becomes the input context of the next reduction. This is fundamental for building complex protocols from simpler components.

\subsubsection{Composition of Protocol Specifications}

We begin by defining how to compose protocol specifications and their associated structures.

\begin{definition}[Protocol Specification Append]
    \label{def:protocol_spec_append}
    Given two protocol specifications $\pSpec_1 : \ProtocolSpec\ m$ and $\pSpec_2 : \ProtocolSpec\ n$, their sequential composition is:
    \[ \pSpec_1 \mathrel{++_\mathsf{p}} \pSpec_2 : \ProtocolSpec\ (m + n) \]
    \lean{ProtocolSpec.append}
    \uses{def:protocol_spec}
\end{definition}

\begin{definition}[Full Transcript Append]
    \label{def:transcript_append}
    Given full transcripts $T_1 : \FullTranscript\ \pSpec_1$ and $T_2 : \FullTranscript\ \pSpec_2$, their sequential composition is:
    \[ T_1 \mathrel{++_\mathsf{t}} T_2 : \FullTranscript\ (\pSpec_1 \mathrel{++_\mathsf{p}} \pSpec_2) \]
    \lean{ProtocolSpec.FullTranscript.append}
    \uses{def:transcript}
\end{definition}

\subsubsection{Composition of Provers and Verifiers}

\begin{definition}[Prover Append]
    \label{def:prover_append}
    Given provers $P_1 : \Prover\ \pSpec_1\ \oSpec\ \StmtIn_1\ \WitIn_1\ \StmtOut_1\ \WitOut_1$ and $P_2 : \Prover\ \pSpec_2\ \oSpec\ \StmtOut_1\ \WitOut_1\ \StmtOut_2\ \WitOut_2$, their sequential composition is:
    \[ P_1.\append\ P_2 : \Prover\ (\pSpec_1 \mathrel{++_\mathsf{p}} \pSpec_2)\ \oSpec\ \StmtIn_1\ \WitIn_1\ \StmtOut_2\ \WitOut_2 \]

    The composed prover works by:
    \begin{itemize}
        \item Running $P_1$ on the input context to produce an intermediate context
        \item Using this intermediate context as input to $P_2$
        \item Outputting the final context from $P_2$
    \end{itemize}
    \lean{Prover.append}
    \uses{def:prover}
\end{definition}

\begin{definition}[Verifier Append]
    \label{def:verifier_append}
    Given verifiers $V_1 : \Verifier\ \pSpec_1\ \oSpec\ \StmtIn_1\ \StmtOut_1$ and $V_2 : \Verifier\ \pSpec_2\ \oSpec\ \StmtOut_1\ \StmtOut_2$, their sequential composition is:
    \[ V_1.\append\ V_2 : \Verifier\ (\pSpec_1 \mathrel{++_\mathsf{p}} \pSpec_2)\ \oSpec\ \StmtIn_1\ \StmtOut_2 \]

    The composed verifier first runs $V_1$ on the first part of the transcript, then runs $V_2$ on the second part using the intermediate statement from $V_1$.
    \lean{Verifier.append}
    \uses{def:verifier}
\end{definition}

\begin{definition}[Reduction Append]
    \label{def:reduction_append}
    Sequential composition of reductions combines the corresponding provers and verifiers:
    \[ R_1.\append\ R_2 : \Reduction\ (\pSpec_1 \mathrel{++_\mathsf{p}} \pSpec_2)\ \oSpec\ \StmtIn_1\ \WitIn_1\ \StmtOut_2\ \WitOut_2 \]
    \lean{Reduction.append}
    \uses{def:reduction, def:prover_append, def:verifier_append}
\end{definition}

\begin{definition}[Oracle Reduction Append]
    \label{def:oracle_reduction_append}
    Sequential composition extends naturally to oracle reductions by composing the oracle provers and oracle verifiers.
    \lean{OracleReduction.append}
    \uses{def:oracle_reduction}
\end{definition}

\subsubsection{General Sequential Composition}

For composing an arbitrary number of reductions, we provide a general composition operation.

\begin{definition}[General Protocol Specification Composition]
    \label{def:protocol_spec_compose}
    Given a family of protocol specifications $\pSpec : \forall i : \Fin(m+1), \ProtocolSpec\ (n\ i)$, their composition is:
    \[ \compose\ m\ n\ \pSpec : \ProtocolSpec\ (\sum_{i} n\ i) \]
    \lean{ProtocolSpec.compose}
    \uses{def:protocol_spec}
\end{definition}

\begin{definition}[General Prover Composition]
    \label{def:prover_compose}
    \lean{Prover.compose}
    \uses{def:prover, def:protocol_spec_compose}
\end{definition}

\begin{definition}[General Verifier Composition]
    \label{def:verifier_compose}
    \lean{Verifier.compose}
    \uses{def:protocol_spec_compose}
\end{definition}

\begin{definition}[General Reduction Composition]
    \label{def:reduction_compose}
    \lean{Reduction.compose}
    \uses{def:prover_compose, def:verifier_compose}
\end{definition}

\subsubsection{Security Properties of Sequential Composition}

The key insight is that security properties are preserved under sequential composition.

\begin{theorem}[Completeness Preservation under Append]
    \label{thm:completeness_append}
    If reductions $R_1$ and $R_2$ satisfy completeness with compatible relations and respective errors $\epsilon_1$ and $\epsilon_2$, then their sequential composition $R_1.\append\ R_2$ satisfies completeness with error $\epsilon_1 + \epsilon_2$.
    \lean{Reduction.completeness_append}
\end{theorem}

\begin{theorem}[Perfect Completeness Preservation under Append]
    \label{thm:perfect_completeness_append}
    If reductions $R_1$ and $R_2$ satisfy perfect completeness with compatible relations, then their sequential composition also satisfies perfect completeness.
    \lean{Reduction.perfectCompleteness_append}
\end{theorem}

\begin{theorem}[Soundness Preservation under Append]
    \label{thm:soundness_append}
    If verifiers $V_1$ and $V_2$ satisfy soundness with respective errors $\epsilon_1$ and $\epsilon_2$, then their sequential composition satisfies soundness with error $\epsilon_1 + \epsilon_2$.
    \lean{Verifier.append_soundness}
\end{theorem}

\begin{theorem}[Knowledge Soundness Preservation under Append]
    \label{thm:knowledge_soundness_append}
    If verifiers $V_1$ and $V_2$ satisfy knowledge soundness with respective errors $\epsilon_1$ and $\epsilon_2$, then their sequential composition satisfies knowledge soundness with error $\epsilon_1 + \epsilon_2$.
    \lean{Verifier.append_knowledgeSoundness}
\end{theorem}

\begin{theorem}[Round-by-Round Soundness Preservation under Append]
    \label{thm:rbr_soundness_append}
    If verifiers $V_1$ and $V_2$ satisfy round-by-round soundness, then their sequential composition also satisfies round-by-round soundness.
    \lean{Verifier.append_rbrSoundness}
\end{theorem}

\begin{theorem}[Round-by-Round Knowledge Soundness Preservation under Append]
    \label{thm:rbr_knowledge_soundness_append}
    If verifiers $V_1$ and $V_2$ satisfy round-by-round knowledge soundness, then their sequential composition also satisfies round-by-round knowledge soundness.
    \lean{Verifier.append_rbrKnowledgeSoundness}
\end{theorem}

Similar preservation theorems hold for the general composition of multiple reductions:

\begin{theorem}[General Completeness Preservation]
    \label{thm:completeness_compose}
    \lean{Reduction.completeness_compose}
\end{theorem}

\begin{theorem}[General Soundness Preservation]
    \label{thm:soundness_compose}
    \lean{Verifier.compose_soundness}
\end{theorem}

\begin{theorem}[General Knowledge Soundness Preservation]
    \label{thm:knowledge_soundness_compose}
    \lean{Verifier.compose_knowledgeSoundness}
\end{theorem}

\subsection{Lifting Contexts}\label{sec:lifting_contexts}

Another essential tool for modular oracle reductions is the ability to adapt reductions from one context to another. This allows us to apply reductions designed for simple contexts to more complex scenarios.

\subsubsection{Context Lenses}

The fundamental abstraction for context adaptation is a \emph{context lens}, which provides bidirectional mappings between outer and inner contexts.

\begin{definition}[Statement Lens]
    \label{def:statement_lens}
    A statement lens between outer context types $(\StmtIn_{\mathsf{outer}}, \StmtOut_{\mathsf{outer}})$ and inner context types $(\StmtIn_{\mathsf{inner}}, \StmtOut_{\mathsf{inner}})$ consists of:
    \begin{itemize}
        \item $\projStmt : \StmtIn_{\mathsf{outer}} \to \StmtIn_{\mathsf{inner}}$ (projection to inner context)
        \item $\liftStmt : \StmtIn_{\mathsf{outer}} \times \StmtOut_{\mathsf{inner}} \to \StmtOut_{\mathsf{outer}}$ (lifting back to outer context)
    \end{itemize}
    \lean{StatementLens}
\end{definition}

\begin{definition}[Witness Lens]
    \label{def:witness_lens}
    A witness lens between outer witness types $(\WitIn_{\mathsf{outer}}, \WitOut_{\mathsf{outer}})$ and inner witness types $(\WitIn_{\mathsf{inner}}, \WitOut_{\mathsf{inner}})$ consists of:
    \begin{itemize}
        \item $\projWit : \WitIn_{\mathsf{outer}} \to \WitIn_{\mathsf{inner}}$ (projection to inner context)
        \item $\liftWit : \WitIn_{\mathsf{outer}} \times \WitOut_{\mathsf{inner}} \to \WitOut_{\mathsf{outer}}$ (lifting back to outer context)
    \end{itemize}
    \lean{WitnessLens}
\end{definition}

\begin{definition}[Context Lens]
    \label{def:context_lens}
    A context lens combines a statement lens and a witness lens for adapting complete reduction contexts.
    \lean{ContextLens}
    \uses{def:statement_lens, def:witness_lens}
\end{definition}

\begin{definition}[Oracle Context Lens]
    \label{def:oracle_context_lens}
    For oracle reductions, we additionally need lenses for oracle statements that can simulate oracle access between contexts.
    \lean{OracleContextLens}
    \uses{def:context_lens}
\end{definition}

\subsubsection{Lifting Reductions}

Given a context lens, we can lift reductions from inner contexts to outer contexts.

\begin{definition}[Prover Context Lifting]
    \label{def:prover_lift_context}
    Given a prover $P$ for the inner context and a context lens, the lifted prover works by:
    \begin{itemize}
        \item Projecting the outer input to the inner context
        \item Running the inner prover
        \item Lifting the output back to the outer context
    \end{itemize}
    \lean{Prover.liftContext}
\end{definition}

\begin{definition}[Verifier Context Lifting]
    \label{def:verifier_lift_context}
    \lean{Verifier.liftContext}
\end{definition}

\begin{definition}[Reduction Context Lifting]
    \label{def:reduction_lift_context}
    \lean{Reduction.liftContext}
    \uses{def:prover_lift_context, def:verifier_lift_context}
\end{definition}

\subsubsection{Conditions for Security Preservation}

For lifting to preserve security properties, the context lens must satisfy certain conditions.

\begin{definition}[Completeness-Preserving Context Lens]
    \label{def:context_lens_is_complete}
    A context lens preserves completeness if it maintains relation satisfaction under projection and lifting.
    \lean{ContextLens.IsComplete}
\end{definition}

\begin{definition}[Soundness-Preserving Statement Lens]
    \label{def:statement_lens_is_sound}
    A statement lens preserves soundness if it maps invalid statements to invalid statements.
    \lean{StatementLens.IsSound}
\end{definition}

\begin{definition}[RBR Soundness-Preserving Statement Lens]
    \label{def:statement_lens_is_rbr_sound}
    For round-by-round soundness, we need a slightly relaxed soundness condition.
    \lean{StatementLens.IsRBRSound}
\end{definition}

\begin{definition}[Knowledge Soundness-Preserving Context Lens]
    \label{def:context_lens_is_knowledge_sound}
    A context lens preserves knowledge soundness if it maintains witness extractability.
    \lean{ContextLens.IsKnowledgeSound}
\end{definition}

\subsubsection{Security Preservation Theorems for Context Lifting}

\begin{theorem}[Completeness Preservation under Context Lifting]
    \label{thm:lift_context_completeness}
    If a reduction satisfies completeness and the context lens is completeness-preserving, then the lifted reduction also satisfies completeness.
    \lean{Reduction.liftContext_completeness}
\end{theorem}

\begin{theorem}[Soundness Preservation under Context Lifting]
    \label{thm:lift_context_soundness}
    If a verifier satisfies soundness and the statement lens is soundness-preserving, then the lifted verifier also satisfies soundness.
    \lean{Verifier.liftContext_soundness}
\end{theorem}

\begin{theorem}[Knowledge Soundness Preservation under Context Lifting]
    \label{thm:lift_context_knowledge_soundness}
    If a verifier satisfies knowledge soundness and the context lens is knowledge soundness-preserving, then the lifted verifier also satisfies knowledge soundness.
    \lean{Verifier.liftContext_knowledgeSoundness}
\end{theorem}

\begin{theorem}[RBR Soundness Preservation under Context Lifting]
    \label{thm:lift_context_rbr_soundness}
    If a verifier satisfies round-by-round soundness and the statement lens is RBR soundness-preserving, then the lifted verifier also satisfies round-by-round soundness.
    \lean{Verifier.liftContext_rbr_soundness}
\end{theorem}

\begin{theorem}[RBR Knowledge Soundness Preservation under Context Lifting]
    \label{thm:lift_context_rbr_knowledge_soundness}
    If a verifier satisfies round-by-round knowledge soundness and the context lens is knowledge soundness-preserving, then the lifted verifier also satisfies round-by-round knowledge soundness.
    \lean{Verifier.liftContext_rbr_knowledgeSoundness}
\end{theorem}

\subsubsection{Extractors and State Functions}

Context lifting also applies to extractors and state functions used in knowledge soundness and round-by-round soundness.

\begin{definition}[Straightline Extractor Lifting]
    \label{def:straightline_extractor_lift_context}
    \lean{StraightlineExtractor.liftContext}
\end{definition}

\begin{definition}[Round-by-Round Extractor Lifting]
    \label{def:rbr_extractor_lift_context}
    \lean{RBRExtractor.liftContext}
\end{definition}

\begin{definition}[State Function Lifting]
    \label{def:state_function_lift_context}
    \lean{Verifier.StateFunction.liftContext}
\end{definition}

These composition and lifting operators provide the essential building blocks for constructing complex oracle reductions from simpler components while preserving their security properties.

% \subsection{Virtualization}

% Another tool we will repeatedly use is the ability to change the context of an oracle reduction. This is often needed when we want to adapt an oracle reduction in a simple context into one for a more complex context.

% See the section on sum-check~\ref{sec:sumcheck} for an example.

% \begin{definition}[Mapping into Virtual Context]
%     \label{def:virtual_context_mapping}
%     In order to apply an oracle reduction on virtual data, we will need to provide a mapping from the current context to the virtual context. This includes:
%     \begin{itemize}
%         \item A mapping from the current public inputs to the virtual public inputs.
%         \item A simulation of the oracle inputs for the virtual context using the public and oracle
%         inputs for the current context.
%         \item A mapping from the current private inputs to the virtual private inputs.
%         \item A simulation of the shared oracle for the virtual context using the shared oracle for
%         the current context.
%     \end{itemize}
% \end{definition}

% \begin{definition}[Virtual Oracle Reduction]
%     \label{def:virtual_oracle_reduction}
%     Given a suitable mapping into a virtual context, we may define an oracle reduction via the following construction:
%     \begin{itemize}
%         \item The prover first applies the mappings to obtain the virtual context. The verifier does the same, but only for the non-private inputs.
%         \item The prover and verifier then run the virtual oracle reduction on the virtual context.
%     \end{itemize}
% \end{definition}

% We will show security properties for this virtualization process. One can see that completeness and soundness are inherited from the completeness and soundness of the virtual oracle reduction. However, (round-by-round) knowledge soundness is more tricky; this is because we must extract back to the witness of the original context from the virtual context.

% % virtual-ctx rule of the form: if we have an oracle reduction from R1 to R2 for context prime, a mapping `f` from ctx to ctx prime, and an inverse mapping `g` from witness of ctx prime to witness of ctx, then we have an oracle reduction from (R1 \circ f) to (R2 \circ g) for ctx. The prover & verifier & the state function are composed with `f`, and the extractor needs to be composed with both `f` and `g`.

% % Virtualization rule
% \[
% \frac{%
%   \begin{array}{c}
%     \Psi'; \Theta'; \Sigma' \vdash \{\mathcal{R}_1\} \; \langle\mathcal{P}, \mathcal{V}, \mathcal{E}\rangle^{\mathcal{O}} : \tau \; \{\!\!\{\mathcal{R}_2; \mathsf{St}; \epsilon\}\!\!\} \\[1.5ex]
%     f : (\Psi, \Theta, \Sigma) \to (\Psi', \Theta', \Sigma') \\[1.5ex]
%     g : \Psi' \to \Psi \\[1.5ex]
%     f.\mathsf{fst} \circ g = \mathsf{id}
%   \end{array}%
% }{%
%   \Psi; \Theta; \Sigma \vdash \{\mathcal{R}_1 \circ f\} \; \langle\mathcal{P} \circ f, \mathcal{V} \circ f, \mathcal{E} \circ (f, g)\rangle^{\mathcal{O}} : \tau \; \{\!\!\{\mathcal{R}_2 \circ f; \mathsf{St} \circ f; \epsilon\}\!\!\}
% } \quad \text{(Virtual-Ctx)}
% \]

% \subsection{Substitution}

% Finally, we need a transformation / inference rule that allows us to change the message type in a given round of an oracle reduction. In other words, we substitute a value in the round with another value, followed by a reduction establishing the relationship between the new and old values.

% % May need multiple rules for different types of substitutions (i.e. whether in context or in protocol type, where replacing oracle / public message with oracle / public / private message, etc.)

% Examples include:
% \begin{enumerate}
%     \item Substituting an oracle input by a public input:
%     \begin{itemize}
%         \item Often by just revealing the underlying data. This has no change on the prover, and for
%         the verifier, this means that any query to the oracle input can be locally computed.
%         \item A variant of this is when the oracle input consists of a data along with a proof that
%         the data satisfies some predicate. In this case, the verifier needs to additionally check
%         that the predicate holds for the substituted data.
%         \item Another common substitution is to replace a vector with its Merkle commitment, or a
%         polynomial with its polynomial commitment.
%     \end{itemize}
%     \item Substituting an oracle input by another oracle input, followed by a reduction for each
%     oracle query the verifier makes to the old oracle:
%     \begin{itemize}
%         \item This is also a variant of the previous case, where we do not fully substitute with a
%         public input, but do a ``half-substitution'' by substituting with another oracle input. This
%         happens e.g. when using a polynomial commitment scheme that is itself based on a vector
%         commitment scheme. One can cast protocols like Ligero / Brakedown / FRI / STIR in this
%         two-step process.
%     \end{itemize}
% \end{enumerate}


\section{The Fiat-Shamir Transformation}\label{sec:fiat_shamir}

(NOTE: generated by Claude 4 Sonnet, will need to be cleaned up)

The Fiat-Shamir transformation is a fundamental cryptographic technique that converts a public-coin interactive reduction into a non-interactive reduction by replacing verifier challenges with queries to a random oracle. This transformation removes the need for interaction while preserving important security properties under certain assumptions.

In our formalization, the Fiat-Shamir transformation takes an interactive reduction $R$ and produces a non-interactive reduction where the prover computes all messages at once, and the verifier derives the challenges using queries to a hash function (modeled as a random oracle) applied to the statement and the messages up to each challenge round.

\subsection{Oracle Interface for Fiat-Shamir Challenges}\label{sec:fiat_shamir_oracle_interface}

The key insight of the Fiat-Shamir transformation is to replace interactive challenges with deterministic computations based on the protocol messages so far.

\begin{definition}[Fiat-Shamir Challenge Oracle Interface]
    \label{def:fiat_shamir_challenge_oracle_interface}
    For a protocol specification $\pSpec$ and input statement type $\StmtIn$, the Fiat-Shamir challenge oracle interface for the $i$-th challenge is defined as follows:
    \begin{itemize}
        \item \textbf{Query type}: $\StmtIn \times \pSpec.\MessagesUpTo \;i.\mathsf{val}.\mathsf{castSucc}$
        \item \textbf{Response type}: $\pSpec.\Challenge \;i$
        \item \textbf{Oracle behavior}: Returns the challenge (which is determined by the random oracle)
    \end{itemize}

    The query consists of the input statement and all prover messages sent up to (but not including) round $i$.
    \lean{instChallengeOracleInterfaceFiatShamir}
\end{definition}

\begin{definition}[Fiat-Shamir Oracle Specification]
    \label{def:fiat_shamir_spec}
    The Fiat-Shamir oracle specification for a protocol $\pSpec$ with input statement type $\StmtIn$ is:
    \[ \mathsf{fiatShamirSpec} \;\pSpec \;\StmtIn : \OracleSpec \;\pSpec.\ChallengeIdx \]
    where for each challenge index $i$, the oracle domain is $\StmtIn \times \pSpec.\MessagesUpTo \;i.\mathsf{val}.\mathsf{castSucc}$ and the range is $\pSpec.\Challenge \;i$.

    This specification defines a family of oracles, one for each challenge round, that deterministically computes challenges based on the statement and messages up to that round.
    \lean{fiatShamirSpec}
\end{definition}

\subsection{Fiat-Shamir Transformation for Provers}\label{sec:fiat_shamir_prover}

The Fiat-Shamir transformation modifies the prover's execution to compute all messages non-interactively while simulating the verifier's challenges using oracle queries.

\begin{definition}[Fiat-Shamir Round Processing]
    \label{def:prover_process_round_fs}
    The modified round processing function for Fiat-Shamir maintains the prover messages (but not challenges) and the input statement throughout execution:

    \[ \mathsf{processRoundFS} \;j \;\mathsf{prover} \;\mathsf{currentResult} \]

    For each round $j$:
    \begin{itemize}
        \item If $j$ is a challenge round: Query the Fiat-Shamir oracle with the statement and messages so far, then update the prover state with the received challenge
        \item If $j$ is a message round: Generate the message using the prover's $\mathsf{sendMessage}$ function and append it to the message history
    \end{itemize}

    The key difference from standard execution is that challenges are derived via oracle queries rather than received from an interactive verifier.
    \lean{Prover.processRoundFS}
    \uses{def:fiat_shamir_spec}
\end{definition}

\begin{definition}[Fiat-Shamir Prover Execution]
    \label{def:prover_run_to_round_fs}
    The Fiat-Shamir prover execution up to round $i$ is defined as:

    \[ \mathsf{runToRoundFS} \;i \;\mathsf{stmt} \;\mathsf{prover} \;\mathsf{state} \]

    This executes the prover inductively using $\mathsf{processRoundFS}$, starting from the initial state and accumulating messages and the statement. Returns the messages up to round $i$, the input statement, and the prover's final state.
    \lean{Prover.runToRoundFS}
    \uses{def:prover_process_round_fs}
\end{definition}

\begin{definition}[Fiat-Shamir Prover Transformation]
    \label{def:prover_fiat_shamir}
    Given an interactive prover $P$ for protocol $\pSpec$, the Fiat-Shamir transformation produces a non-interactive prover:

    \[ P.\mathsf{fiatShamir} : \NonInteractiveProver \;(\forall i, \pSpec.\Message \;i) \;(\oSpec \mathrel{++_\mathsf{o}} \mathsf{fiatShamirSpec} \;\pSpec \;\StmtIn) \;\StmtIn \;\WitIn \;\StmtOut \;\WitOut \]

    The transformed prover:
    \begin{itemize}
        \item Has state type that combines the statement with the original prover's state at round 0, and uses the final state type for subsequent rounds
        \item On input, stores both the statement and initializes the original prover's state
        \item Sends a single message containing all of the original prover's messages, computed via $\mathsf{runToRoundFS}$
        \item Never receives challenges (since it's non-interactive)
        \item Outputs using the original prover's output function
    \end{itemize}
    \lean{Prover.fiatShamir}
    \uses{def:prover_run_to_round_fs}
\end{definition}

\subsection{Transcript Derivation and Verifier Transformation}\label{sec:fiat_shamir_verifier}

The Fiat-Shamir verifier must reconstruct the full interactive transcript from the prover's messages in order to run the original verification logic.

\begin{definition}[Fiat-Shamir Transcript Derivation]
    \label{def:derive_transcript_fs}
    Given a collection of prover messages and an input statement, the function $\mathsf{deriveTranscriptFS}$ reconstructs the full protocol transcript up to round $k$:

    \[ \mathsf{messages}.\mathsf{deriveTranscriptFS} \;\mathsf{stmt} \;k : \OracleComp \;(\oSpec \mathrel{++_\mathsf{o}} \mathsf{fiatShamirSpec} \;\pSpec \;\StmtIn) \;(\pSpec.\Transcript \;k) \]

    This is computed inductively:
    \begin{itemize}
        \item For challenge rounds: Query the Fiat-Shamir oracle with the statement and messages up to that point
        \item For message rounds: Use the corresponding message from the prover
    \end{itemize}

    The result is a complete transcript that includes both prover messages and verifier challenges.
    \lean{ProtocolSpec.Messages.deriveTranscriptFS}
    \uses{def:fiat_shamir_spec}
\end{definition}

\begin{definition}[Fiat-Shamir Verifier Transformation]
    \label{def:verifier_fiat_shamir}
    Given an interactive verifier $V$ for protocol $\pSpec$, the Fiat-Shamir transformation produces a non-interactive verifier:

    \[ V.\mathsf{fiatShamir} : \NonInteractiveVerifier \;(\forall i, \pSpec.\Message \;i) \;(\oSpec \mathrel{++_\mathsf{o}} \mathsf{fiatShamirSpec} \;\pSpec \;\StmtIn) \;\StmtIn \;\StmtOut \]

    The transformed verifier:
    \begin{itemize}
        \item Takes the input statement and a proof consisting of all prover messages
        \item Derives the full transcript using $\mathsf{deriveTranscriptFS}$
        \item Runs the original verifier's verification logic on the reconstructed transcript
    \end{itemize}
    \lean{Verifier.fiatShamir}
    \uses{def:derive_transcript_fs}
\end{definition}

\subsection{Fiat-Shamir Transformation for Reductions}\label{sec:fiat_shamir_reduction}

\begin{definition}[Fiat-Shamir Reduction Transformation]
    \label{def:reduction_fiat_shamir}
    Given an interactive reduction $R$ for protocol $\pSpec$, the Fiat-Shamir transformation produces a non-interactive reduction:

    \[ R.\mathsf{fiatShamir} : \NonInteractiveReduction \;(\forall i, \pSpec.\Message \;i) \;(\oSpec \mathrel{++_\mathsf{o}} \mathsf{fiatShamirSpec} \;\pSpec \;\StmtIn) \;\StmtIn \;\WitIn \;\StmtOut \;\WitOut \]

    This transformation simply applies the Fiat-Shamir transformation to both the prover and verifier components of the reduction.
    \lean{Reduction.fiatShamir}
    \uses{def:prover_fiat_shamir, def:verifier_fiat_shamir}
\end{definition}

\subsection{Security Properties}\label{sec:fiat_shamir_security}

The Fiat-Shamir transformation preserves important security properties of the original interactive reduction, under appropriate assumptions about the random oracle.

\begin{theorem}[Fiat-Shamir Preserves Completeness]
    \label{thm:fiat_shamir_completeness}
    Let $R$ be an interactive reduction with completeness error $\epsilon$ with respect to input relation $R_{\text{in}}$ and output relation $R_{\text{out}}$. Then the Fiat-Shamir transformed reduction $R.\mathsf{fiatShamir}$ also satisfies completeness with error $\epsilon$ with respect to the same relations.

    Formally: $R.\mathsf{completeness} \;R_{\text{in}} \;R_{\text{out}} \;\epsilon \to (R.\mathsf{fiatShamir}).\mathsf{completeness} \;R_{\text{in}} \;R_{\text{out}} \;\epsilon$
    \lean{fiatShamir_completeness}
    \uses{def:reduction_fiat_shamir}
\end{theorem}

\begin{remark}[Additional Security Properties]
    While completeness is straightforward to establish, soundness properties require more careful analysis. In particular:
    \begin{itemize}
        \item State-restoration knowledge soundness of the original reduction implies knowledge soundness of the Fiat-Shamir transformed reduction
        \item Honest-verifier zero-knowledge of the original reduction implies zero-knowledge of the transformed reduction
    \end{itemize}
    These results require the random oracle model and careful handling of the oracle programming needed for simulation and extraction. The formal statements and proofs of these results are currently under development.
\end{remark}

\begin{remark}[Implementation Considerations]
    Our formalization models the "theoretical" version of Fiat-Shamir where the entire statement and transcript prefix are hashed to derive each challenge. In practice, more efficient variants use cryptographic sponges or other techniques to incrementally absorb transcript elements and squeeze out challenges. Our theoretical model provides the foundation for analyzing these practical variants.
\end{remark}


\chapter{Proof Systems}\label{chap:proof_systems}

\section{Simple Oracle Reductions}

We start by introducing a number of simple oracle reductions that serve as fundamental building blocks for more complex proof systems. These components can be composed together to construct larger protocols.

\subsection{The Trivial Reduction: DoNothing}

The simplest possible oracle reduction is one that performs no computation at all. Both the prover and verifier simply pass their inputs through unchanged. While seemingly trivial, this reduction serves as an important identity element for composition and provides a base case for lifting and lens operations.

\begin{definition}[DoNothing Reduction]
    \label{def:donothing_reduction}
    The DoNothing reduction is a zero-round protocol with the following components:
    \begin{itemize}
        \item \textbf{Protocol specification}: $\pSpec := []$ (empty protocol, no messages exchanged)
        \item \textbf{Prover}: Simply stores the input statement and witness, and outputs them unchanged
        \item \textbf{Verifier}: Takes the input statement and outputs it directly
        \item \textbf{Input relation}: Any relation $R_{\mathsf{in}} : \StmtIn \to \WitIn \to \Prop$
        \item \textbf{Output relation}: The same relation $R_{\mathsf{out}} := R_{\mathsf{in}}$
    \end{itemize}
    \lean{DoNothing.reduction, DoNothing.oracleReduction}
\end{definition}

\begin{theorem}[DoNothing Perfect Completeness]
    The DoNothing reduction satisfies perfect completeness for any input relation.
    \lean{DoNothing.reduction_completeness, DoNothing.oracleReduction_completeness}
    \uses{def:donothing_reduction}
\end{theorem}

The oracle version of DoNothing handles oracle statements by passing them through unchanged as well. The prover receives both non-oracle and oracle statements as input, and outputs them in the same form to the verifier.

\subsection{Witness Transmission: SendWitness}

A fundamental building block in many proof systems is the ability for the prover to transmit witness information to the verifier. The SendWitness reduction provides this functionality in both direct and oracle settings.

\begin{definition}[SendWitness Reduction]
    \label{def:sendwitness_reduction}
    The SendWitness reduction is a one-round protocol where the prover sends the complete witness to the verifier:
    \begin{itemize}
        \item \textbf{Protocol specification}: $\pSpec := [(\PtoVdir, \WitIn)]$ (single message from prover to verifier)
        \item \textbf{Prover}: Sends the witness $\mathbbm{w}$ as its single message
        \item \textbf{Verifier}: Receives the witness and combines it with the input statement to form the output
        \item \textbf{Input relation}: $R_{\mathsf{in}} : \StmtIn \to \WitIn \to \Prop$
        \item \textbf{Output relation}: $R_{\mathsf{out}} : (\StmtIn \times \WitIn) \to \Unit \to \Prop$ defined by $((\mathsf{stmt}, \mathsf{wit}), ()) \mapsto R_{\mathsf{in}}(\mathsf{stmt}, \mathsf{wit})$
    \end{itemize}
    \lean{SendWitness.reduction}
    \uses{def:reduction}
\end{definition}

\begin{theorem}[SendWitness Perfect Completeness]
    The SendWitness reduction satisfies perfect completeness.
    \lean{SendWitness.reduction_completeness}
    \uses{def:sendwitness_reduction}
\end{theorem}

In the oracle setting, we consider two variants:

\begin{definition}[SendWitness Oracle Reduction]
    \label{def:sendwitness_oracle_reduction}
    The oracle version handles witnesses that are indexed families of types with oracle interfaces:
    \begin{itemize}
        \item \textbf{Witness type}: $\WitIn : \iota_w \to \Type$ where each $\WitIn(i)$ has an oracle interface
        \item \textbf{Protocol specification}: $\pSpec := [(\PtoVdir, \forall i, \WitIn(i))]$
        \item \textbf{Output oracle statements}: Combination of input oracle statements and the transmitted witness
    \end{itemize}
    % \lean{SendWitness.oracleReduction}
    \uses{def:oracle_reduction}
\end{definition}

\begin{definition}[SendSingleWitness Oracle Reduction]
    \label{def:sendsinglewitness_oracle_reduction}
    A specialized variant for a single witness with oracle interface:
    \begin{itemize}
        \item \textbf{Witness type}: $\WitIn : \Type$ with oracle interface
        \item \textbf{Protocol specification}: $\pSpec := [(\PtoVdir, \WitIn)]$
        \item \textbf{Conversion}: Implicitly converts to indexed family $\WitIn : \Fin(1) \to \Type$
    \end{itemize}
    \lean{SendSingleWitness.oracleReduction}
\end{definition}

\begin{theorem}[SendSingleWitness Perfect Completeness]
    The SendSingleWitness oracle reduction satisfies perfect completeness.
    \lean{SendSingleWitness.oracleReduction_completeness}
    \uses{def:sendsinglewitness_oracle_reduction}
\end{theorem}

\subsection{Oracle Equality Testing: RandomQuery}

One of the most fundamental oracle reductions is testing whether two oracles of the same type are equal. This is achieved through random sampling from the query space.

\begin{definition}[RandomQuery Oracle Reduction]
    \label{def:randomquery_oracle_reduction}
    The RandomQuery reduction tests equality between two oracles by random querying:
    \begin{itemize}
        \item \textbf{Input}: Two oracles $\mathsf{a}, \mathsf{b}$ of the same type with oracle interface
        \item \textbf{Protocol specification}: $\pSpec := [(\VtoPdir, \mathsf{Query})]$ (single challenge from verifier)
        \item \textbf{Input relation}: $R_{\mathsf{in}}((), (\mathsf{a}, \mathsf{b}), ()) := (\mathsf{a} = \mathsf{b})$
        \item \textbf{Verifier}: Samples random query $q$ and sends it to prover
        \item \textbf{Prover}: Receives query $q$, performs no computation
        \item \textbf{Output relation}: $R_{\mathsf{out}}((q, (\mathsf{a}, \mathsf{b})), ()) := (\mathsf{oracle}(\mathsf{a}, q) = \mathsf{oracle}(\mathsf{b}, q))$
    \end{itemize}
    \lean{RandomQuery.oracleReduction}
    \uses{def:oracle_reduction}
\end{definition}

\begin{theorem}[RandomQuery Perfect Completeness]
    The RandomQuery oracle reduction satisfies perfect completeness: if two oracles are equal, they will agree on any random query.
    \lean{RandomQuery.oracleReduction_completeness}
    \uses{def:randomquery_oracle_reduction}
\end{theorem}

The key security property of RandomQuery depends on the notion of oracle distance:

\begin{definition}[Oracle Distance]
    \label{def:oracle_distance}
    For oracles $\mathsf{a}, \mathsf{b}$ of the same type, we define their distance as:
    \[ \mathsf{distance}(\mathsf{a}, \mathsf{b}) := |\{q : \mathsf{Query} \mid \mathsf{oracle}(\mathsf{a}, q) \neq \mathsf{oracle}(\mathsf{b}, q)\}| \]

    We say an oracle type has distance bound $d$ if for any two distinct oracles $\mathsf{a} \neq \mathsf{b}$, we have $\mathsf{distance}(\mathsf{a}, \mathsf{b}) \geq |\mathsf{Query}| - d$.
\end{definition}

\begin{theorem}[RandomQuery Knowledge Soundness]
    If the oracle type has distance bound $d$, then the RandomQuery oracle reduction satisfies round-by-round knowledge soundness with error probability $\frac{d}{|\mathsf{Query}|}$.
    \lean{RandomQuery.rbr_knowledge_soundness}
    \uses{def:randomquery_oracle_reduction, def:oracle_distance}
\end{theorem}

\begin{definition}[RandomQueryWithResponse Variant]
    \label{def:randomquery_with_response}
    A variant of RandomQuery where the second oracle is replaced with an explicit response:
    \begin{itemize}
        \item \textbf{Input}: Single oracle $\mathsf{a}$ and target response $r$
        \item \textbf{Output relation}: $R_{\mathsf{out}}(((q, r), \mathsf{a}), ()) := (\mathsf{oracle}(\mathsf{a}, q) = r)$
    \end{itemize}
    This variant is useful when one wants to verify a specific query-response pair rather than oracle equality.
    % \lean{RandomQueryWithResponse}
    \uses{def:randomquery_oracle_reduction}
\end{definition}

\subsection{Polynomial Equality Testing}

A common application of oracle reductions is testing equality between polynomial oracles. This is a specific instance of the RandomQuery reduction applied to polynomial evaluation oracles.

\begin{definition}[Polynomial Equality Testing]
    \label{def:polynomial_equality_testing}
    Consider two univariate polynomials $P, Q \in \mathbb{F}[X]$ of degree at most $d$, available as polynomial evaluation oracles. The polynomial equality testing reduction is defined as:
    \begin{itemize}
        \item \textbf{Input relation}: $P = Q$ as polynomials
        \item \textbf{Protocol specification}: Single challenge of type $\mathbb{F}$ from verifier to prover
        \item \textbf{Honest prover}: Receives the random field element $r$ but performs no computation
        \item \textbf{Honest verifier}: Checks that $P(r) = Q(r)$ by querying both polynomial oracles
        \item \textbf{Output relation}: $P(r) = Q(r)$ for the sampled point $r$
    \end{itemize}
\end{definition}

\begin{theorem}[Polynomial Equality Testing Completeness]
    The polynomial equality testing reduction satisfies perfect completeness: if $P = Q$ as polynomials, then $P(r) = Q(r)$ for any field element $r$.
\end{theorem}

\begin{theorem}[Polynomial Equality Testing Soundness]
    The polynomial equality testing reduction satisfies round-by-round knowledge soundness with error probability $\frac{d}{|\mathbb{F}|}$, where $d$ is the maximum degree bound. This follows from the Schwartz-Zippel lemma: distinct polynomials of degree at most $d$ can agree on at most $d$ points.
\end{theorem}

The state function for this reduction corresponds precisely to the input and output relations, transitioning from checking polynomial equality to checking evaluation equality at the sampled point.

\subsection{Batching Polynomial Evaluation Claims}

Another important building block is the ability to batch multiple polynomial evaluation claims into a single check using random linear combinations.

TODO: express this as a lifted version of $\mathsf{RandomQuery}$ over a virtual polynomial whose
variables are the random linear combination coefficients.

\begin{definition}[Batching Polynomial Evaluation Claims]
    \label{def:batching_polynomial_evaluation}
    Consider an $n$-tuple of values $v = (v_1, \ldots, v_n) \in \mathbb{F}^n$ and a polynomial map $E : \mathbb{F}^k \to \mathbb{F}^n$. The batching reduction is defined as:
    \begin{itemize}
        \item \textbf{Protocol specification}: Two messages:
        \begin{enumerate}
            \item Verifier sends random $r \in \mathbb{F}^k$ to prover
            \item Prover sends $\langle E(r), v \rangle := \sum_{i=1}^n E(r)_i \cdot v_i$ to verifier
        \end{enumerate}
        \item \textbf{Honest prover}: Computes the inner product $\langle E(r), v \rangle$ and sends it
        \item \textbf{Honest verifier}: Verifies that the received value equals the expected inner product
        \item \textbf{Extractor}: Trivial since there is no witness to extract
    \end{itemize}
\end{definition}

\begin{theorem}[Batching Completeness]
    The batching polynomial evaluation reduction satisfies perfect completeness.
\end{theorem}

\begin{remark}[Batching Security]
    The security of this reduction depends on the degree and non-degeneracy properties of the polynomial map $E$. The specific error bounds depend on the structure of $E$ and require careful analysis of the polynomial's properties.
\end{remark}

\subsection{Claim Reduction: ReduceClaim}

A fundamental building block for constructing complex proof systems is the ability to locally reduce one type of claim to another. The ReduceClaim reduction provides this functionality through mappings between statement and witness types.

\begin{definition}[ReduceClaim Reduction]
    \label{def:reduceclaim_reduction}
    The ReduceClaim reduction is a zero-round protocol that transforms claims via explicit mappings:
    \begin{itemize}
        \item \textbf{Protocol specification}: $\pSpec := []$ (no messages exchanged)
        \item \textbf{Statement mapping}: $\mathsf{mapStmt} : \StmtIn \to \StmtOut$
        \item \textbf{Witness mapping}: $\mathsf{mapWit} : \WitIn \to \WitOut$
        \item \textbf{Prover}: Applies both mappings to input statement and witness
        \item \textbf{Verifier}: Applies statement mapping to input statement
        \item \textbf{Input relation}: $R_{\mathsf{in}} : \StmtIn \to \WitIn \to \Prop$
        \item \textbf{Output relation}: $R_{\mathsf{out}} : \StmtOut \to \WitOut \to \Prop$
        \item \textbf{Relation condition}: $R_{\mathsf{in}}(\mathsf{stmt}, \mathsf{wit}) \iff R_{\mathsf{out}}(\mathsf{mapStmt}(\mathsf{stmt}), \mathsf{mapWit}(\mathsf{wit}))$
    \end{itemize}
    \lean{ReduceClaim.reduction}
    \uses{def:reduction}
\end{definition}

\begin{theorem}[ReduceClaim Perfect Completeness]
    The ReduceClaim reduction satisfies perfect completeness when the relation condition holds.
    \lean{ReduceClaim.reduction_completeness}
    \uses{def:reduceclaim_reduction}
\end{theorem}

\begin{definition}[ReduceClaim Oracle Reduction]
    \label{def:reduceclaim_oracle_reduction}
    The oracle version additionally handles oracle statements through an embedding:
    \begin{itemize}
        \item \textbf{Oracle statement mapping}: Embedding $\mathsf{embedIdx} : \iota_{\mathsf{out}} \hookrightarrow \iota_{\mathsf{in}}$
        \item \textbf{Type compatibility}: $\OStmtIn(\mathsf{embedIdx}(i)) = \OStmtOut(i)$ for all $i$
        \item \textbf{Oracle embedding}: Maps output oracle indices to corresponding input oracle indices
    \end{itemize}
    \lean{ReduceClaim.oracleReduction}
    \uses{def:oracle_reduction, def:reduceclaim_reduction}
\end{definition}

\begin{remark}[ReduceClaim Oracle Completeness]
    The oracle version's completeness proof is currently under development in the Lean formalization.
    % TODO: Complete when oracleReduction_completeness is implemented
\end{remark}

\subsection{Claim Verification: CheckClaim}

Another essential building block is the ability to verify that a given predicate holds for a statement without requiring additional witness information.

\begin{definition}[CheckClaim Reduction]
    \label{def:checkclaim_reduction}
    The CheckClaim reduction is a zero-round protocol that verifies predicates:
    \begin{itemize}
        \item \textbf{Protocol specification}: $\pSpec := []$ (no messages exchanged)
        \item \textbf{Predicate}: $\mathsf{pred} : \StmtIn \to \Prop$ (decidable)
        \item \textbf{Prover}: Simply stores and outputs the input statement with unit witness
        \item \textbf{Verifier}: Checks $\mathsf{pred}(\mathsf{stmt})$ and outputs statement if successful
        \item \textbf{Input relation}: $R_{\mathsf{in}}(\mathsf{stmt}, ()) := \mathsf{pred}(\mathsf{stmt})$
        \item \textbf{Output relation}: $R_{\mathsf{out}}(\mathsf{stmt}, ()) := \mathsf{True}$ (trivial after verification)
    \end{itemize}
    \lean{CheckClaim.reduction}
    \uses{def:reduction}
\end{definition}

\begin{theorem}[CheckClaim Perfect Completeness]
    The CheckClaim reduction satisfies perfect completeness.
    \lean{CheckClaim.reduction_completeness}
    \uses{def:checkclaim_reduction}
\end{theorem}

\begin{definition}[CheckClaim Oracle Reduction]
    \label{def:checkclaim_oracle_reduction}
    The oracle version handles predicates that require oracle access:
    \begin{itemize}
        \item \textbf{Oracle predicate}: $\mathsf{pred} : \StmtIn \to \OracleComp[\OStmtIn]_{\mathcal{O}} \Prop$
        \item \textbf{Never-fails condition}: $\mathsf{pred}(\mathsf{stmt})$ never fails for any statement
        \item \textbf{Oracle computation}: Verifier executes oracle computation to check predicate
        \item \textbf{Input relation}: Defined via oracle simulation of the predicate
    \end{itemize}
    \lean{CheckClaim.oracleReduction}
    \uses{def:oracle_reduction, def:checkclaim_reduction}
\end{definition}

\begin{theorem}[CheckClaim Oracle Perfect Completeness]
    The CheckClaim oracle reduction satisfies perfect completeness.
    \lean{CheckClaim.oracleReduction_completeness}
    \uses{def:checkclaim_oracle_reduction}
\end{theorem}

\begin{remark}[CheckClaim Security Analysis]
    The round-by-round knowledge soundness proofs for both reduction and oracle versions are currently under development in the Lean formalization.
\end{remark}

\subsection{Claim Transmission: SendClaim}

The SendClaim reduction enables a prover to transmit a claim (oracle statement) to the verifier, who then verifies a relationship between the original and transmitted claims.

\begin{definition}[SendClaim Oracle Reduction]
    \label{def:sendclaim_oracle_reduction}
    The SendClaim reduction is a one-round protocol for claim transmission:
    \begin{itemize}
        \item \textbf{Protocol specification}: $\pSpec := [(\PtoVdir, \OStmtIn)]$ (single oracle message)
        \item \textbf{Input}: Statement and single oracle statement (via $\mathsf{Unique}$ index type)
        \item \textbf{Prover}: Sends the input oracle statement as protocol message
        \item \textbf{Verifier}: Executes oracle computation $\mathsf{relComp} : \StmtIn \to \OracleComp[\OStmtIn]_{\mathcal{O}} \Unit$
        \item \textbf{Output oracle statements}: Sum type $\OStmtIn \oplus \OStmtIn$ containing both original and transmitted claims
        \item \textbf{Output relation}: $R_{\mathsf{out}}((), \mathsf{oracles}) := \mathsf{oracles}(\mathsf{inl}) = \mathsf{oracles}(\mathsf{inr})$
    \end{itemize}
    \lean{SendClaim.oracleReduction}
    \uses{def:oracle_reduction}
\end{definition}

\begin{theorem}[SendClaim Perfect Completeness]
    The SendClaim oracle reduction satisfies perfect completeness when the input relation matches the oracle computation requirement.
    \lean{SendClaim.completeness}
    \uses{def:sendclaim_oracle_reduction}
\end{theorem}

\begin{remark}[SendClaim Development Status]
    The SendClaim reduction is currently under active development in the Lean formalization. Several components including the verifier embedding and completeness proof require further implementation. The current version represents a specialized case that may be generalized in future iterations.
    % TODO: Complete implementation as noted in the Lean code
\end{remark}


\section{The Sum-Check Protocol}\label{sec:sumcheck}

This section documents our formalization of the sum-check protocol.
We first describe the sum-check protocol as it is typically described in the literature, and then
present a modular description that maximally relies on our general oracle reduction framework.

\subsection{Standard Description}\label{sec:sumcheck_standard}

\subsubsection{Protocol Parameters}

The sum-check protocol is parameterized by the following:
\begin{itemize}
    \item $R$: the underlying ring (for soundness, required to be finite and an integral domain)
    \item $n \in \mathbb{N}$: the number of variables (and the number of rounds for the protocol)
    \item $d \in \mathbb{N}$: the individual degree bound for the polynomial
    \item $\mathcal{D}: \{0, 1, \ldots, m-1\} \hookrightarrow R$: the set of $m$ evaluation points for each variable, represented as an injection. The image of $\mathcal{D}$ as a finite subset of $R$ is written as $\text{Image}(\mathcal{D})$.
    \item $\mathcal{O}$: the set of underlying oracles (e.g., random oracles) that may be needed for other reductions. However, the sum-check protocol itself does \emph{not} use any oracles.
\end{itemize}

\subsubsection{Input and Output Statements}

For the standard description of the sum-check protocol, we specify the complete input and output data:

\paragraph{Input Statement.} The claimed sum $T \in R$.

\paragraph{Input Oracle Statement.} The polynomial $P \in R[X_0, X_1, \ldots, X_{n-1}]_{\leq d}$ of $n$ variables with bounded individual degrees $d$.

\paragraph{Input Witness.} None (the unit type).

\paragraph{Input Relation.} The sum-check relation:
\[
\sum_{x \in (\text{Image}(\mathcal{D}))^n} P(x) = T
\]

\paragraph{Output Statement.} The claimed evaluation $e \in R$ and the challenge vector $(r_0, r_1, \ldots, r_{n-1}) \in R^n$.

\paragraph{Output Oracle Statement.} The same polynomial $P \in R[X_0, X_1, \ldots, X_{n-1}]_{\leq d}$.

\paragraph{Output Witness.} None (the unit type).

\paragraph{Output Relation.} The evaluation relation:
\[
P(r_0, r_1, \ldots, r_{n-1}) = e
\]

\subsubsection{Protocol Description}

The sum-check protocol proceeds in $n$ rounds of interaction between the prover and verifier. The protocol reduces the claim that a multivariate polynomial $P$ sums to a target value $T$ over the domain $(\text{Image}(\mathcal{D}))^n$ to the claim that $P$ evaluates to a specific value $e$ at a random point $(r_0, r_1, \ldots, r_{n-1})$.

In each round, the prover sends a univariate polynomial of bounded degree, and the verifier responds with a random challenge. The verifier performs consistency checks by querying the polynomial oracle at specific evaluation points. After $n$ rounds, the protocol terminates with an output statement asserting that $P(r_0, r_1, \ldots, r_{n-1}) = e$, where the challenges $(r_0, r_1, \ldots, r_{n-1})$ are the random values chosen by the verifier during the protocol execution.

The protocol is described as an oracle reduction, where the polynomial $P$ is accessed only through evaluation queries rather than being explicitly represented.

\subsubsection{Security Properties}

We prove the following security properties for the sum-check protocol:

\begin{theorem}[Perfect Completeness]
    \label{thm:sumcheck_perfect_completeness}
    The sum-check protocol satisfies perfect completeness. That is, for any valid input statement and oracle statement satisfying the input relation, the protocol accepts with probability 1.
\end{theorem}

\begin{theorem}[Knowledge Soundness]
    \label{thm:sumcheck_knowledge_soundness}
    The sum-check protocol satisfies knowledge soundness. The soundness error is bounded by $n \cdot d / |R|$, where $n$ is the number of rounds, $d$ is the degree bound, and $|R|$ is the size of the field.
\end{theorem}

\begin{theorem}[Round-by-Round Knowledge Soundness]
    \label{thm:sumcheck_rbr_knowledge_soundness_standard}
    The sum-check protocol satisfies round-by-round knowledge soundness with respect to an appropriate state function (to be specified). Each round maintains the security properties compositionally, allowing for modular security analysis.
\end{theorem}

\subsubsection{Implementation Notes}

Our formalization includes several important implementation considerations:

\paragraph{Oracle Reduction Level.} This description of the sum-check protocol stays at the \textbf{oracle reduction} level, describing the protocol before being compiled with concrete cryptographic primitives such as polynomial commitment schemes for $P$. The oracle model allows us to focus on the logical structure and security properties of the protocol without being concerned with the specifics of how polynomial evaluations are implemented or verified.

\paragraph{Abstract Protocol Description.} The protocol description above does not consider implementation details and optimizations that would be necessary in practice. For instance, we do not address computational efficiency, concrete polynomial representations, or specific algorithms for polynomial evaluation. This abstraction allows us to establish the fundamental security properties that any concrete implementation must preserve.

\paragraph{Degree Constraints.} To represent sum-check as a series of Interactive Oracle Reductions (IORs), we implicitly constrain the degree of the polynomials via using subtypes such as $R[X]_{\leq d}$ and appropriate multivariate polynomial degree bounds. This is necessary because the oracle verifier only gets oracle access to evaluating the polynomials, but does not see the polynomials in the clear.

\paragraph{Polynomial Commitments.} When this protocol is compiled to an interactive proof (rather than an oracle reduction), the corresponding polynomial commitment schemes will enforce that the declared degree bounds hold, by letting the (non-oracle) verifier perform explicit degree checks.

\paragraph{Formalization Alignment.} \textbf{TODO:} Align the sum-check protocol formalization in Lean to use $n$ variables and $n$ rounds (as in this standard description) rather than $n+1$ variables and $n+1$ rounds. This should be achievable by refactoring the current implementation to better match the standard presentation.

\subsubsection{Future Extensions}

Several generalizations are considered for future work:

\begin{itemize}
    \item \textbf{Variable Degree Bounds:} Generalize to $d : \{0, 1, \ldots, n+1\} \to \mathbb{N}$ and $\mathcal{D} : \{0, 1, \ldots, n+1\} \to (\{0, 1, \ldots, m-1\} \hookrightarrow R)$, allowing different degree bounds and summation domains for each variable.

    \item \textbf{Restricted Challenge Domains:} Generalize the challenges to come from suitable subsets of $R$ (e.g., subtractive sets), rather than the entire domain. This modification is used in lattice-based protocols.

    \item \textbf{Module-based Sum-check:} Extend to sum-check over modules instead of just rings. This would require extending multivariate polynomial evaluation to modules, defining something like $\text{evalModule} : (R^n \to M) \to R[X_0, \ldots, X_{n-1}] \to M$ where $M$ is an $R$-module.
\end{itemize}

The sum-check protocol, as described in the original paper and many expositions thereafter, is a
protocol to reduce the claim that \[ \sum_{x \in \{0, 1\}^n} P(x) = c, \] where $P$ is an
$n$-variate polynomial of certain individual degree bounds, and $c$ is some field element, to the
claim that \[ P(r) = v, \] for some claimed value $v$ (derived from the protocol transcript), where
$r$ is a vector of random challenges from the verifier sent during the protocol.

In our language, the initial context of the sum-check protocol is the pair $(P, c)$, where $P$ is an
oracle input and $c$ is public. The protocol proceeds in $n$ rounds of interaction, where in each
round $i$ the prover sends a univariate polynomial $s_i$ of bounded degree and the verifier sends a
challenge $r_i \gets \mathbb{F}$. The honest prover would compute \[ s_i(X) = \sum_{x \in \{0,
1\}^{n - i - 1}} P(r_1, \ldots, r_{i - 1}, X, x), \] and the honest verifier would check that
$s_i(0) + s_i(1) = s_{i - 1}(r_{i - 1})$, with the convention that $s_0(r_0) = c$.

\subsection{Modular Description}\label{sec:sumcheck_modular}

\subsubsection{Round-by-Round Analysis}

Our modular approach breaks down the sum-check protocol into individual rounds, each of which can be analyzed as a two-message Interactive Oracle Reduction. This decomposition allows us to prove security properties compositionally and provides a more granular understanding of the protocol's structure.

\paragraph{Round-Specific Statements.} For the $i$-th round, where $i \in \{0, 1, \ldots, n\}$, the statement contains:
\begin{itemize}
    \item $\text{target} \in R$: the target value for sum-check at this round
    \item $\text{challenges} \in R^i$: the list of challenges sent from the verifier to the prover in previous rounds
\end{itemize}

The oracle statement remains the same polynomial $P \in R[X_0, X_1, \ldots, X_{n-1}]_{\leq d}$.

\paragraph{Round-Specific Relations.} The sum-check relation for the $i$-th round checks that:
\[
\sum_{x \in (\text{Image}(\mathcal{D}))^{n-i}} P(\text{challenges}, x) = \text{target}
\]

Note that when $i = n$, this becomes the output statement of the sum-check protocol, checking that $P(\text{challenges}) = \text{target}$.

\subsubsection{Individual Round Protocol}

For $i = 0, 1, \ldots, n-1$, the $i$-th round of the sum-check protocol consists of the following:

\paragraph{Step 1: Prover's Message.} The prover sends a univariate polynomial $p_i \in R[X]_{\leq d}$ of degree at most $d$. If the prover is honest, then:
\[
p_i(X) = \sum_{x \in (\text{Image}(\mathcal{D}))^{n-i}} P(\text{challenges}_0, \ldots, \text{challenges}_{i-1}, X, x)
\]

Here, $P(\text{challenges}_0, \ldots, \text{challenges}_{i-1}, X, x)$ is the polynomial $P$ evaluated at the concatenation of:
\begin{itemize}
    \item the prior challenges $\text{challenges}_0, \ldots, \text{challenges}_{i-1}$
    \item the $i$-th variable as the new indeterminate $X$
    \item the remaining values $x \in (\text{Image}(\mathcal{D}))^{n-i}$
\end{itemize}

In the oracle protocol, this polynomial $p_i$ is turned into an oracle for which the verifier can query evaluations at arbitrary points.

\paragraph{Step 2: Verifier's Challenge.} The verifier sends the $i$-th challenge $r_i$ sampled uniformly at random from $R$.

\paragraph{Step 3: Verifier's Check.} The (oracle) verifier performs queries for the evaluations of $p_i$ at all points in $\text{Image}(\mathcal{D})$, and checks that:
\[
\sum_{x \in \text{Image}(\mathcal{D})} p_i(x) = \text{target}
\]

If the check fails, the verifier outputs $\texttt{failure}$. Otherwise, it outputs a statement for the next round as follows:
\begin{itemize}
    \item $\text{target}$ is updated to $p_i(r_i)$
    \item $\text{challenges}$ is updated to the concatenation of the previous challenges and $r_i$
\end{itemize}

\subsubsection{Single Round Security Analysis}

\begin{definition}[Single Round Protocol]
    \label{def:sumcheck_single_round}
    The $i$-th round of sum-check consists of:
    \begin{enumerate}
        \item \textbf{Input:} A statement containing target value and prior challenges, along with an oracle for the multivariate polynomial
        \item \textbf{Prover's message:} A univariate polynomial $p_i \in R[X]_{\leq d}$
        \item \textbf{Verifier's challenge:} A random element $r_i \gets R$
        \item \textbf{Output:} An updated statement with new target $p_i(r_i)$ and extended challenges
    \end{enumerate}
\end{definition}

\begin{theorem}[Single Round Completeness]
    \label{thm:sumcheck_single_round_completeness}
    Each individual round of the sum-check protocol is perfectly complete.
\end{theorem}

\begin{theorem}[Single Round Soundness]
    \label{thm:sumcheck_single_round_soundness}
    Each individual round of the sum-check protocol is sound with error probability at most $d / |R|$, where $d$ is the degree bound and $|R|$ is the size of the field.
\end{theorem}

\begin{theorem}[Round-by-Round Knowledge Soundness]
    \label{thm:sumcheck_rbr_knowledge_soundness}
    The sum-check protocol satisfies round-by-round knowledge soundness. Each individual round can be analyzed independently, and the soundness error in each round is bounded by $d / |R|$.
\end{theorem}

\subsubsection{Virtual Protocol Decomposition}

We now proceed to break down this protocol into individual messages, and then specify the predicates that should hold before and after each message is exchanged.

First, it is clear that we can consider each round in isolation. In fact, each round can be seen as an instantiation of the following simpler "virtual" protocol:

\begin{definition}
    \label{def:virtual_sumcheck_round_protocol}
    \begin{enumerate}
        \item In this protocol, the context is a pair $(p, d)$, where $p$ is now a \emph{univariate} polynomial of bounded degree. The predicate / relation is that $p(0) + p(1) = d$.
        \item The prover first sends a univariate polynomial $s$ of the same bounded degree as $p$. In
        the honest case, it would just send $p$ itself.
        \item The verifier samples and sends a random challenge $r \gets R$.
        \item The verifier checks that $s(0) + s(1) = d$. The predicate on the resulting output context
        is that $p(r) = s(r)$.
    \end{enumerate}
\end{definition}

The reason why this simpler protocol is related to a sum-check round is that we can \emph{emulate} the simpler protocol using variables in the context at the time:
\begin{itemize}
    \item The univariate polynomial $p$ is instantiated as $\sum_{x \in (\text{Image}(\mathcal{D}))^{n - i - 1}} P(r_0, \ldots, r_{i - 1}, X, x)$.
    \item The scalar $d$ is instantiated as $T$ if $i = 0$, and as $s_{i - 1}(r_{i - 1})$ otherwise.
\end{itemize}

It is "clear" that the simpler protocol is perfectly complete. It is sound (and since there is no
witness, also knowledge sound) since by the Schwartz-Zippel Lemma, the probability that $p \ne s$
and yet $p(r) = s(r)$ for a random challenge $r$ is at most the degree of $p$ over the size of the
field.

\begin{theorem}
    The virtual sum-check round protocol is sound.
    \label{thm:virtual_sumcheck_round_protocol_sound}
    \uses{def:soundness, def:virtual_sumcheck_round_protocol, thm:schwartz_zippel}
\end{theorem}

Note that there is no witness, so knowledge soundness follows trivially from soundness.

\begin{theorem}
    The virtual sum-check round protocol is knowledge sound.
    \label{thm:virtual_sumcheck_round_protocol_knowledge_sound}
    \uses{def:knowledge_soundness, thm:virtual_sumcheck_round_protocol_sound}
\end{theorem}

Moreover, we can define the following state function for the simpler protocol:
\begin{enumerate}
    \item The initial state function is the same as the predicate on the initial context, namely that
    $p(0) + p(1) = d$.
    \item The state function after the prover sends $s$ is the predicate that $p(0) + p(1) = d$ and
    $s(0) + s(1) = d$. Essentially, we add in the verifier's check.
    \item The state function for the output context (after the verifier sends $r$) is the predicate that $s(0) + s(1) = d$ and $p(r) = s(r)$.
\end{enumerate}
Seen in this light, it should be clear that the simpler protocol satisfies round-by-round soundness.

\begin{theorem}
    The virtual sum-check round protocol is round-by-round sound.
    \label{thm:virtual_sumcheck_round_protocol_rbr_sound}
    \uses{def:round_by_round_soundness, thm:virtual_sumcheck_round_protocol_knowledge_sound}
\end{theorem}

In fact, we can break down this simpler protocol even more: consider the two sub-protocols that each
consists of a single message. Then the intermediate state function is the same as the predicate on
the intermediate context, and is given in a "strongest post-condition" style where it incorporates
the verifier's check along with the initial predicate. We can also view the final state function as
a form of "canonical" post-condition, that is implied by the previous predicate except with small
probability.


\section{The Spartan Protocol}\label{sec:spartan}

\subsection{Preliminaries}

R1CS relation:



\subsection{Description in Paper}

\subsection{Formalization using IOR Composition}


% Copyright (c) 2025 ZKLib Contributors. All rights reserved.
% Released under Apache 2.0 license as described in the file LICENSE.
% Authors: Poulami Das (Least Authority)

\section{Stir}

\subsection{Tools for Reed Solomon codes}
\subsubsection{Random linear combination as a proximity generator}\label{sec:proximity_gap}

\begin{theorem}\label{thm:proximity_gap}
\lean{proximity_gap}
\uses{def:distance_from_code,def:reed_solomon_code,def:list_close_codewords}
    Let $\code := \rscode[\field,\evaldomain,\degree]$ be a Reed Solomon code with rate $\rate:=\frac{\degree}{|\evaldomain|}$ and let $B^*(\rate):=\sqrt{\rate}$. For every $\distance \in (0,1 - B^*(\rate))$ and functions $f_0,\ldots,f_{m-1} : \evaldomain \to \field$, if
    \[
    \Pr_{r\leftarrow\field}\!\Bigl[
      \Delta\Bigl(\sum_{j=0}^{m-1} r^j\cdot {f_j},\rscode[\field,\evaldomain,\degree]\Bigr)
      \le \delta
    \Bigr]>\err^*(\degree,\rate,\distance,m),
    \]
    then there exists a subset $S \subseteq \evaldomain$ with $|S| \ge (1 - \delta)\cdot|L|$,
    and for every $i \in [m]$, there exists $u \in \rscode[\field,\evaldomain,\degree]$ such that $f_i(S) = u(S)$.
    
    \medskip
    
    \noindent
    Above, $\err^*(\degree,\rate,\distance,m)$ is defined as follows:
    \begin{itemize}
        \item if $\distance \in \left(0,\frac{1-\rate}{2}\right]$ then
            \[
                \err^*(\degree,\rate,\distance,m)=\frac{(m-1)\cdot \degree}{\rate\cdot|\field|}
            \]
        \item if $\distance \in \Bigl(\frac{1-\rate}{2}, 1-\sqrt{\rate}\Bigr)$ then
        \[
            \err^*(\degree,\rate,\distance,m)=\frac{(m-1)\cdot {\degree}^2}{|\field|\cdot{\Bigl(2\cdot\min\{1-\sqrt{\rate}-\distance,\frac{\sqrt{\rate}}{20}\}\Bigr)}^7}
        \]
    \end{itemize}
\end{theorem}

\subsubsection{Univariate Function Quotienting}\label{sec:quotienting}

In the following, we start by defining the \emph{quotient} of a univariate function.
\begin{definition}\label{def:quotient}
\lean{Quotienting.funcQuotient}
    Let $f:\evaldomain\to\field$ be a function, $S\subseteq\field$ be a set, and $\mathsf{Ans},\mathsf{Fill}: S\rightarrow\field$ be functions. Let $\hat{\mathsf{Ans}}\in\field^{<|S|}[X]$ be the (unique) polynomial with $\hat{\mathsf{Ans}}(x)=\mathsf{Ans}(x)$ for every $x\in S$, and let $\hat{V}_S\in\field^{<|S|+1}[X]$ be the unique non-zero polynomial with $\hat{V}_S(x)=0$ for every $x\in S$.
    The \emph{quotient function} $\mathsf{Quotient}\bigl(f,S,\mathsf{Ans},\mathsf{Fill}\bigr): \evaldomain\to\field$
    is defined as follows:

    \[
    \forall x \in \evaldomain, \quad
    \mathsf{Quotient}\bigl(f, S, \mathsf{Ans}, \mathsf{Fill} \bigr)(x)
    :=
    \left\{
    \begin{array}{ll}
        \mathsf{Fill}(x)
        & \text{if } x \in S \\[6pt]
        \dfrac{f(x) - \hat{\mathsf{Ans}}(x)}{\hat{V}_S(x)}
        & \text{otherwise}
    \end{array}
    \right.
    \]

\end{definition}

Next we define the polynomial quotient operator, which quotients a polynomial relative to its output on evaluation points. The polynomial quotient is a polynomial of lower degree.

\begin{definition}\label{def:poly_quotient}
\lean{Quotienting.polyQuotient}
    Let $\hat{f}\in\field^{<\degree}[X]$ be a polynomial and $S\subseteq\field$ be a set, let $\hat{V}_S\in\field^{<|S|+1}[X]$ be the unique non-zero polynomial with $\hat{V}_S(x)=0$ for every $x\in S$. The \emph{polynomial quotient} $\mathsf{PolyQuotient}(\hat{f},S)\in\field^{<d-|S|}[X]$ is defined as follows:
    \[
            \mathsf{PolyQuotient}(\hat{f},S)(X):=\frac{\hat{f}(X)-\hat{\mathsf{Ans}}(X)}{\hat{V}_S(X)}
    \]
where $\hat{Ans}\in\field^{<|S|}[X]$ is the unique non-zero polynomial with $\hat{Ans}(x)=\hat{f}(x)$ for every $x \in S$.
\end{definition}

The following lemma, implicit in prior works, shows that if the function is ``quotiented by the wrong value'', then its quotient is far from low-degree.

\begin{lemma}\label{lemma:quotienting}
\lean{Quotienting.quotienting}
\uses{def:quotient,def:reed_solomon_code,def:distance_from_code,def:list_close_codewords}
    Let $f:\evaldomain\rightarrow\field$ be a function, $\degree\in\N$ be the degree parameter, $\distance\in(0,1)$
    be a distance parameter, $S\subseteq\field$ be a set with $|S|<\degree$, and $\mathsf{Ans},\mathsf{Fill}:S\rightarrow\field$ are functions. Suppose that for every $u\in\listcode(f,\degree,\distance)$ there exists $x\in S$ with $\hat{u}(x)\neq\mathsf{Ans}(x)$. Then 

    \[
            \Delta(\mathsf{Quotient}(f,S,\mathsf{Ans},\mathsf{Fill}),\rscode[\field,\evaldomain,\degree-|S|])+\frac{|T|}{|\evaldomain|}>\distance,
    \]
    where $T:=\{x\in\evaldomain\cap S: \hat{\mathsf{Ans}}(x)\neq f(x)\}$.
\end{lemma}

\subsubsection{Out of domain sampling}\label{sec:out_of_domain_smpl}
\begin{lemma}\label{lemma:out_of_domain_smpl}
\lean{OutOfDomSmpl.out_of_dom_smpl_1, OutOfDomSmpl.out_of_dom_smpl_2}
\uses{def:reed_solomon_code,def:list_decodable,def:list_close_codewords}
    Let $f:\evaldomain\rightarrow\field$ be a function, $\degree\in\N$ be a degree parameter, $s\in\N$ be a repetition parameter, and $\distance\in[0,1]$ be a distance parameter. If $\rscode[\field,\evaldomain,\degree]$ be $(\degree,l)$-list decodable then
        \[
        \begin{array}{rcl}
        \Pr_{\substack{r_0, \ldots, r_{s-1} \gets \field \setminus \evaldomain}} 
        \left[
        \begin{array}{c}
            \exists \text{ distinct } u, u' \in \listcode(f, \degree, \distance): \\
            \forall i \in [s],\ \hat{u}(r_i) = \hat{u}'(r_i)
        \end{array}
        \right]
        & \leq & \binom{l}{2} \cdot \left( \frac{\degree - 1}{|\field| - |\evaldomain|} \right)^s \\
        & \leq & \left( \frac{l^2}{2} \right) \cdot \left( \frac{\degree}{|\field| - |\evaldomain|} \right)^s
        \end{array}
        \]

        
        
\end{lemma}

\subsubsection{Folding univariate functions}\label{sec:folding_uf}
STIR relies on $k$-wise folding of functions and polynomials - this is similar to prior works, although presented in a slightly different form. As shown below, folding a function preserves proximity from the Reed-Solomon code with high probability. The folding operator is based on the following fact, decomposing univariate polynomials into bivariate ones.

\begin{lemma}\label{fact:poly_folding} 
\lean{Folding.exists_unique_bivariate,Folding.degree_bound_bivariate}
Given a polynomial $\hat{q}\in\field[X]$:
\begin{itemize}
        \item For every univariate polynomial $\hat{f}\in\field[X]$, there exists a unique bivariate polynomial $\hat{Q}\in\field[X,Y]$ with:
        \[
          \polydeg_X(\hat{Q}) := \left\lfloor \frac{\polydeg(\hat{f})}{\polydeg(\hat{q})} \right\rfloor,\quad \polydeg_Y(\hat{Q}) < \polydeg(\hat{q})
        \]
        such that $\hat{f}(Z)=\hat{Q}(\hat{q}(Z),Z)$. Moreover, $\hat{Q}$ can be computed efficiently given $\hat{f}$ and $\hat{q}$. Observe that if $\polydeg(\hat{f})<t\cdot\polydeg(\hat{q})$ then
        $\polydeg(\hat{Q})<t$.
        
        \item For every $\hat{Q}[X,Y]$ with $\polydeg_X(\hat{Q})<t$ and $\polydeg_Y(\hat{Q})<\polydeg(\hat{q})$, the polynomial $\hat{f}(Z)=\hat{Q}(\hat{q}(Z),Z)$ has degree $\polydeg(\hat{f})<t\cdot\polydeg(\hat{q})$.
\end{itemize}
\end{lemma}

Below, we define folding of a polynomial followed by folding of a function.
\begin{definition}\label{def:poly_folding}
\lean{Folding.polyFold}
\uses{fact:poly_folding}
    Given a polynomial $\hat{f}\in\field^{<\degree}[X]$, a folding parameter $k\in\N$ and $r\in\field$, we define a polynomial $\mathsf{PolyFold}(\hat{f},k,r)\in\field^{\degree/k}[X]$ as follows. Let $\hat{Q}[X,Y]$ be the bivariate polynomial derived from $\hat{f}$ using Fact~\ref{fact:poly_folding} with $\hat{q}(X):=X^k$. Then $\mathsf{PolyFold}(\hat{f},k,r)(X):=\hat{Q}(X,r)$.
\end{definition}

\begin{definition}\label{def:fn_folding}
\lean{Folding.fold}
Let $f:\evaldomain\rightarrow\field$ be a function, $k\in\N$ a folding parameter and $\alpha\in\field$. For every $x\in{\evaldomain}^k$, let $\hat{p}_x\in\field^{<k}[X]$ be the polynomial where $\hat{p}_x(y)=f(y)$ for every $y\in{\evaldomain}$ such that $y^k=x$. We define $\mathsf{Fold}(f,k,\alpha):\evaldomain\rightarrow\field$ as follows.
\[
    \mathsf{Fold}(f,k,\alpha):=\hat{p}_x(\alpha).
\]
In order to compute $\mathsf{Fold}(f,k,\alpha)(x)$ it suffices to interpolate the $k$ values $\{f(y):y\in\evaldomain\text{ s.t. }y^k=x\}$ into the polynomial $\hat{p}_x$ and evaluate this polynomial at $\alpha$.
\end{definition}

The following lemma shows that the distance of a function is preserved under folding. If a functions $f$ has distance $\distance$ to a Reed-Solomon code then, with high probability over the choice of folding randomness, its folding also has a distance of $\distance$ to the ``$k$-wise folded'' Reed-Solomon code.

\begin{lemma}\label{lemma:folding}
\lean{Folding.folding}
\uses{def:reed_solomon_code,def:fn_folding,def:distance_from_code}
    For every function $f:\evaldomain\rightarrow\field$, degree parameter $\degree\in\N$, folding parameter $k\in\N$, distance parameter $\distance\in(0,\min\{{\Delta(\mathsf{Fold}[f,k,r^{\mathsf{fold}}],\rscode[\field,{\evaldomain}^k, \degree/k]),1-{\mathsf{B}}^*(\rate)}\})$, letting $\rate:=\frac{\degree}{|\evaldomain|}$,
    \[
        \Pr_{r^{\mathsf{fold}}\gets\field}[\Delta(\mathsf{Fold}[f,k,r^{\mathsf{fold}}],\rscode[\field,{\evaldomain}^k, \degree/k])<\distance]>\err^*(\degree/k,\rate,\distance,k).
    \]
    Above, ${\mathsf{B}}^*$ and ${\err}^*$ are the proximity bound and error (respectively) described in Section~\ref{sec:proximity_gap}.
\end{lemma}

\subsubsection{Combine functions of varying degrees}\label{sec:combine}
We show a new method for combining functions of varying degrees with minimal proximity require- ments using geometric sums. We begin by recalling a fact about geometric sums.

\begin{lemma}\label{fact:geometric_sum}
\lean{Combine.geometric_sum_units}
    Let $\field$ be a field, $r\in\field$ be a field element, $a\in\N$ be a natural number. Then
    \[
    \sum_{i=0}^{a} r^i :=
    \left\{
    \begin{array}{ll}
        \displaystyle\left(\frac{1 - r^{a+1}}{1 - r}\right)
        & \text{if } r \neq 1 \\[6pt]
        a + 1
        & \text{if } r = 1
    \end{array}
    \right.
    \]

\end{lemma}

\begin{definition}\label{def:combine}
\lean{Combine.combineInterm,Combine.combineFinal}
    Given target degree $\degree^*\in\N$, shifting parameter $r\in\field$, functions $f_0,\ldots,f_{m-1}:\evaldomain\rightarrow\field$, and degrees $0\leq \degree_0,\ldots,\degree_{m-1}\leq {\degree}^*$, we define $\combine(\degree^*,r,(f_0,\degree_0),\ldots,(f_{m-1},\degree_{m-1})):\evaldomain\rightarrow\field$ as follows:
        \[
        \begin{array}{rcl}
        \combine(\degree^*, r, (f_0, \degree_0), \ldots, (f_{m-1}, \degree_{m-1}))(x)
        & := & \displaystyle\sum_{i=0}^{m-1} r_i \cdot f_i(x) \cdot \left( \sum_{l=0}^{\degree^* - \degree_i} (r \cdot x)^l \right) \\[10pt]
        & = & \left\{
        \begin{array}{ll}
            \displaystyle\sum_{i=0}^{m-1} r_i \cdot f_i(x) \cdot \left( \dfrac{1 - (x r)^{\degree^* - \degree_i + 1}}{1 - x r} \right)
            & \text{if } x \cdot r \neq 1 \\[8pt]
            \displaystyle\sum_{i=0}^{m-1} r_i \cdot f_i(x) \cdot (\degree^* - \degree_i + 1)
            & \text{if } x \cdot r = 1
        \end{array}
        \right.
        \end{array}
        \]

Above, $r_i:=r^{i-1+\sum_{j<i}(\degree^*-\degree_j)}$.
\end{definition}

\begin{definition}\label{def:deg_corr}
\lean{Combine.degreeCorrInterm,Combine.degreeCorrFinal}
\uses{def:combine,def:reed_solomon_code,def:distance_from_code}
    Given target degree $\degree^*\in\N$, shifting parameter $r\in\field$, function $f:\evaldomain\rightarrow\field$, and degree $0\leq\degree\leq\degree^*$, we define $\degcorr(\degree^*,r,f,\degree)$ as follows.
    \[
    \begin{array}{rcl}
    \degcorr(\degree^*, r, f, \degree)(x) 
    & := & \displaystyle f(x) \cdot \left( \sum_{l=0}^{\degree^* - \degree} (r \cdot x)^l \right) \\[10pt]
    & = & \left\{
    \begin{array}{ll}
        \displaystyle f(x) \cdot \left( \dfrac{1 - (x r)^{\degree^* - \degree_i + 1}}{1 - x r} \right)
        & \text{if } x \cdot r \neq 1 \\[8pt]
        \displaystyle f(x) \cdot (\degree^* - \degree_i + 1)
        & \text{if } x \cdot r = 1
    \end{array}
    \right.
    \end{array}
    \]


(Observe that $\degcorr(\degree^*,r,f,\degree)=\combine(\degree^*,r,(f,\degree))$.)
\end{definition}

Below it is shown that combining multiple polynomials of varying degrees can be done as long as the proximity error is bounded by $(\min{\{1-{\mathsf{B}}^*(\rate),1-\rate-1/|\evaldomain|\}})$.

\begin{lemma}\label{lemma:combine}
\lean{Combine.combine}
\uses{def:reed_solomon_code,def:combine,def:deg_corr,thm:proximity_gap}
    Let $\degree^*$ be a target degree, $f_0,\ldots,f_{m-1}:\evaldomain\rightarrow\field$ be functions, $0\leq \degree_0,\ldots,\degree_{m-1}\leq \degree^*$ be degrees, $\distance\in\min{\{1-{\mathsf{B}}^*(\rate),1-\rate-1/|\evaldomain|\}}$ be a distance parameter, where $\rate=\degree^*/|\evaldomain|$. If
    \[
        \Pr_{r\gets\field}[\Delta(\combine(\degree^*,r,(f_0,\degree_0),\ldots,(f_{m-1},\degree_{m-1})),\rscode[\field,\evaldomain,\degree^*])]>\err^*(\degree^*,\rate,\distance,m\cdot(\degree^*+1)-\sum_{i=0}^{m-1}\degree_i),
    \] 
    then there exists $S\subseteq \evaldomain$ with $|S|\geq(1-\distance)\cdot|\evaldomain|$, and
    \[
        \forall i\in[m-1],\exists u\in\rscode[\field,\evaldomain,\degree_i], f_i(S)=u(S).
    \]
    Note that this implies $\Delta(f_i,\rscode[\field,\evaldomain,\degree_i])<\distance$ for every $i$. Above, ${\mathsf{B}}^*$ and ${\err}^*$ are the proximity bound and error (respectively) described in the proximity gap theorem.
\end{lemma}

\subsection{Stir Main theorems}

\begin{theorem}[STIR Main Theorem]\label{thm:stir}
\lean{StirIOP.stir_main}
\uses{def:reed_solomon_code,lemma:rnd_by_rnd_soundness}
    Consider the following ingrediants:
    \begin{itemize}
        \item A security parameter $\lambda\in\N$.
        \item A Reed-Solomon code $\rscode[\field,\evaldomain,\degree]$ with $\rate:=\frac{\degree}{|\evaldomain|}$ where $\degree$ is a power of $2$, and $\evaldomain$ is a smooth domain.
        \item A proximity parameter $\distance\in(0,1-1.05\cdot\sqrt{\rate})$.
        \item A folding parameter $k\in\N$ that is power of $2$ with $k\geq 4$.
    \end{itemize}
If $|\field|=\Omega(\frac{\lambda\cdot2^\lambda\cdot\degree^2\cdot{|\evaldomain|}^2}{\log(1/\rate)})$, there is a public-coin IOPP for $\rscode[\field,\evaldomain,\degree]$ with the following parameters:
\begin{itemize}
    \item Round-by-round soundness error $2^{-\lambda}$.
    \item Round complexity: $M:=O(\log_k{\degree})$.
    \item Proof length: $|\evaldomain|+O_k(\log{\degree})$.
    \item Query complexity to the input: $\frac{\lambda}{-\log{(1-\distance)}}$.
    \item Query complexity to the proof strings: $O_k(\log{\degree}+\lambda\cdot\log{\Big(\frac{\log{\degree}}{\log{1/\rate}}\Big)})$.
\end{itemize}
\end{theorem}

\begin{lemma}\label{lemma:rnd_by_rnd_soundness}
\lean{StirIOP.stir_rbr_soundness}
\uses{def:reed_solomon_code,def:list_decodable,lemma:folding,lemma:out_of_domain_smpl,lemma:quotienting,lemma:combine}
    Consider $(\field,M,\degree,k_0,\ldots,k_M,\evaldomain_0,\ldots,\evaldomain_M,t_0,\ldots,t_M)$ and for every $i\in\{0,\ldots,M\}$, let $\degree_i:=\frac{\degree}{\prod_{j<i}k^j}$ and $\rate_i:=\degree_i/|\evaldomain_i|$. For every $f\notin\rscode[\field,\evaldomain_0,\degree_0]$ and every $\distance_0,\ldots,\distance_M$ where
    \begin{itemize}
        \item $\distance_0\in(0,\Delta(f,\rscode[\field,\evaldomain_0,\degree_0])]\cap(0,1-{\mathsf{B}}^*(\rate_0))$
        \item for every $0<i\leq M$: $\distance_i\in(0,\min{\{1-\rate_i-\frac{1}{|\evaldomain_i|},1-{\mathsf{B}^*(\rate_i)}\}})$, and
        \item for every $0<i\leq M$: $\rscode[\field,\evaldomain_i,\degree_i]$ is $(\distance_i,l_i)$-list decodable,
    \end{itemize}
    There exists an IOPP with above parameters, that has round-by-round soundness error $(\epsilon^{\mathsf{fold}},\epsilon^{\mathsf{out}}_1,\epsilon^{\mathsf{shift}}_1,\ldots,\epsilon^{\mathsf{out}}_M,\epsilon^{\mathsf{shift}}_M,\epsilon^{\mathsf{fin}})$ where:
    \begin{itemize}
        \item $\epsilon^{\mathsf{fold}}\leq\err^*(\degree_0/k_0,\rate_0,\distance_0,k_0)$.
        \item $\epsilon^{\mathsf{out}}_i\leq\frac{l^2_i}{2}\cdot{\big(\frac{\degree_i}{|\field|-|\evaldomain_i|}\big)}^s$
        \item $\epsilon^{\mathsf{shift}}_i\leq {(1-\distance_{i-1})}^{t_{i-1}}+\err^*(\degree_i,\rate_i,\distance_i,t_{i-1}+s)+\err^*(\degree_i/k_i,\rate_i,\distance_i,k_i)$.
        \item $\epsilon^{\mathsf{fin}}\leq{(1-\distance_M)}^{t_M}$.
    \end{itemize}
    Above, ${\mathsf{B}}^*$ and ${\err}^*$ are the proximity bound and error (respectively) described in Section~\ref{sec:proximity_gap}.
\end{lemma}



% Copyright (c) 2025 ZKLib Contributors. All rights reserved.
% Released under Apache 2.0 license as described in the file LICENSE.
% Authors: Poulami Das (Least Authority)

\section{Whir}

\subsection{Tools for Reed Solomon codes}

\subsubsection{Mutual Correlated Agreement as a Proximity Generator}

\begin{definition}\label{def:proximity_generator}
\lean{Generator.ProximityGenerator}
\uses{def:linear_code,def:distance_from_code}
Let $\code\subseteq \field^{\evaldomain}$ be a linear code. We say that $\mathsf{Gen}$ is a proximity generator for $\code$ with proximity bounds ${\bound}$ and $\err$ if the following implication holds for $f_0,\ldots,f_{\parl-1} : \evaldomain \rightarrow \field$ and $\delta\in(0,1-{\bound}(\rate,\parl))$. If
\begin{align*}
    \Pr_{r_0,\ldots,r_{\parl-1}\leftarrow \gen}[\Delta(\sum_{i\in[0,(\parl-1)]} r_i \cdot f_i, \code) \le \delta] > err(\code,\parl,\delta),
\end{align*}
then there exists $S\subseteq \evaldomain$, $|S|>(1-\delta)\cdot|\evaldomain|$, and
$\forall i \in [0, (\parl-1)]$, $\exists u \in \code, \forall x \in S$, $f_i(x)=u(x)$. 
\end{definition}

\begin{theorem}\label{thm:proximity_gap_whir}
\lean{RSGenerator.proximityGapTheorem}
\uses{def:proximity_generator,def:reed_solomon_code}
    Let $\code = \rscode[\field,\evaldomain,m]$ be a Reed Solomon code with rate $\rate = 2^m/|\evaldomain|$. $\gen(\alpha,\parl)=\{1,\alpha,\ldots,\alpha^{\parl-1}\}$ is a proximity generator for $\code$ with proximity bounds ${\bound}(\rate,\parl)=\sqrt{\rate}$ and $\err(C,\parl,\delta)$ defined below.
    \begin{itemize}
        \item if $\distance \in \left(0,\frac{1-\rate}{2}\right]$ then
            \[
                \err(\code,\parl,\delta)=\frac{(m-1)\cdot \degree}{\rate\cdot|\field|}
            \]
        \item if $\distance \in \Bigl(\frac{1-\rate}{2}, 1-\sqrt{\rate}\Bigr)$ then
        \[
            \err(\code,\parl,\delta)=\frac{(m-1)\cdot {\degree}^2}{|\field|\cdot{\Bigl(2\cdot\min\{1-\sqrt{\rate}-\distance,\frac{\sqrt{\rate}}{20}\}\Bigr)}^7}
        \]
    \end{itemize}
\end{theorem}

\begin{definition}\label{def:gen_mutual_corr_agreement}
    \lean{CorrelatedAgreement.genMutualCorrAgreement}
    \uses{def:proximity_generator}
    Let $\code$ be a linear code. We say that $\gen$ be a proximity generator with mutual correlated agreement with proximity bounds ${\bound}^\star$ and $\err^\star$, if for $f_0,\ldots,f_{\parl-1}:\evaldomain\rightarrow\field$ and $\delta\in(0,1-{\bound}^\star(\code,\parl))$ the following holds.
    \[
    \Pr_{(r_0, \ldots, r_{\parl-1}) \leftarrow \gen(\parl)} \left[
        \exists S \subseteq \evaldomain \;\; s.t.\;\;
        \begin{array}{l}
        |S| \geq (1 - \delta) \cdot |\evaldomain| \\
        \land\; \exists u \in \code, u(S) = \sum_{j \in [0,(\parl-1)]} r_j \cdot f_j(S) \\
        \land\; \exists i \in [0,(\parl-1)], \forall u' \in \code, u'(S) \neq f_i(S)
        \end{array}\right]
    \leq \err^\star(\code, \parl, \delta).
    \]
\end{definition}

\begin{lemma}\label{lemma:gen_mutual_corr_agreement}
\lean{CorrelatedAgreement.genMutualCorrAgreement_le_bound}
\uses{def:proximity_generator, def:gen_mutual_corr_agreement}
    Let $\code$ be a linear code with minimum distance $\delta_{\code}$ and let $\gen$ be a proximity generator for $\code$ with proximity bound ${\bound}$ and error $\err$. Then $\gen$ has mutual correlated agreement with proximity bound ${\bound}^\star(\code, \parl) = \min\{1 - \delta_{\code}/2, \bound(\code, \parl)\}$ and error $\err^\star(\code, \parl, \delta) := \err(\code, \parl, \delta)$.
\end{lemma}

\begin{lemma}
\lean{CorrelatedAgreement.genMutualCorrAgreement_rsc_le_bound}
\uses{lemma:gen_mutual_corr_agreement, def:reed_solomon_code}
    Let $\code := \rscode[\field, \evaldomain, m]$ be a Reed Solomon code with rate $\rate$. The function $\gen(\parl; \alpha) = (1, \alpha, \ldots, \alpha^{\parl - 1})$ is a proximity generator for $\code$ with mutual correlated agreement with proximity bound ${\bound}^\star(\code, \parl) := \frac{1 + \rate}{2}$ and error $\err^\star(\code, \parl, \delta) = \frac{(\parl - 1) \cdot 2^m}{\rate \cdot |\field|}$.
\end{lemma}

\begin{theorem}\label{conjecture:whir}
\lean{CorrelatedAgreement.genMutualCorrAgreement_le_johnsonBound,CorrelatedAgreement.genMutualCorrAgreement_le_capacity}
\uses{def:reed_solomon_code,lemma:gen_mutual_corr_agreement}
    The function $\gen(\parl; \alpha) := (1, \alpha, \ldots, \alpha^{\parl - 1})$ is a proximity generator with mutual correlated agreement for every smooth Reed Solomon code $\code := \rscode[\field, \evaldomain, m]$ (with rate $\rate := 2^m / |\evaldomain|$). We give two conjectures, for the parameters of the proximity bound ${\bound}^\star$ and the error $\err^\star$:
    \begin{enumerate}
        \item \textit{Up to the Johnson bound:} ${\bound}^\star(\code, \parl) := \sqrt{\rate},$ \textit{and}
        \[
        \err(\code, \parl, \delta) := \frac{(\parl - 1) \cdot 2^m}{|\field| \cdot \left( 2 \cdot \min\left\{1 - \sqrt{\rate} - \delta, \frac{\sqrt{\rate}}{20} \right\} \right)^7}.
        \]
      
        \item \textit{Up to capacity:} ${\bound}^\star(\code, \parl) := \rate$, \textit{and there exist constants $c_1, c_2, c_3 \in \mathbb{N}$ such that for every $\eta > 0$ and $0 < \delta < 1 - \rate - \eta$:}
        \[
        \err^\star(\code, \parl, \delta) := \frac{(\parl - 1)^{c_2} \cdot \delta^{c_2}}{\eta^{c_1} \cdot \rate^{c_1 + c_2} \cdot |\field|}.
        \]
      \end{enumerate}
\end{theorem}

\subsubsection{Mutual correlated agreement preserves list decoding}

\begin{lemma}\label{lemma: mutual_corrAgr_listdecoding}
\lean{CorrelatedAgreement.mutualCorrAgreement_list_decoding}
\uses{def:reed_solomon_code,def:interleaved_code,def:list_close_codewords,lemma:gen_mutual_corr_agreement}
    Let $\code \subseteq \field^{\evaldomain}$ be a linear code with minimum distance $\delta_{\code}$, and let $\gen$ be a proximity generator for $\code$ with mutual correlated agreement with proximity bound ${\bound}^\star$ and error $\err^\star$. Then, for every $f_0, \ldots, f_{\parl-1} : \evaldomain \to \field$ and $\delta \in (0, \min\{\delta_{\code}, 1 - {\bound}^\star(\code, \parl)\})$:
    \[
    \Pr_{\substack{\alpha \leftarrow \{0,1\}^{w^\star} \\ \boldsymbol{r} := \gen(\parl; \alpha)}} \left[
    \Lambda\left(\code, \sum_{j \in [0,(\parl-1)]} r_j \cdot f_j, \delta \right) \neq 
    \left\{ \sum_{j \in [0,(\parl-1)]} r_j \cdot u_j : \boldsymbol{u} \in \Lambda\left(\mathcal{C}^\ell, (f_0, \ldots, f_{\parl-1}), \delta \right) \right\}
    \right] \leq \err^\star(\code, \parl, \delta).
    \]
\end{lemma}

\subsubsection{Folding univariate functions}

\begin{definition}\label{def:extract}
\lean{Fold.extract_x}
    Let $\mathsf{extract}:\evaldomain^{2^{k+1}}\rightarrow \evaldomain^{2^k}$ be a function. There exists $x \in \evaldomain$, such that $y = x^{2^{k+1}}\in\evaldomain^{2^{k+1}}$. Then $\mathsf{extract}$ returns $z = \sqrt{y} = x^{2^k}\in\evaldomain^{2^k}$ such that $y = z^2$.
\end{definition}

\begin{definition}\label{def:foldf}
\lean{Fold.foldf}
\uses{def:extract}
    Let $f : \evaldomain^{2^k} \to \field$ be a function, and $\alpha \in \field$. We define $\mathrm{Fold_f}(f, {\alpha}) : \evaldomain^{(2^{k+1})} \to \field$ as follows:
    \[
    \forall x \in \evaldomain^{2^k}, y \in \evaldomain^{2^{k+1}}, \quad \mathrm{Fold_f}(f, \alpha)(y) := \frac{f(x) + f(-x)}{2} + \alpha \cdot \frac{f(x) - f(-x)}{2 \cdot x}.
    \]

    In order to compute $\mathrm{Fold_f}(f, \alpha)(y)$ it suffices to query $f$ at $x$ and $-x$, by retrieving $x=\mathsf{extract}(y)$.
\end{definition}

\begin{definition}\label{def:fold_k}
\lean{Fold.fold_k_core,Fold.fold_k}
\uses{def:foldf}
    For $k \leq m$ and $\vec{\alpha} = (\alpha_0, \ldots, \alpha_{k-1}) \in \field^k$ we define $\mathrm{Fold}(f, \vec{\alpha}) : \evaldomain^{2^k} \to \field$ to equal $\mathrm{Fold}(f, \vec{\alpha}) := f_k$ where $f_k$ is defined recursively as follows: $f_0 := f$, and $f_i := \mathrm{Fold_f}(f_{i-1}, \alpha_i)$. 
\end{definition}

\begin{definition}\label{def:fold_k_set}
\lean{Fold.fold_k_set}
\uses{def:fold_k}
    For a set $S \subseteq \field^{\evaldomain}$ we denote $\mathrm{Fold_{S}}(S, \vec{\alpha}) := \{\mathrm{Fold_{S}}(f, \vec{\alpha}) \mid f \in S\}$.
\end{definition}

\begin{lemma}\label{lemma:fold_fg}
\lean{Fold.fold_f_g}
\uses{def:fold_k,def:smooth_rs_code}
    Let $f : \evaldomain \to \field$ be a function, $\vec{\alpha} \in \field^k$ folding randomness and let $g := \mathrm{Fold}(f, \vec{\alpha})$. If $f \in \rscode[\field, \evaldomain, m]$ and $k \leq m$, then $g \in \rscode[\field, \evaldomain^{2^k}, m - k]$, and further the multilinear extension of $g$ is given by $\hat{g}(X_k, \ldots, X_{m-1}) := \hat{f}(\vec{\alpha}, X_k, \ldots, X_{m-1})$ where $\hat{f}$ is the multilinear extension of $f$.
\end{lemma}

\subsubsection{Block relative distance}

\begin{definition}\label{def:block}
    \lean{BlockRelDistance.block}
    Let $\evaldomain \subseteq \field$ be a smooth evaluation domain and $k \in \mathbb{N}$ be a folding parameter. For $z \in \evaldomain^{2^k}$, define $\mathrm{Block}(\evaldomain, i, k, z) := \{ x \in \evaldomain, y \in \evaldomain^{2^i} : y^{2^{k-i}} = z \}$.
\end{definition}

\begin{definition}\label{def:block_rel_distance}
\lean{BlockRelDistance.blockRelDistance}
\uses{def:block}
    Let $\code := \rscode[\field, \evaldomain, m]$ be a smooth Reed Solomon code and let $f, g : \evaldomain^{2^i} \to \field$. We define the $(i,k)$-wise block relative distance as
    \[
    \Delta_{r}(\code, i, k, f, g) = \frac{ \left| \left\{ z \in \evaldomain^{2^k} : \exists y \in \mathrm{Block}(\evaldomain, i, k, z), f(y) \neq g(y) \right\} \right| }{ |\evaldomain^{2^k}| }
    \]
\end{definition}

\begin{definition}\label{def:min_block_rel_distance}
\lean{BlockRelDistance.minBlockRelDistance}
\uses{def:block_rel_distance}
    For $S \subseteq \field^{\evaldomain}$, we let $\Delta_{r}(\code, i, k, f, S) := \min_{g \in S} \Delta_{r}(\code, i, k, f, g)$.
\end{definition}

{Note that $\Delta_{r}(\code, 0, 0, f, g) = \Delta(f, g)$ for any $\code$. We define the block list decoding of a codeword.}

\begin{definition}\label{def:list_close_codewords_block}
\lean{BlockRelDistance.listBlockRelDistance}
\uses{def:block_rel_distance}
    For a smooth Reed Solomon code $\rscode := \rscode[\field, \evaldomain, m]$, proximity parameter $\delta \in [0,1]$, and $f : \evaldomain^{2^i} \to \field$, we let
    \[
    \Lambda_{r}(\code, i, k, f, \delta) := \{ u \in \code \mid \Delta_{r}(\code, i, k, f, u) \leq \delta \},
    \]
    denote the list of codewords in $\code$ within relative block distance at most $\delta$ from $f$.
\end{definition}

\begin{lemma}\label{lemma:block_rel_distance}
\lean{BlockRelDistance.blockRelDistance_le_hammingDistance}
\uses{def:block_rel_distance,def:distance_from_code,def:list_close_codewords_block,def:list_close_codewords}
    For any $\code := \rscode[\field, \evaldomain, m]$, $k \in \mathbb{N}$, and $f, g : \evaldomain^{2^i} \to \field$, we have that $\Delta(f, g) \leq \Delta_{r}(\code, i, k, f, g)$. Consequently, $\Lambda_{r}(\code, i, k, f, \delta) \subseteq \Lambda(\code, f, \delta)$ for $\delta\in[0,1]$.
\end{lemma}

\subsubsection{Folding preserves list decoding}

\begin{theorem}\label{thm:folding_preserves_listdecoding}
\lean{Fold.folding_listdecoding_if_genMutualCorrAgreement}
\uses{lemma:folding_preserves_listdecoding_base, lemma:folding_preserves_listdecoding_bound, lemma:folding_preserves_listdecoding_base_ne_subset}
    Let $\code = \rscode[\field, \evaldomain, m]$ be a smooth Reed Solomon code and $k \leq m$. For $0 \leq i \leq k$ let $\code^{(i)} := \rscode[\field, \evaldomain^{2^i}, m - i]$. Let $\gen(\parl; \alpha) = (1, \alpha, \ldots, \alpha^{\parl - 1})$ be a proximity generator with mutual correlated agreement for the codes $\code^{(0)}, \ldots, \mathcal{C}^{(k-1)}$ with proximity bound ${\bound}^\star$ and error $\err^\star$. Then for every $f : \evaldomain \to \field$ and $\delta \in \left(0, 1 - \max_{i \in [0,(k-1)]} \{ {\bound}^\star(\code^{(i)}, 2) \} \right)$,
    \[
    \Pr_{\alpha \leftarrow \field^k} \left[
    \mathrm{Fold_S}(\Lambda_{r}(\code, 0, k, f, \delta), \alpha)
    \neq \Lambda(\code^{(k)}, \mathrm{Fold}(f, \alpha), \delta)
    \right] < \err^{(k)}(\code, \delta).
    \]
\end{theorem}

\begin{lemma}\label{lemma:folding_preserves_listdecoding_base}
\lean{Fold.folding_preserves_listdecoding_base}
\uses{def:list_close_codewords_block,def:fold_k,def:fold_k_set}
    Let $\code := \rscode[\field, \evaldomain, m]$ be a Reed Solomon code, and $k \leq m$ be a parameter. Denote $\code' := \rscode[\field, \evaldomain^{2}, m - 1]$. Then for every $f : \evaldomain \to \field$ and $\delta \in (0, 1 - {\bound}^\star(\code', 2))$,
    \[
    \Pr_{\alpha \leftarrow \field} \left[
    \mathrm{Fold_S}(\Lambda_{r}(\code, 0, k, f, \delta), \alpha)
    \neq \Lambda_{r}(\code', 1, k, \mathrm{Fold}(f, \alpha), \delta)
    \right] < \err^\star(\code', 2, \delta).
    \]
\end{lemma}
    
\begin{lemma}\label{lemma:folding_preserves_listdecoding_bound}
\lean{Fold.folding_preserves_listdecoding_bound}
\uses{def:list_close_codewords_block,def:fold_k}
    For every $\alpha \in \field$, $\mathrm{Fold_S}(\Lambda_{r}(\code, 0, k, f, \delta), \alpha) \subseteq \Lambda_{r}(\code', 1, k, \mathrm{Fold}(f, \alpha), \delta)$.
\end{lemma}

\begin{lemma}\label{lemma:folding_preserves_listdecoding_base_ne_subset}
\lean{Fold.folding_preserves_listdecoding_base_ne_subset}
\uses{def:list_close_codewords_block,def:fold_k,def:fold_k_set}
    \[
    \Pr_{\alpha \leftarrow \field} \left[
    \Lambda_{r}(\code', 1, k, \mathrm{Fold}(f, \alpha), \delta)
    \not\subseteq \mathrm{Fold_S}(\Lambda_{r}(\code, 0, k, f, \delta), \alpha)
    \right] < \err^\star(\code', 2, \delta).
    \]
\end{lemma}

\begin{lemma}\label{lemma:crs_equiv_rs_randpompt_agreement}
\lean{OutOfDomSmpl.crs_equiv_rs_randpompt_agreement}
\uses{def:smooth_rs_code,def:constrained_code,def:list_close_codewords}
    Let $f : \evaldomain \rightarrow \field$ be a function, $m \in \mathbb{N}$ be a number of variables, $s \in \mathbb{N}$ be a repetition parameter, and let $\delta \in [0,1]$ be a distance parameter. For every $\vec{r_0}, \dots, \vec{r_{s-1}} \in \field^m$, the following are equivalent statements.
\begin{itemize}
    \item There exist distinct $u, u' \in \Lambda(\rscode[\field, \evaldomain, m], f, \delta)$ such that, for every $i \in [0,s-1]$, $\hat{u}(\vec{r_i}) = \hat{u}'(\vec{r_i})$.
    \item There exists $\sigma_0, \dots, \sigma_{s-1} \in \field$ such that $$\left| \Lambda(\crscode[\field, \evaldomain, m, ((Z \cdot \mathrm{eq}(\vec{r_0}, \cdot), \sigma_0), \dots, (Z \cdot \mathrm{eq}(\vec{r_{s-1}}, \cdot), \sigma_{s-1}))], f, \delta) \right| > 1.$$
\end{itemize}
\end{lemma}

\begin{lemma}\label{lemma:out_of_domain_sampling_crs_eq_rs}
\lean{OutOfDomSmpl.out_of_domain_sampling_crs_eq_rs}
\uses{def:smooth_rs_code,def:constrained_code,def:list_close_codewords,def:list_decodable}
    Let $f : \evaldomain \rightarrow \field$ be a function, $m \in \mathbb{N}$ be a number of variables, $s \in \mathbb{N}$ be a repetition parameter, and $\delta \in [0,1]$ be a distance parameter. If $\rscode[\field, \evaldomain, m]$ is $(\delta, \ell)$-list decodable then
    \[
    \Pr_{r_0, \dots, r_{s-1} \leftarrow \field} \left[
    \begin{array}{c}
    \exists \sigma_0, \dots, \sigma_{s-1} \in \field \text{ s.t.} \\
    \left| \Lambda(\crscode[\field, \evaldomain, m, ((Z \cdot \mathrm{eq}(\mathrm{pow}(r_{i}, m), \cdot), \sigma_{i}))_{i \in [s]}], f, \delta) \right| > 1
    \end{array}
    \right]
    \]

    \[
    = \Pr_{r_0, \dots, r_{s-1} \leftarrow \field} \left[
    \begin{array}{c}
    \exists \text{ distinct } u, u' \in \Lambda(\rscode[\field, \evaldomain, m], f, \delta) \\
    \text{ s.t. } \forall i \in [s],\ \hat{u}(\mathrm{pow}(r_i, m)) = \hat{u}'(\mathrm{pow}(r_i, m))
    \end{array}
    \right]
    \]

    \[
    \leq \frac{\ell^2}{2} \cdot \left( \frac{2^m}{|\field|} \right)^s.
    \]
\end{lemma}

\begin{theorem}\label{thm: whir_rbr_soundness}
\lean{WhirIOP.whir_rbr_soundness}
\uses{thm:folding_preserves_listdecoding,lemma:out_of_domain_sampling_crs_eq_rs,lemma:gen_mutual_corr_agreement,def:constrained_code}
    Consider $(\field, M, (k_i, m_i, \evaldomain_i, t_i)_{0 \leq i \leq M}, \widehat{w}_0, \sigma_0, m, d^\star, d)$ with the following ingrediants and conditions,
    \begin{itemize}
    \item a constrained Reed Solomon code $\crscode[\field, \evaldomain_0, m_0, \widehat{w}_0, \sigma_0]$;
    \item an iteration count $M \in \mathbb{N}$;
    \item folding parameters $k_0, \ldots, k_{M}$ such that $\sum_{i=0}^{M} k_i \leq m$;
    \item evaluation domains $\evaldomain_0, \ldots, \evaldomain_{M} \subseteq \field$ where $\evaldomain_i$ is a smooth coset of $\field^*$ with order $|\evaldomain_i| \geq 2^{m_i}$;
    \item repetition parameters $t_0, \ldots, t_M$ with $t_i \leq |\evaldomain_i|$;
    \item define $m_0 := m$ and $m_i := m - \sum_{j < i} k_j$;
    \item define $d^\star := 1 + \deg_{\mathbb{Z}}(\widehat{w}_0) + \max_{i \in [m_0]} \deg_{X_i}(\widehat{w}_0)$ and $d := \max\{d^\star, 3\}$.
    \end{itemize}
    For every $f \notin \crscode[\field, \evaldomain_0, m_0, \widehat{w}_0, \sigma_0]$ and every $\delta_0, \dots, \delta_{M}$ and $(\parl_{i,s})_{0 \leq i \leq M}^{0 \leq s \leq k_i}$ where
    \begin{itemize}
        \item $\delta_0 \in (0, \Delta(f, \crscode[\field, \evaldomain_0, m_0, \widehat{w}_0, \sigma_0]))$;
        \item the function $\gen(\parl; \alpha) = (1, \alpha, \dots, \alpha^{\parl-1})$ is a proximity generator with mutual correlated agreement for the codes $(\mathcal{C}_{\mathrm{RS}}^{(i,s)})_{0 \leq i \leq M}^{0 \leq s \leq k_i}$ where $\mathcal{C}_{\mathrm{RS}}^{(i,s)} := \rscode[\field, \evaldomain_i^{(2^s)}, m_i - s]$ with bound ${\bound}^\star$ and error $\err^\star$;
        \item for every $0 \leq i \le M$, $\delta_i \in (0, 1 - {\bound}^\star(\mathcal{C}_{\mathrm{RS}}^{(i,s)}, 2))$;
        \item for every $0 \leq i \le M$, $\mathcal{C}_{\mathrm{RS}}^{(i,s)}$ is $(\ell_{i,s}, \delta_i)$-list decodable.
    \end{itemize}
   Then there exists an IOPP for $\crscode[\field, \evaldomain_0, m_0, \widehat{w}_0, \sigma_0]$ with above parameters, with round-by-round soundness error

    \[
((\varepsilon_{0,s}^{\mathrm{fold}})_{s \leq k_0},\ (\varepsilon_i^{\mathrm{out}},\ \varepsilon_i^{\mathrm{shift}})_{i \leq M},\ (\varepsilon_{i,s}^{\mathrm{fold}})_{i \in [M], s \leq k_i},\ \varepsilon^{\mathrm{fin}}),
\]

{where:}

\begin{itemize}
    \item $\varepsilon_{0,s}^{\mathrm{fold}} \leq \dfrac{d^* \cdot \ell_{0,s-1}}{|\mathbb{F}|} + \mathrm{err}^*(\mathcal{C}_{\mathrm{RS}}^{(0,s)}, 2, \delta_0)$;
    \item $\varepsilon_i^{\mathrm{out}} \leq \dfrac{2^{m_i} \cdot \ell_{i,0}^2}{2 \cdot |\mathbb{F}|}$;
    \item $\varepsilon_i^{\mathrm{shift}} \leq (1 - \delta_{i-1})^{t_i - 1} + \dfrac{\ell_{i,0} \cdot (t_i - 1 + 1)}{|\mathbb{F}|}$;
    \item $\varepsilon_{i,s}^{\mathrm{fold}} \leq \dfrac{d \cdot \ell_{i,s-1}}{|\mathbb{F}|} + \mathrm{err}^*(\mathcal{C}_{\mathrm{RS}}^{(i,s)}, 2, \delta_i)$;
    \item $\varepsilon^{\mathrm{fin}} \leq (1 - \delta_{M-1})^{t_M - 1}$.
\end{itemize}
\end{theorem}




\section{The Spartan Protocol}

\section{The Ligero Polynomial Commitment Scheme}

\chapter{Commitment Schemes}\label{chap:commitment_schemes}

\section{Definitions}

\section{Merkle Trees}\label{sec:merkle_trees}

% TODO: add Merkle tree definitions and theorems about (multi-)extractability, and privacy

% TODO: add KZG

% We probably want to describe any commitment scheme with multi-round opening in the proof systems chapter (including Ligero, FRI, Hyrax, etc.)

\chapter{Supporting Theories}\label{chap:supporting_theories}

\section{Definitions}\label{sec:oracle_reductions_defs}

In this section, we give the basic definitions of a public-coin interactive oracle reduction
(henceforth called an oracle reduction or IOR). We will define its building blocks, and various
security properties.

\subsection{Format}\label{sec:oracle_reductions_defs_format}

An \textbf{(interactive) oracle reduction (IOR)} is an interactive protocol between two parties, a
\emph{prover} $\mathcal{P}$ and a \emph{verifier} $\mathcal{V}$. In ArkLib, IORs are defined in the
following setting:
\begin{enumerate}
    \item We work in an ambient dependent type theory (in our case, Lean).

    \item The protocol flow is fixed and defined by a given \emph{type signature}, which
    describes in each round which party sends a message to the other, and the type of that message.

    \item The prover and verifier has access to some inputs (called the \emph{(oracle) context}) at
    the beginning of the protocol. These inputs are classified as follows:
    \begin{itemize}
        \item \emph{Public inputs} (or \emph{statement}) $\mathbbm{x}$: available to both parties;
        \item \emph{Private inputs} (or \emph{witness}) $\mathbbm{w}$: available only to the prover;
        \item \emph{Oracle inputs} (or \emph{oracle statement}) $\mathbbm{ox}$: the underlying data
        is available to the prover, but it's only exposed as an oracle to the verifier. See~\Cref{def:oracle_interface} for more information.
        \item \emph{Shared oracle} $\mathcal{O}$: the oracle is available to both parties via an
        interface; in most cases, it is either empty, a probabilistic sampling oracle, a random
        oracle, or a group oracle (for the Algebraic Group Model). See~\Cref{sec:vcvio} for more
        information on oracle computations.
    \end{itemize}

    \item The messages sent from the prover may either: 1) be seen directly by the verifier, or 2)
    only available to a verifier through an \emph{oracle interface} (which specifies the type for
    the query and response, and the oracle's behavior given the underlying message).

    Currently, in the oracle reduction setting, we \emph{only} allow messages sent to be available
    through oracle interfaces. In the (non-oracle) reduction setting, all messages are available
    directly. Future extensions may allow for mixed visibility for prover's messages.

    \item $\mathcal{V}$ is assumed to be \emph{public-coin}, meaning that its challenges are chosen
    uniformly at random from the finite type corresponding to that round, and it uses no randomness
    otherwise, except from those coming from the shared oracle.

    \item At the end of the protocol, the prover and verifier outputs a new (oracle) context, which consists of:
    \begin{itemize}
        \item The verifier takes in the input statement and the challenges, performs an \emph{oracle} computation on the input oracle statements and the oracle messages, and outputs a new output statement.

        The verifier also outputs the new oracle statement in an implicit manner, by specifying a
        subset of the input oracle statements \& the oracle messages. Future extensions may allow for more flexibility in specifying output oracle statements (i.e. not just a subset, but a linear combination, or any other function).
        \item The prover takes in some final private state (maintained during protocol execution), and outputs a new output statement, new output oracle statement, and new output witness.
    \end{itemize}
\end{enumerate}

\begin{remark}[Literature Comparison]
In the literature, our definition corresponds to the notion of \emph{functional} IORs. Historically,
(vector) IOPs were the first notion to be introduced by~\cite{IOPs}; these are IORs where the output
statement is true/false, all oracle statements and messages are vectors over some alphabet $\Sigma$,
and the oracle interfaces are for querying specific positions in the vector. More recent works have
considered other oracle interfaces, e.g., polynomial oracles~\cite{Marlin, DARK}, generalized proofs
to reductions~\cite{ARoK, WARP, Arc, fics-facs}, and considered general oracle
interfaces~\cite{WHIR}. Most of the IOP theory has been distilled in the
textbook~\cite{ChiesaYogev2024}.

We have not seen any work that considers our most general setting, of IORs with arbitrary oracle interfaces.
\end{remark}

We now go into more details on these objects, and how they are represented in Lean. Our description will aim to be as close as possible to the Lean code, and hence may differ somewhat from ``mainstream'' mathematical \& cryptographic notation.

\begin{definition}[Oracle Interface]
    \label{def:oracle_interface}
    An oracle interface for an underlying data type $\mathsf{D}$ consists of the following:
    \begin{itemize}
        \item A type $\mathsf{Q}$ for queries to the oracle,
        \item A type $\mathsf{R}$ for responses from the oracle,
        \item A function $\mathsf{oracle} : \mathsf{D} \to \mathsf{Q} \to \mathsf{R}$ that specifies
        the oracle's behavior given the underlying data and a query.
    \end{itemize}
    \lean{OracleInterface}
\end{definition}

See \texttt{OracleInterface.lean} for common instances of $\mathsf{OracleInterface}$.


\begin{definition}[Context]
    \label{def:context}
    In an (oracle) reduction, its \emph{(oracle) context} consists of a statement type, a witness
    type, and (in the oracle case) an indexed list of oracle statement types.

    Currently, we do not abstract out / bundle the context as a separate structure, but rather
    specifies the types explicitly. This may change in the future.
\end{definition}

\begin{definition}[Protocol Specification]
    \label{def:protocol_spec}
    A protocol specification for an $n$-message (oracle) reduction, is an element of the following type:
    \begin{align*}
        \ProtocolSpec\ n &:= \Fin\ n \to \Direction \times \Type.
    \end{align*}
    In the above, $\Direction := \{ \PtoVdir, \VtoPdir \}$ is the type of possible directions of messages, and $\Fin\ n := \{ i : \bbN \quotient i < n \}$ is the type of all natural numbers less than $n$.

    In other words, for each step $i$ of interaction, the protocol specification describes the \emph{direction} of the message sent in that step, i.e., whether it is from the prover or from the verifier. It also describes the \emph{type} of that message.

    In the oracle setting, we also expect an oracle interface for each message from the prover to the verifier.
    \lean{ProtocolSpec}
\end{definition}

We define some supporting definitions for a protocol specification.

\begin{definition}[Protocol Specification Components]
    \label{def:protocol_spec_components}
    Given a protocol spec $\pSpec : \ProtocolSpec\ n$, we define:
    \begin{itemize}
        \item $\pSpec.\Dir\ i := (\pSpec\ i).\mathsf{fst}$ extracts the direction of the $i$-th message.
        \item $\pSpec.\Type\ i := (\pSpec\ i).\mathsf{snd}$ extracts the type of the $i$-th message.
        \item $\pSpec.\MessageIdx := \{i : \Fin\ n \quotient \pSpec.\Dir\ i = \PtoVdir\}$ is the subtype of indices corresponding to prover messages.
        \item $\pSpec.\ChallengeIdx := \{i : \Fin\ n \quotient \pSpec.\Dir\ i = \VtoPdir\}$ is the subtype of indices corresponding to verifier challenges.
        \item $\pSpec.\mathsf{Message}\ i := (i : \pSpec.\MessageIdx) \to \pSpec.\Type\ i.\mathsf{val}$ is an indexed family of message types in the protocol.
        \item $\pSpec.\mathsf{Challenge}\ i := (i : \pSpec.\ChallengeIdx) \to \pSpec.\Type\ i.\mathsf{val}$ is an indexed family of challenge types in the protocol.
    \end{itemize}
    \lean{ProtocolSpec.dir, ProtocolSpec.Type, ProtocolSpec.MessageIdx, ProtocolSpec.ChallengeIdx, ProtocolSpec.Message, ProtocolSpec.Challenge}
    \uses{def:protocol_spec}
\end{definition}

\begin{definition}[Protocol Transcript]
    \label{def:transcript}
        Given protocol specification $\pSpec : \ProtocolSpec\ n$, we define:
    \begin{itemize}
        \item A \emph{transcript} up to round $k : \Fin\ (n + 1)$ is an element of type
                \[ \Transcript\ k\ \pSpec := (i : \Fin\ k) \to \pSpec.\Type\ (\uparrow i : \Fin\ n) \]
        where $\uparrow i : \Fin\ n$ denotes casting $i : \Fin\ k$ to $\Fin\ n$ (valid since $k \leq n + 1$).

        \item A \emph{full transcript} is $\FullTranscript\ \pSpec := (i : \Fin\ n) \to \pSpec.\Type\ i$.

        \item The type of all \emph{messages} from prover to verifier is
        \[ \pSpec.\Messages := \prod_{i : \pSpec.\MessageIdx} \pSpec.\Message\ i \]

        \item The type of all \emph{challenges} from verifier to prover is
        \[ \pSpec.\Challenges := \prod_{i : \pSpec.\ChallengeIdx} \pSpec.\Challenge\ i \]
    \end{itemize}
    \lean{ProtocolSpec.Transcript, ProtocolSpec.Message, ProtocolSpec.Challenge}
    \uses{def:protocol_spec, def:protocol_spec_components}
\end{definition}

% In the interactive protocols we consider, both parties $P$ and $V$ may have access to a shared
% oracle $O$. An interactive protocol becomes an \emph{interactive (oracle) reduction} if its
% execution reduces an input relation $R_{\mathsf{in}}$ to an output relation $R_{\mathsf{out}}$. Here
% a relation is just a function $\mathsf{IsValid}: \mathsf{Statement} \times \mathsf{Witness} \to
% \mathsf{Bool}$, for some types \verb|Statement| and \verb|Witness|. We do not concern ourselves with
% the running time of $\mathsf{IsValid}$ in this project (though future extensions may prove that
% relations can be decided in polynomial time, for a suitable model of computation).

\begin{remark}[Design Decision]
    We do not enforce a particular interaction flow in the definition of an interactive (oracle) reduction. This is done so that we can capture all protocols in the most generality. Also, we want to allow the prover to send multiple messages in a row, since each message may have a different oracle representation (for instance, in the Plonk protocol, the prover's first message is a 3-tuple of polynomial commitments.)
\end{remark}

\begin{definition}[Type Signature of a Prover]
    \label{def:prover}
    A prover $\mathcal{P}$ in a reduction consists of the following components:

    \begin{itemize}
        \item \textbf{Prover State}: A family of types $\mathsf{PrvState} : \Fin(n+1) \to \Type$ representing the prover's internal state at each round of the protocol.

        \item \textbf{Input Processing}: A function
        \[ \mathsf{input} : \StmtIn \to \WitIn \to \mathsf{PrvState}(0) \]
        that initializes the prover's state from the input statement and witness.

        \item \textbf{Message Sending}: For each message index $i : \pSpec.\MessageIdx$, a function
        \[ \mathsf{sendMessage}_i : \mathsf{PrvState}(i.\mathsf{val}.\mathsf{castSucc}) \to \OracleComp(\oSpec, \pSpec.\Message(i) \times \mathsf{PrvState}(i.\mathsf{val}.\mathsf{succ})) \]
        that generates the message and updates the prover's state.

        \item \textbf{Challenge Processing}: For each challenge index $i : \pSpec.\ChallengeIdx$, a function
        \[ \mathsf{receiveChallenge}_i : \mathsf{PrvState}(i.\mathsf{val}.\mathsf{castSucc}) \to \pSpec.\Challenge(i) \to \mathsf{PrvState}(i.\mathsf{val}.\mathsf{succ}) \]
        that updates the prover's state upon receiving a challenge.

        \item \textbf{Output Generation}: A function
        \[ \mathsf{output} : \mathsf{PrvState}(\Fin.\mathsf{last}(n)) \to \StmtOut \times \WitOut \]
        that produces the final output statement and witness from the prover's final state.
    \end{itemize}
    \lean{Prover, ProverState, ProverInput, ProverRound, ProverOutput}
\end{definition}

\begin{definition}[Type Signature of an Oracle Prover]
    \label{def:oracle_prover}
    An oracle prover is a prover whose input statement includes the underlying data for oracle statements, and whose output includes oracle statements. Formally, it is a prover with input statement type $\StmtIn \times (\forall i : \iota_{\mathsf{si}}, \OStmtIn(i))$ and output statement type $\StmtOut \times (\forall i : \iota_{\mathsf{so}}, \OStmtOut(i))$, where:
    \begin{itemize}
        \item $\OStmtIn : \iota_{\mathsf{si}} \to \Type$ are the input oracle statement types
        \item $\OStmtOut : \iota_{\mathsf{so}} \to \Type$ are the output oracle statement types
    \end{itemize}
    \lean{OracleProver}
\end{definition}

% Our modeling of oracle reductions only consider \emph{public-coin} verifiers; that is, verifiers who
% only outputs uniformly random challenges drawn from the (finite) types, and uses no other
% randomness. Because of this fixed functionality, we can bake the verifier's behavior in the
% interaction phase directly into the protocol execution semantics.

After the interaction phase, the verifier may then run some verification procedure to check the
validity of the prover's responses. In this procedure, the verifier gets access to the public part
of the context, and oracle access to either the shared oracle, or the oracle inputs.
% This procedure differs depending on whether the verifier has
% full access, or only oracle access, to the prover's messages. Note that there is no difference on
% the prover side whether the protocol is an \emph{interactive oracle reduction (IOR)} or simply an
% \emph{interactive reduction (IR)}.

\begin{definition}[Type Signature of a Verifier]
    \label{def:verifier}
    A verifier $\mathcal{V}$ in a reduction is specified by a single function:
    \[ \mathsf{verify} : \StmtIn \to \FullTranscript(\pSpec) \to \OracleComp(\oSpec, \StmtOut) \]

    This function takes the input statement and the complete transcript of the protocol interaction, and performs an oracle computation (potentially querying the shared oracle $\oSpec$) to produce an output statement.

    The verifier is assumed to be \emph{public-coin}, meaning it only sends uniformly random challenges and uses no other randomness beyond what is provided by the shared oracle.
    \lean{Verifier}
\end{definition}

\begin{definition}[Type Signature of an Oracle Verifier]
    \label{def:oracle_verifier}
    An oracle verifier $\mathcal{V}$ consists of the following components:

    \begin{itemize}
        \item \textbf{Verification Logic}: A function
        \[ \mathsf{verify} : \StmtIn \to \pSpec.\Challenges \to \OracleComp(\oSpec \mathrel{++_\mathsf{o}} ([\OStmtIn]_\mathsf{o} \mathrel{++_\mathsf{o}} [\pSpec.\Message]_\mathsf{o}), \StmtOut) \]
        that takes the input statement and verifier challenges, and performs oracle queries to the shared oracle, input oracle statements, and prover messages to produce an output statement.

        \item \textbf{Output Oracle Embedding}: An injective function
        \[ \mathsf{embed} : \iota_{\mathsf{so}} \hookrightarrow \iota_{\mathsf{si}} \oplus \pSpec.\MessageIdx \]
        that specifies how each output oracle statement is derived from either an input oracle statement or a prover message.

        \item \textbf{Type Compatibility}: A proof term
        \[ \mathsf{hEq} : \forall i : \iota_{\mathsf{so}}, \OStmtOut(i) = \begin{cases}
            \OStmtIn(j) & \text{if } \mathsf{embed}(i) = \mathsf{inl}(j) \\
            \pSpec.\Message(k) & \text{if } \mathsf{embed}(i) = \mathsf{inr}(k)
        \end{cases} \]
        ensuring that output oracle statement types match their sources.
    \end{itemize}

    This design ensures that output oracle statements are always a subset of the available input oracle statements and prover messages.
    \lean{OracleVerifier}
\end{definition}

\begin{definition}[Oracle Verifier to Verifier Conversion]
    \label{def:oracle_verifier_to_verifier}
    An oracle verifier can be converted to a standard verifier through a natural simulation process. The key insight is that while an oracle verifier only has oracle access to certain data (input oracle statements and prover messages), a standard verifier can be given the actual underlying data directly.

    The conversion works as follows: when the oracle verifier needs to make an oracle query to some data, the converted verifier can respond to this query immediately using the actual underlying data it possesses. This is accomplished through the \texttt{OracleInterface} type class, which specifies for each data type how to respond to queries given the underlying data.

    Specifically, given an oracle verifier $\mathcal{V}_{\text{oracle}}$:
    \begin{itemize}
        \item The converted verifier $\mathcal{V}_{\text{oracle}}.\mathsf{toVerifier}$ takes as input both the statement \emph{and} the actual underlying data for all oracle statements
        \item When $\mathcal{V}_{\text{oracle}}$ attempts to query an oracle statement or prover message, the converted verifier uses the corresponding \texttt{OracleInterface} instance to compute the response from the actual data
        \item The output oracle statements are constructed according to the embedding specification, selecting the appropriate subset of input oracle statements and prover messages
    \end{itemize}
    \lean{OracleVerifier.toVerifier}
    \uses{def:oracle_verifier}
\end{definition}

An oracle reduction then consists of a type signature for the interaction, and a pair of prover and
verifier for that type signature.

\begin{definition}[Interactive Reduction]
    \label{def:reduction}
    An interactive reduction for protocol specification $\pSpec : \ProtocolSpec(n)$ and oracle specification $\oSpec$ consists of:
    \begin{itemize}
        \item A \textbf{prover} $\mathcal{P} : \Prover(\pSpec, \oSpec, \StmtIn, \WitIn, \StmtOut, \WitOut)$
        \item A \textbf{verifier} $\mathcal{V} : \Verifier(\pSpec, \oSpec, \StmtIn, \StmtOut)$
    \end{itemize}

    The reduction establishes a relationship between input relations on $(\StmtIn, \WitIn)$ and output relations on $(\StmtOut, \WitOut)$ through the interactive protocol defined by $\pSpec$.
    \lean{Reduction}
    \uses{def:prover, def:verifier}
\end{definition}

\begin{definition}[Interactive Oracle Reduction]
    \label{def:oracle_reduction}
    An interactive oracle reduction for protocol specification $\pSpec : \ProtocolSpec(n)$ with oracle interfaces for all prover messages, and oracle specification $\oSpec$, consists of:
    \begin{itemize}
        \item An \textbf{oracle prover} $\mathcal{P} : \OracleProver(\pSpec, \oSpec, \StmtIn, \WitIn, \StmtOut, \WitOut, \OStmtIn, \OStmtOut)$
        \item An \textbf{oracle verifier} $\mathcal{V} : \OracleVerifier(\pSpec, \oSpec, \StmtIn, \StmtOut, \OStmtIn, \OStmtOut)$
    \end{itemize}

    where:
    \begin{itemize}
        \item $\OStmtIn : \iota_{\mathsf{si}} \to \Type$ are the input oracle statement types with oracle interfaces
        \item $\OStmtOut : \iota_{\mathsf{so}} \to \Type$ are the output oracle statement types
    \end{itemize}

    The oracle reduction allows the verifier to access prover messages and oracle statements only through specified oracle interfaces, enabling more flexible and composable protocol designs.
    \lean{OracleReduction}
    \uses{def:oracle_prover, def:oracle_verifier}
\end{definition}

\subsection{Execution Semantics}\label{sec:execution_semantics}

We now define what it means to execute an oracle reduction. This is essentially achieved by first
executing the prover, interspersed with oracle queries to get the verifier's challenges (these will
be given uniform random probability semantics later on), and then executing the verifier's checks.
Any message exchanged in the protocol will be added to the context. We may also log information
about the execution, such as the log of oracle queries for the shared oracles, for analysis purposes
(i.e. feeding information into the extractor).

\begin{definition}[Prover Execution to Round]
    \label{def:prover_run_to_round}
    The execution of a prover up to round $i : \Fin(n+1)$ is defined inductively:

    \[ \mathsf{Prover}.\mathsf{runToRound}(i, \mathsf{stmt}, \mathsf{wit}) := \]
    \[ \mathsf{Fin}.\mathsf{induction}( \]
    \[ \quad \mathsf{pure}(\langle \mathsf{default}, \mathsf{prover}.\mathsf{input}(\mathsf{stmt}, \mathsf{wit}) \rangle), \]
    \[ \quad \mathsf{prover}.\mathsf{processRound}, \]
    \[ \quad i \]
    \[ ) \]

    where $\mathsf{processRound}$ handles individual rounds by either:
    \begin{itemize}
        \item \textbf{Verifier Challenge} ($\pSpec.\mathsf{dir}(j) = \mathsf{V\_to\_P}$): Query for a challenge and update prover state
        \item \textbf{Prover Message} ($\pSpec.\mathsf{dir}(j) = \mathsf{P\_to\_V}$): Generate message via $\mathsf{sendMessage}$ and update state
    \end{itemize}

    Returns the transcript up to round $i$ and the prover's state after round $i$.
    \lean{Prover.runToRound, Prover.processRound}
    \uses{def:prover, def:protocol_spec, def:transcript}
\end{definition}

\begin{definition}[Complete Prover Execution]
    \label{def:prover_run}
    The complete execution of a prover is defined as:

    \[ \mathsf{Prover}.\mathsf{run}(\mathsf{stmt}, \mathsf{wit}) := \mathsf{do} \; \{ \]
    \[ \quad \langle \mathsf{transcript}, \mathsf{state} \rangle \leftarrow \mathsf{prover}.\mathsf{runToRound}(\Fin.\mathsf{last}(n), \mathsf{stmt}, \mathsf{wit}) \]
    \[ \quad \langle \mathsf{stmtOut}, \mathsf{witOut} \rangle := \mathsf{prover}.\mathsf{output}(\mathsf{state}) \]
    \[ \quad \mathsf{return} \; \langle \mathsf{stmtOut}, \mathsf{witOut}, \mathsf{transcript} \rangle \]
    \[ \} \]

    Returns the output statement, output witness, and complete transcript.
    \lean{Prover.run}
    \uses{def:prover, def:prover_run_to_round}
\end{definition}

\begin{definition}[Verifier Execution]
    \label{def:verifier_run}
    The execution of a verifier is simply the application of its verification function:

    \[ \mathsf{Verifier}.\mathsf{run}(\mathsf{stmt}, \mathsf{transcript}) := \mathsf{verifier}.\mathsf{verify}(\mathsf{stmt}, \mathsf{transcript}) \]

    This takes the input statement and full transcript, and returns the output statement via an oracle computation.
    \lean{Verifier.run}
    \uses{def:verifier}
\end{definition}

\begin{definition}[Oracle Verifier Execution]
    \label{def:oracle_verifier_run}
    The execution of an oracle verifier is defined as:

    \[ \mathsf{OracleVerifier}.\mathsf{run}(\mathsf{stmt}, \mathsf{oStmtIn}, \mathsf{transcript}) := \mathsf{do} \; \{ \]
    \[ \quad \mathsf{f} := \mathsf{simOracle2}(\oSpec, \mathsf{oStmtIn}, \mathsf{transcript}.\mathsf{messages}) \]
    \[ \quad \mathsf{stmtOut} \leftarrow \mathsf{simulateQ}(\mathsf{f}, \mathsf{verifier}.\mathsf{verify}(\mathsf{stmt}, \mathsf{transcript}.\mathsf{challenges})) \]
    \[ \quad \mathsf{return} \; \mathsf{stmtOut} \]
    \[ \} \]

    This simulates the oracle access to input oracle statements and prover messages, then executes the verification logic.
    \lean{OracleVerifier.run}
    \uses{def:oracle_verifier, def:oracle_interface}
\end{definition}

\begin{definition}[Interactive Reduction Execution]
    \label{def:reduction_run}
    The execution of an interactive reduction consists of running the prover followed by the verifier:

    \[ \mathsf{Reduction}.\mathsf{run}(\mathsf{stmt}, \mathsf{wit}) := \mathsf{do} \; \{ \]
    \[ \quad \langle \mathsf{prvStmtOut}, \mathsf{witOut}, \mathsf{transcript} \rangle \leftarrow \mathsf{reduction}.\mathsf{prover}.\mathsf{run}(\mathsf{stmt}, \mathsf{wit}) \]
    \[ \quad \mathsf{stmtOut} \leftarrow \mathsf{reduction}.\mathsf{verifier}.\mathsf{run}(\mathsf{stmt}, \mathsf{transcript}) \]
    \[ \quad \mathsf{return} \; ((\mathsf{prvStmtOut}, \mathsf{witOut}), \mathsf{stmtOut}, \mathsf{transcript}) \]
    \[ \} \]

    Returns both the prover's output (statement and witness) and the verifier's output statement, along with the complete transcript.
    \lean{Reduction.run}
    \uses{def:reduction, def:prover_run, def:verifier_run}
\end{definition}

\begin{definition}[Oracle Reduction Execution]
    \label{def:oracle_reduction_run}
    The execution of an interactive oracle reduction is similar to a standard reduction but includes logging of oracle queries:

    \[ \mathsf{OracleReduction}.\mathsf{run}(\mathsf{stmt}, \mathsf{wit}, \mathsf{oStmt}) := \mathsf{do} \; \{ \]
    \[ \quad \langle \langle \mathsf{prvStmtOut}, \mathsf{witOut}, \mathsf{transcript} \rangle, \mathsf{proveQueryLog} \rangle \leftarrow \]
    \[ \qquad (\mathsf{simulateQ}(\mathsf{loggingOracle}, \mathsf{reduction}.\mathsf{prover}.\mathsf{run}(\langle \mathsf{stmt}, \mathsf{oStmt} \rangle, \mathsf{wit}))).\mathsf{run} \]
    \[ \quad \langle \mathsf{stmtOut}, \mathsf{verifyQueryLog} \rangle \leftarrow \]
    \[ \qquad (\mathsf{simulateQ}(\mathsf{loggingOracle}, \mathsf{reduction}.\mathsf{verifier}.\mathsf{run}(\mathsf{stmt}, \mathsf{oStmt}, \mathsf{transcript}))).\mathsf{run} \]
    \[ \quad \mathsf{return} \; ((\mathsf{prvStmtOut}, \mathsf{witOut}), \mathsf{stmtOut}, \mathsf{transcript}, \mathsf{proveQueryLog}, \mathsf{verifyQueryLog}) \]
    \[ \} \]

    Returns the same outputs as a standard reduction, plus logs of all oracle queries made by both the prover and verifier.
    \lean{OracleReduction.run}
    \uses{def:oracle_reduction, def:prover_run, def:oracle_verifier_run}
\end{definition}


\subsection{Security Properties}\label{sec:security}

We can now define properties of interactive reductions. The two main properties we consider in this
project are completeness and various notions of soundness. We will cover zero-knowledge at a later
stage.

First, for completeness, this is essentially probabilistic Hoare-style conditions on the execution
of the oracle reduction (with the honest prover and verifier). In other words, given a predicate on
the initial context, and a predicate on the final context, we require that if the initial predicate
holds, then the final predicate holds with high probability (except for some \emph{completeness}
error).

\begin{definition}[Completeness]
    \label{def:completeness}
    A reduction satisfies \textbf{completeness} with error $\epsilon \geq 0$ and with respect to
    input relation $R_{\text{in}}$ and output relation $R_{\text{out}}$, if for all valid statement-witness pair
    $(x_{\text{in}}, w_{\text{in}})$ for $R_{\text{in}}$, the execution between the honest prover and the honest verifier
    will result in a tuple $((x_{\text{out}}^P, w_{\text{out}}), x_{\text{out}}^V)$ such that:
    \begin{itemize}
        \item $R_{\text{out}}(x_{\text{out}}^V, w_{\text{out}}) = \text{True}$ (the output statement-witness pair is valid), and
        \item $x_{\text{out}}^P = x_{\text{out}}^V$ (the output statements are the same from both prover and verifier)
    \end{itemize}
    except with probability $\epsilon$.
    \lean{Reduction.completeness}
    \uses{def:reduction, def:reduction_run}
\end{definition}

\begin{definition}[Perfect Completeness]
    \label{def:perfect_completeness}
    A reduction satisfies \textbf{perfect completeness} if it satisfies completeness with error $0$.
    This means that the probability of the reduction outputting a valid statement-witness pair is
    \emph{exactly} 1 (instead of at least $1 - 0$).
    \lean{Reduction.perfectCompleteness}
    \uses{def:completeness}
\end{definition}

Almost all oracle reductions we consider actually satisfy \emph{perfect completeness}, which
simplifies the proof obligation. In particular, this means we only need to show that no matter what challenges are chosen, the verifier will always accept given messages from the honest prover.

\subsubsection{Extractors}

For knowledge soundness, we need to consider different types of extractors that can recover witnesses from malicious provers.

\begin{definition}[Straightline Extractor]
    \label{def:straightline_extractor}
    A \textbf{straightline, deterministic, non-oracle-querying extractor} takes in:
    \begin{itemize}
        \item the output witness $w_{\text{out}}$,
        \item the initial statement $x_{\text{in}}$,
        \item the IOR transcript $\tau$,
        \item the query logs from the prover and verifier
    \end{itemize}
    and returns a corresponding initial witness $w_{\text{in}}$.

    Note that the extractor does not need to take in the output statement, since it can be derived
    via re-running the verifier on the initial statement, the transcript, and the verifier's query
    log.

    This form of extractor suffices for proving knowledge soundness of most hash-based IOPs.
    \lean{Extractor.Straightline}
    \uses{def:transcript}
\end{definition}

\begin{definition}[Round-by-Round Extractor]
    \label{def:rbr_extractor}
    A \textbf{round-by-round extractor} with index $m$ is given:
    \begin{itemize}
        \item the input statement $x_{\text{in}}$,
        \item a partial transcript of length $m$,
        \item the prover's query log
    \end{itemize}
    and returns a witness to the statement.

    Note that the RBR extractor does not need to take in the output statement or witness.
    \lean{Extractor.RoundByRound}
    \uses{def:transcript}
\end{definition}

\begin{definition}[Rewinding Extractor]
    \label{def:rewinding_extractor}
    A \textbf{rewinding extractor} consists of:
    \begin{itemize}
        \item An extractor state type
        \item Simulation oracles for challenges and oracle queries for the prover
        \item A function that runs the extractor with the prover's oracle interface, allowing for calling the prover multiple times
    \end{itemize}
    This allows the extractor to rewind the prover to earlier states and try different challenges.
    \lean{Extractor.Rewinding}
    \uses{def:prover, def:oracle_interface}
\end{definition}

\subsubsection{Adversarial Provers}

% \begin{definition}[Adaptive Prover]
%     \label{def:adaptive_prover}
%     An \textbf{adaptive prover} extends the basic prover type with the ability to choose the input statement adaptively based on oracle access. This models stronger adversaries that can choose their statements after seeing some oracle responses.
%     \lean{Prover.Adaptive}
%     \uses{def:prover}
% \end{definition}

\begin{definition}[State-Restoration Prover]
    \label{def:sr_prover}
    A \textbf{state-restoration prover} is a modified prover that has query access to challenge oracles that can return the $i$-th challenge, for all $i$, given the input statement and the transcript up to that point.

    It takes in the input statement and witness, and outputs a full transcript of interaction,
    along with the output statement and witness.

    This models adversaries in the state-restoration setting where challenges can be queried programmably.
    \lean{Prover.StateRestoration.KnowledgeSoundness}
    \uses{def:transcript}
\end{definition}

\subsubsection{Soundness Definitions}

For soundness, we need to consider different notions. These notions differ in two main aspects:
\begin{itemize}
    \item Whether we consider the plain soundness, or knowledge soundness. The latter relies on the
    notion of an \emph{extractor}.
    \item Whether we consider plain, state-restoration, round-by-round, or rewinding notion of
    soundness.
\end{itemize}

We note that state-restoration knowledge soundness is necessary for the security of the SNARK
protocol obtained from the oracle reduction after composing with a commitment scheme and applying
the Fiat-Shamir transform. It in turn is implied by either round-by-round knowledge soundness, or
special soundness (via rewinding). At the moment, we only care about non-rewinding soundness, so mostly we will care about round-by-round knowledge soundness.

\begin{definition}[Soundness]
    \label{def:soundness}
    A reduction satisfies \textbf{soundness} with error $\epsilon \geq 0$ and with respect to input
    language $L_{\text{in}} \subseteq \text{Statement}_{\text{in}}$ and output language $L_{\text{out}} \subseteq \text{Statement}_{\text{out}}$ if:
    \begin{itemize}
        \item for all (malicious) provers with arbitrary types for witness types,
        \item for all arbitrary input witness,
        \item for all input statement $x_{\text{in}} \notin L_{\text{in}}$,
    \end{itemize}
    the execution between the prover and the honest verifier will result in an output statement
    $x_{\text{out}} \in L_{\text{out}}$ with probability at most $\epsilon$.
    \lean{Verifier.soundness}
    \uses{def:verifier, def:prover_run}
\end{definition}

\begin{definition}[Knowledge Soundness]
    \label{def:knowledge_soundness}
    A reduction satisfies \textbf{(straightline) knowledge soundness} with error $\epsilon \geq 0$ and
    with respect to input relation $R_{\text{in}}$ and output relation $R_{\text{out}}$ if:
    \begin{itemize}
        \item there exists a straightline extractor $E$, such that
        \item for all input statement $x_{\text{in}}$, witness $w_{\text{in}}$, and (malicious) prover,
        \item if the execution with the honest verifier results in a pair $(x_{\text{out}}, w_{\text{out}})$,
        \item and the extractor produces some $w'_{\text{in}}$,
    \end{itemize}
    then the probability that $(x_{\text{in}}, w'_{\text{in}})$ is not valid for $R_{\text{in}}$ and yet $(x_{\text{out}}, w_{\text{out}})$ is valid for $R_{\text{out}}$ is at most $\epsilon$.

    A (straightline) extractor for knowledge soundness is a deterministic algorithm that takes in the output public context after executing the oracle reduction, the side information (i.e. log of oracle queries from the malicious prover) observed during execution, and outputs the witness for the input context.

    Note that since we assume the context is append-only, and we append only the public (or oracle)
    messages obtained during protocol execution, it follows that the witness stays the same throughout
    the execution.
    \lean{Verifier.knowledgeSoundness}
    \uses{def:verifier, def:reduction_run, def:straightline_extractor}
\end{definition}

\subsubsection{Round-by-Round Security}

To define round-by-round (knowledge) soundness, we need to define the notion of a \emph{state function}. This is a (possibly inefficient) function $\mathsf{StateF}$ that, for every challenge sent by the verifier, takes in the transcript of the protocol so far and outputs whether the state is doomed or not. Roughly speaking, the requirement of round-by-round soundness is that, for any (possibly malicious) prover $P$, if the state function outputs that the state is doomed on some partial transcript of the protocol, then the verifier will reject with high probability.

\begin{definition}[State Function]
    \label{def:state_function}
    A \textbf{(deterministic) state function} for a verifier, with respect to input language $L_{\text{in}}$ and
    output language $L_{\text{out}}$, consists of a function that maps partial transcripts to boolean values, satisfying:
    \begin{itemize}
        \item For all input statements not in the language, the state function is false for the empty transcript
        \item If the state function is false for a partial transcript, and the next message is from the
        prover to the verifier, then the state function is also false for the new partial transcript
        regardless of the message
        \item If the state function is false for a full transcript, the verifier will not output a statement
        in the output language
    \end{itemize}
    \lean{Verifier.StateFunction}
\end{definition}

\begin{definition}[Knowledge State Function]
    \label{def:knowledge_state_function}
    A \textbf{knowledge state function} for a verifier, with respect to input relation $R_{\text{in}}$, output
    relation $R_{\text{out}}$, and intermediate witness types, extends the basic state function to track
    witness validity throughout the protocol execution. This is used to define round-by-round knowledge soundness.
    \lean{Verifier.KnowledgeStateFunction}
\end{definition}

\begin{definition}[Round-by-Round Soundness]
    \label{def:round_by_round_soundness}
    A protocol with verifier $\mathcal{V}$ satisfies \textbf{round-by-round soundness} with respect to input language
    $L_{\text{in}}$, output language $L_{\text{out}}$, and error function $\epsilon: \text{ChallengeIdx} \to \mathbb{R}_{\geq 0}$ if:
    \begin{itemize}
        \item there exists a state function for the verifier and the input/output languages, such that
        \item for all initial statements $x_{\text{in}} \notin L_{\text{in}}$,
        \item for all initial witnesses,
        \item for all provers,
        \item for all challenge rounds $i$,
    \end{itemize}
    the probability that:
    \begin{itemize}
        \item the state function is false for the partial transcript output by the prover
        \item the state function is true for the partial transcript appended by next challenge (chosen randomly)
    \end{itemize}
    is at most $\epsilon(i)$.
    \lean{Verifier.rbrSoundness}
    \uses{def:verifier, def:state_function, def:prover_run_to_round}
\end{definition}

\begin{definition}[Round-by-Round Knowledge Soundness]
    \label{def:round_by_round_knowledge_soundness}
    A protocol with verifier $\mathcal{V}$ satisfies \textbf{round-by-round knowledge soundness} with respect to input
    relation $R_{\text{in}}$, output relation $R_{\text{out}}$, and error function $\epsilon: \text{ChallengeIdx} \to \mathbb{R}_{\geq 0}$ if:
    \begin{itemize}
        \item there exists a knowledge state function for the verifier and the languages of the input/output relations,
        \item there exists a round-by-round extractor,
        \item for all initial statements,
        \item for all initial witnesses,
        \item for all provers,
        \item for all challenge rounds $i$,
    \end{itemize}
    the probability that:
    \begin{itemize}
        \item the extracted witness does not satisfy the input relation
        \item the state function is false for the partial transcript output by the prover
        \item the state function is true for the partial transcript appended by next challenge (chosen randomly)
    \end{itemize}
    is at most $\epsilon(i)$.
    \lean{Verifier.rbrKnowledgeSoundness}
    \uses{def:verifier, def:knowledge_state_function, def:rbr_extractor, def:prover_run_to_round}
\end{definition}

% \begin{remark}[Alternative Formulations of RBR Knowledge Soundness]
%     There are different ways to formulate round-by-round knowledge soundness, differing in whether
%     the extractor's failure to produce a valid witness is included as part of the security condition.
%     Some formulations condition on the extractor producing an invalid witness while the state function
%     transitions from false to true, while others may condition on the state function transition
%     regardless of extractor success. The current formalization includes the extractor failure as
%     part of the security condition.
% \end{remark}

% \subsubsection{Extractor Properties}

% These definitions are highly experimental and may change in the future. The goal is to put some conditions on the extractor in order for prove sequential composition preserves knowledge soundness.

% \begin{definition}[Monotone Straightline Extractor]
%     \label{def:monotone_straightline_extractor}
%     An extractor is \textbf{monotone} if its success probability on a given query log is the same as
%     the success probability on any extension of that query log. This property ensures that the extractor's
%     performance does not degrade when given more information.
%     \lean{Verifier.Extractor.Straightline.IsMonotone}
%     \uses{def:straightline_extractor}
% \end{definition}

% \begin{definition}[Monotone RBR Extractor]
%     \label{def:monotone_rbr_extractor}
%     A round-by-round extractor is \textbf{monotone} if its success probability on a given query log
%     is the same as the success probability on any extension of that query log.
%     \lean{Extractor.RoundByRoundOneShot.IsMonotone}
%     \uses{def:rbr_extractor}
% \end{definition}

\subsubsection{Implications Between Security Notions}

We have a lattice of security notions, with knowledge and round-by-round being two strengthenings of soundness.

\begin{theorem}[Knowledge Soundness Implies Soundness]
    \label{thm:knowledge_soundness_implies_soundness}
    Knowledge soundness with knowledge error $\epsilon < 1$ implies soundness with the same
    soundness error $\epsilon$, and for the corresponding input and output languages.
    \lean{Verifier.knowledgeSoundness_implies_soundness}
    \uses{def:knowledge_soundness, def:soundness}
\end{theorem}

\begin{theorem}[RBR Soundness Implies Soundness]
    \label{thm:rbr_soundness_implies_soundness}
    Round-by-round soundness with error function $\epsilon$ implies soundness with error
    $\sum_i \epsilon(i)$, where the sum is over all challenge rounds $i$.
    \lean{Verifier.rbrSoundness_implies_soundness}
    \uses{def:round_by_round_soundness, def:soundness}
\end{theorem}

\begin{theorem}[RBR Knowledge Soundness Implies RBR Soundness]
    \label{thm:rbr_knowledge_soundness_implies_rbr_soundness}
    Round-by-round knowledge soundness with error function $\epsilon$ implies round-by-round
    soundness with the same error function $\epsilon$.
    \lean{Verifier.rbrKnowledgeSoundness_implies_rbrSoundness}
    \uses{def:round_by_round_knowledge_soundness, def:round_by_round_soundness}
\end{theorem}

\begin{theorem}[RBR Knowledge Soundness Implies Knowledge Soundness]
    \label{thm:rbr_knowledge_soundness_implies_knowledge_soundness}
    Round-by-round knowledge soundness with error function $\epsilon$ implies knowledge soundness
    with error $\sum_i \epsilon(i)$, where the sum is over all challenge rounds $i$.
    \lean{Verifier.rbrKnowledgeSoundness_implies_knowledgeSoundness}
    \uses{def:round_by_round_knowledge_soundness, def:knowledge_soundness}
\end{theorem}

\subsubsection{Zero-Knowledge}

\begin{definition}[Simulator]
    \label{def:simulator}
    A \textbf{simulator} consists of:
    \begin{itemize}
        \item Oracle simulation capabilities for the shared oracles
        \item A prover simulation function that takes an input statement and produces a transcript
    \end{itemize}
    The simulator should have programming access to the shared oracles and be able to generate
    transcripts that are indistinguishable from real protocol executions.
    \lean{Reduction.Simulator}
\end{definition}

\begin{remark}[Zero-Knowledge Definition]
    We define honest-verifier zero-knowledge as follows: There exists a simulator such that for all
    (malicious) verifiers, the distributions of transcripts generated by the simulator and the
    interaction between the verifier and the prover are (statistically) indistinguishable.
    A full definition will be provided in future versions.
\end{remark}

\subsubsection{Oracle-Specific Security}

For oracle reductions, the security definitions are analogous to those for standard reductions, but adapted to work with oracle interfaces:

\begin{definition}[Oracle Reduction Completeness]
    \label{def:oracle_reduction_completeness}
    Completeness of an oracle reduction is the same as for non-oracle reductions, but applied to the
    converted reduction where oracle statements are handled through their interfaces.
    \lean{OracleReduction.completeness}
    \uses{def:oracle_reduction, def:completeness, def:oracle_verifier_to_verifier}
\end{definition}

\begin{definition}[Oracle Verifier Soundness]
    \label{def:oracle_verifier_soundness}
    Soundness of an oracle verifier is defined by converting it to a standard verifier and applying
    the standard soundness definition.
    \lean{OracleVerifier.soundness}
    \uses{def:oracle_verifier, def:soundness, def:oracle_verifier_to_verifier}
\end{definition}

\begin{definition}[Oracle Verifier Knowledge Soundness]
    \label{def:oracle_verifier_knowledge_soundness}
    Knowledge soundness of an oracle verifier is defined by converting it to a standard verifier
    and applying the standard knowledge soundness definition.
    \lean{OracleVerifier.knowledgeSoundness}
    \uses{def:oracle_verifier, def:knowledge_soundness, def:oracle_verifier_to_verifier}
\end{definition}

\begin{definition}[Oracle Verifier RBR Soundness]
    \label{def:oracle_verifier_rbr_soundness}
    Round-by-round soundness of an oracle verifier is defined by converting it to a standard verifier
    and applying the standard round-by-round soundness definition.
    \lean{OracleVerifier.rbrSoundness}
    \uses{def:oracle_verifier, def:round_by_round_soundness, def:oracle_verifier_to_verifier}
\end{definition}

\begin{definition}[Oracle Verifier RBR Knowledge Soundness]
    \label{def:oracle_verifier_rbr_knowledge_soundness}
    Round-by-round knowledge soundness of an oracle verifier is defined by converting it to a standard
    verifier and applying the standard round-by-round knowledge soundness definition.
    \lean{OracleVerifier.rbrKnowledgeSoundness}
    \uses{def:oracle_verifier, def:round_by_round_knowledge_soundness, def:oracle_verifier_to_verifier}
\end{definition}

By default, the properties we consider are perfect completeness and (straightline) round-by-round knowledge soundness. We can encapsulate these properties into the following typing judgement:

\[
    \Gamma := (\Psi; \Theta; \varSigma; \rho; \mathcal{O}) \vdash \{\mathcal{R}_1\} \quad \langle\mathcal{P}, \mathcal{V}, \mathcal{E}\rangle \quad \{\!\!\{\mathcal{R}_2; \mathsf{St}; \epsilon\}\!\!\}
\]

\subsubsection{State-Restoration Security}

\begin{definition}[State-Restoration Soundness]
    \label{def:sr_soundness}
    \textbf{State-restoration soundness} is a security notion where the adversarial prover has access to
    challenge oracles that can return the $i$-th challenge for any round $i$, given the input statement
    and the transcript up to that point. This models stronger adversaries in the programmable random
    oracle model or when challenges can be computed deterministically.

    A verifier satisfies state-restoration soundness if for all input statements not in the language,
    for all witnesses, and for all state-restoration provers, the probability that the verifier
    outputs a statement in the output language is bounded by the soundness error.

    \emph{Note: This definition is currently under development in the Lean formalization.}
    % \lean{Verifier.srSoundness}
\end{definition}

\begin{definition}[State-Restoration Knowledge Soundness]
    \label{def:sr_knowledge_soundness}
    \textbf{State-restoration knowledge soundness} extends state-restoration soundness with the
    requirement that there exists a straightline extractor that can recover valid witnesses from
    any state-restoration prover that convinces the verifier.

    \emph{Note: This definition is currently under development in the Lean formalization.}
    % \lean{Verifier.srKnowledgeSoundness}
\end{definition}


% We can now define properties of interactive reductions. The two main properties we consider in this
% project are completeness and various notions of soundness. We will cover zero-knowledge at a later
% stage.

% First, for completeness, this is essentially probabilistic Hoare-style conditions on the execution
% of the oracle reduction (with the honest prover and verifier). In other words, given a predicate on
% the initial context, and a predicate on the final context, we require that if the initial predicate
% holds, then the final predicate holds with high probability (except for some \emph{completeness}
% error).

% \begin{definition}[Completeness]
%     \label{def:completeness}
%     \lean{Reduction.completeness}
%     \uses{def:oracle_reduction}
% \end{definition}

% Almost all oracle reductions we consider actually satisfy \emph{perfect completeness}, which
% simplifies the proof obligation. In particular, this means we only need to show that no matter what challenges are chosen, the verifier will always accept given messages from the honest prover.

% For soundness, we need to consider different notions. These notions differ in two main aspects:
% \begin{itemize}
%     \item Whether we consider the plain soundness, or knowledge soundness. The latter relies on the
%     notion of an \emph{extractor}.
%     \item Whether we consider plain, state-restoration, round-by-round, or rewinding notion of
%     soundness.
% \end{itemize}

% We note that state-restoration knowledge soundness is necessary for the security of the SNARK
% protocol obtained from the oracle reduction after composing with a commitment scheme and applying
% the Fiat-Shamir transform. It in turn is implied by either round-by-round knowledge soundness, or
% special soundness (via rewinding). At the moment, we only care about non-rewinding soundness, so mostly we will care about round-by-round knowledge soundness.

% \begin{definition}[Soundness]
%     \label{def:soundness}
%     \lean{Verifier.soundness}
%     \uses{def:oracle_reduction}
% \end{definition}

% A (straightline) extractor for knowledge soundness is a deterministic algorithm that takes in the output public context after executing the oracle reduction, the side information (i.e. log of oracle queries from the malicious prover) observed during execution, and outputs the witness for the input context.

% Note that since we assume the context is append-only, and we append only the public (or oracle)
% messages obtained during protocol execution, it follows that the witness stays the same throughout
% the execution.

% \begin{definition}[Knowledge Soundness]
%     \label{def:knowledge_soundness}
%     \lean{Verifier.knowledgeSoundness}
%     \uses{def:oracle_reduction}
% \end{definition}

% To define round-by-round (knowledge) soundness, we need to define the notion of a \emph{state function}. This is a (possibly inefficient) function $\mathsf{StateF}$ that, for every challenge sent by the verifier, takes in the transcript of the protocol so far and outputs whether the state is doomed or not. Roughly speaking, the requirement of round-by-round soundness is that, for any (possibly malicious) prover $P$, if the state function outputs that the state is doomed on some partial transcript of the protocol, then the verifier will reject with high probability.

% \begin{definition}[State Function]
%     \label{def:state_function}
%     \lean{Verifier.StateFunction}
% \end{definition}

% \begin{definition}[Round-by-Round Soundness]
%     \label{def:round_by_round_soundness}
%     \lean{Verifier.rbrSoundness}
%     \uses{def:oracle_reduction}
% \end{definition}

% \begin{definition}[Round-by-Round Knowledge Soundness]
%     \label{def:round_by_round_knowledge_soundness}
%     \lean{Verifier.rbrKnowledgeSoundness}
%     \uses{def:oracle_reduction}
% \end{definition}

% \textbf{PL Formalization.} We write our definitions in PL notation in~\Cref{fig:type-defs}. The set of types $\Type$ is the same as Lean's dependent type theory (omitting universe levels); in particular, we care about basic dependent types (Pi and Sigma), finite natural numbers, finite fields, lists, vectors, and polynomials.

% \begin{figure}[t]
%     \[\begin{array}{rcl}
%         % Basic types
%         \mathsf{Type} &::=& \mathsf{Unit} \mid \mathsf{Bool} \mid \mathbb{N} \mid \mathsf{Fin}\; n \mid \mathbb{F}_q \mid \mathsf{List}\;(\alpha : \mathsf{Type}) \mid (i : \iota) \to \alpha\; i \mid (i : \iota) \times \alpha\; i \mid \dots \\[1em]
%         % Protocol message types
%         \mathsf{Dir} &::=& \mathsf{P2V.Pub} \mid \mathsf{P2V.Orac} \mid \mathsf{V2P} \\
%         \mathsf{OI}\; (\mathrm{M} : \Type) &::=& \langle \mathrm{Q}, \mathrm{R}, \mathrm{M} \to \mathrm{Q} \to \mathrm{R} \rangle \\
%         % Protocol type signature
%         \pSpec\; (n : \mathbb{N}) &::=& \mathsf{Fin}\; n \to (d : \mathsf{Dir}) \times (M : \Type) \times (\mathsf{if}\; d = \mathsf{P2V.Orac} \; \mathsf{then} \; \mathsf{OI}(M) \; \mathsf{else} \; \mathsf{Unit}) \\
%         % Oracle type signature
%         \oSpec \; (\iota : \mathsf{Type}) &::=& (i : \iota) \to \mathsf{dom}\; i \times \mathsf{range}\; i \\[1em]
%         % Contexts
%         \varSigma &::=& \emptyset \mid \varSigma \times \Type \\
%         \Omega &::=& \emptyset \mid \Omega \times \langle \mathrm{M} : \Type, \mathsf{OI}(\mathrm{M}) \rangle \\
%         \Psi &::=& \emptyset \mid \Psi \times \Type\\
%     \end{array}\]
%     \[\begin{array}{rcl}
%         \Gamma &::=& (\Psi; \Omega; \varSigma; \rho; \mathcal{O})\\
%         \mathsf{OComp}^{\mathcal{O}}\; (\alpha : \Type) &::=& \mid\; \mathsf{pure}\; (a : \alpha) \\
%         && \mid\; \mathsf{queryBind}\;(i : \iota)\; (q : \mathsf{dom}\; i)\; (k : \mathsf{range}\; i \to \mathsf{OComp}^{\mathcal{O}}\; \alpha) \\
%         && \mid\; \mathsf{fail} \\[1em]
%         \tau_{\mathsf{P}}(\Gamma) &::=& (i : \mathsf{Fin}\; n) \to (h : (\rho \; i).\mathsf{fst} = \mathsf{P2V}) \to \\
%         && \varSigma \to \Omega \to \Psi \to \rho_{[:i]} \to \mathsf{OComp}^{\mathcal{O}}\;\left( (\rho \; i).\mathsf{snd}\right) \\[1em]

%         \tau_{\mathsf{V}}(\Gamma) &::=& \varSigma \to (\rho.\mathsf{Chals}) \to \mathsf{OComp}^{\mathcal{O} :: \OI(\Omega) :: \OI(\rho.\mathsf{Msg.Orac})}\; \mathsf{Unit} \\[1em]
%         \tau_{\mathsf{E}}(\Gamma) &::=& \varSigma \to \Omega \to \rho.\mathsf{Transcript} \to \calO.\mathsf{QueryLog} \to \Psi
%     \end{array}\]
%     \caption{Type definitions for interactive oracle reductions}
%     \label{fig:type-defs}
% \end{figure}

% Using programming language notation, we can express an interactive oracle reduction as a typing judgment:
% \[
%     \Gamma := (\Psi; \Theta; \varSigma; \rho; \mathcal{O}) \vdash \mathcal{P} : \tau_{\mathsf{P}}(\Gamma), \; \mathcal{V} : \tau_{\mathsf{V}}(\Gamma)
% \]
% where:
% \begin{itemize}
%     \item $\Psi$ represents the witness (private) inputs
%     \item $\Theta$ represents the oracle inputs
%     \item $\varSigma$ represents the public inputs (i.e. statements)
%     \item $\mathcal{O} : \oSpec\; \iota$ represents the shared oracle
%     \item $\rho : \pSpec\; n$ represents the protocol type signature
%     \item $\mathcal{P}$ and $\mathcal{V}$ are the prover and verifier, respectively, being of the given types $\tau_{\mathsf{P}}(\Gamma)$ and $\tau_{\mathsf{V}}(\Gamma)$.
% \end{itemize}

% To exhibit valid elements for the prover and verifier types, we will use existing functions in the ambient programming language (e.g. Lean).

% By default, the properties we consider are perfect completeness and (straightline) round-by-round knowledge soundness. We can encapsulate these properties into the following typing judgement:

% \[
%     \Gamma := (\Psi; \Theta; \varSigma; \rho; \mathcal{O}) \vdash \{\mathcal{R}_1\} \quad \langle\mathcal{P}, \mathcal{V}, \mathcal{E}\rangle \quad \{\!\!\{\mathcal{R}_2; \mathsf{St}; \epsilon\}\!\!\}
% \]


\section{Definitions}\label{sec:oracle_reductions_defs}

In this section, we give the basic definitions of a public-coin interactive oracle reduction
(henceforth called an oracle reduction or IOR). We will define its building blocks, and various
security properties.

\subsection{Format}\label{sec:oracle_reductions_defs_format}

An \textbf{(interactive) oracle reduction (IOR)} is an interactive protocol between two parties, a
\emph{prover} $\mathcal{P}$ and a \emph{verifier} $\mathcal{V}$. In ArkLib, IORs are defined in the
following setting:
\begin{enumerate}
    \item We work in an ambient dependent type theory (in our case, Lean).

    \item The protocol flow is fixed and defined by a given \emph{type signature}, which
    describes in each round which party sends a message to the other, and the type of that message.

    \item The prover and verifier has access to some inputs (called the \emph{(oracle) context}) at
    the beginning of the protocol. These inputs are classified as follows:
    \begin{itemize}
        \item \emph{Public inputs} (or \emph{statement}) $\mathbbm{x}$: available to both parties;
        \item \emph{Private inputs} (or \emph{witness}) $\mathbbm{w}$: available only to the prover;
        \item \emph{Oracle inputs} (or \emph{oracle statement}) $\mathbbm{ox}$: the underlying data
        is available to the prover, but it's only exposed as an oracle to the verifier. See~\Cref{def:oracle_interface} for more information.
        \item \emph{Shared oracle} $\mathcal{O}$: the oracle is available to both parties via an
        interface; in most cases, it is either empty, a probabilistic sampling oracle, a random
        oracle, or a group oracle (for the Algebraic Group Model). See~\Cref{sec:vcvio} for more
        information on oracle computations.
    \end{itemize}

    \item The messages sent from the prover may either: 1) be seen directly by the verifier, or 2)
    only available to a verifier through an \emph{oracle interface} (which specifies the type for
    the query and response, and the oracle's behavior given the underlying message).

    Currently, in the oracle reduction setting, we \emph{only} allow messages sent to be available
    through oracle interfaces. In the (non-oracle) reduction setting, all messages are available
    directly. Future extensions may allow for mixed visibility for prover's messages.

    \item $\mathcal{V}$ is assumed to be \emph{public-coin}, meaning that its challenges are chosen
    uniformly at random from the finite type corresponding to that round, and it uses no randomness
    otherwise, except from those coming from the shared oracle.

    \item At the end of the protocol, the prover and verifier outputs a new (oracle) context, which consists of:
    \begin{itemize}
        \item The verifier takes in the input statement and the challenges, performs an \emph{oracle} computation on the input oracle statements and the oracle messages, and outputs a new output statement.

        The verifier also outputs the new oracle statement in an implicit manner, by specifying a
        subset of the input oracle statements \& the oracle messages. Future extensions may allow for more flexibility in specifying output oracle statements (i.e. not just a subset, but a linear combination, or any other function).
        \item The prover takes in some final private state (maintained during protocol execution), and outputs a new output statement, new output oracle statement, and new output witness.
    \end{itemize}
\end{enumerate}

\begin{remark}[Literature Comparison]
In the literature, our definition corresponds to the notion of \emph{functional} IORs. Historically,
(vector) IOPs were the first notion to be introduced by~\cite{IOPs}; these are IORs where the output
statement is true/false, all oracle statements and messages are vectors over some alphabet $\Sigma$,
and the oracle interfaces are for querying specific positions in the vector. More recent works have
considered other oracle interfaces, e.g., polynomial oracles~\cite{Marlin, DARK}, generalized proofs
to reductions~\cite{ARoK, WARP, Arc, fics-facs}, and considered general oracle
interfaces~\cite{WHIR}. Most of the IOP theory has been distilled in the
textbook~\cite{ChiesaYogev2024}.

We have not seen any work that considers our most general setting, of IORs with arbitrary oracle interfaces.
\end{remark}

We now go into more details on these objects, and how they are represented in Lean. Our description will aim to be as close as possible to the Lean code, and hence may differ somewhat from ``mainstream'' mathematical \& cryptographic notation.

\begin{definition}[Oracle Interface]
    \label{def:oracle_interface}
    An oracle interface for an underlying data type $\mathsf{D}$ consists of the following:
    \begin{itemize}
        \item A type $\mathsf{Q}$ for queries to the oracle,
        \item A type $\mathsf{R}$ for responses from the oracle,
        \item A function $\mathsf{oracle} : \mathsf{D} \to \mathsf{Q} \to \mathsf{R}$ that specifies
        the oracle's behavior given the underlying data and a query.
    \end{itemize}
    \lean{OracleInterface}
\end{definition}

See \texttt{OracleInterface.lean} for common instances of $\mathsf{OracleInterface}$.


\begin{definition}[Context]
    \label{def:context}
    In an (oracle) reduction, its \emph{(oracle) context} consists of a statement type, a witness
    type, and (in the oracle case) an indexed list of oracle statement types.

    Currently, we do not abstract out / bundle the context as a separate structure, but rather
    specifies the types explicitly. This may change in the future.
\end{definition}

\begin{definition}[Protocol Specification]
    \label{def:protocol_spec}
    A protocol specification for an $n$-message (oracle) reduction, is an element of the following type:
    \begin{align*}
        \ProtocolSpec\ n &:= \Fin\ n \to \Direction \times \Type.
    \end{align*}
    In the above, $\Direction := \{ \PtoVdir, \VtoPdir \}$ is the type of possible directions of messages, and $\Fin\ n := \{ i : \bbN \quotient i < n \}$ is the type of all natural numbers less than $n$.

    In other words, for each step $i$ of interaction, the protocol specification describes the \emph{direction} of the message sent in that step, i.e., whether it is from the prover or from the verifier. It also describes the \emph{type} of that message.

    In the oracle setting, we also expect an oracle interface for each message from the prover to the verifier.
    \lean{ProtocolSpec}
\end{definition}

We define some supporting definitions for a protocol specification.

\begin{definition}[Protocol Specification Components]
    \label{def:protocol_spec_components}
    Given a protocol spec $\pSpec : \ProtocolSpec\ n$, we define:
    \begin{itemize}
        \item $\pSpec.\Dir\ i := (\pSpec\ i).\mathsf{fst}$ extracts the direction of the $i$-th message.
        \item $\pSpec.\Type\ i := (\pSpec\ i).\mathsf{snd}$ extracts the type of the $i$-th message.
        \item $\pSpec.\MessageIdx := \{i : \Fin\ n \quotient \pSpec.\Dir\ i = \PtoVdir\}$ is the subtype of indices corresponding to prover messages.
        \item $\pSpec.\ChallengeIdx := \{i : \Fin\ n \quotient \pSpec.\Dir\ i = \VtoPdir\}$ is the subtype of indices corresponding to verifier challenges.
        \item $\pSpec.\mathsf{Message}\ i := (i : \pSpec.\MessageIdx) \to \pSpec.\Type\ i.\mathsf{val}$ is an indexed family of message types in the protocol.
        \item $\pSpec.\mathsf{Challenge}\ i := (i : \pSpec.\ChallengeIdx) \to \pSpec.\Type\ i.\mathsf{val}$ is an indexed family of challenge types in the protocol.
    \end{itemize}
    \lean{ProtocolSpec.dir, ProtocolSpec.Type, ProtocolSpec.MessageIdx, ProtocolSpec.ChallengeIdx, ProtocolSpec.Message, ProtocolSpec.Challenge}
    \uses{def:protocol_spec}
\end{definition}

\begin{definition}[Protocol Transcript]
    \label{def:transcript}
        Given protocol specification $\pSpec : \ProtocolSpec\ n$, we define:
    \begin{itemize}
        \item A \emph{transcript} up to round $k : \Fin\ (n + 1)$ is an element of type
                \[ \Transcript\ k\ \pSpec := (i : \Fin\ k) \to \pSpec.\Type\ (\uparrow i : \Fin\ n) \]
        where $\uparrow i : \Fin\ n$ denotes casting $i : \Fin\ k$ to $\Fin\ n$ (valid since $k \leq n + 1$).

        \item A \emph{full transcript} is $\FullTranscript\ \pSpec := (i : \Fin\ n) \to \pSpec.\Type\ i$.

        \item The type of all \emph{messages} from prover to verifier is
        \[ \pSpec.\Messages := \prod_{i : \pSpec.\MessageIdx} \pSpec.\Message\ i \]

        \item The type of all \emph{challenges} from verifier to prover is
        \[ \pSpec.\Challenges := \prod_{i : \pSpec.\ChallengeIdx} \pSpec.\Challenge\ i \]
    \end{itemize}
    \lean{ProtocolSpec.Transcript, ProtocolSpec.Message, ProtocolSpec.Challenge}
    \uses{def:protocol_spec, def:protocol_spec_components}
\end{definition}

% In the interactive protocols we consider, both parties $P$ and $V$ may have access to a shared
% oracle $O$. An interactive protocol becomes an \emph{interactive (oracle) reduction} if its
% execution reduces an input relation $R_{\mathsf{in}}$ to an output relation $R_{\mathsf{out}}$. Here
% a relation is just a function $\mathsf{IsValid}: \mathsf{Statement} \times \mathsf{Witness} \to
% \mathsf{Bool}$, for some types \verb|Statement| and \verb|Witness|. We do not concern ourselves with
% the running time of $\mathsf{IsValid}$ in this project (though future extensions may prove that
% relations can be decided in polynomial time, for a suitable model of computation).

\begin{remark}[Design Decision]
    We do not enforce a particular interaction flow in the definition of an interactive (oracle) reduction. This is done so that we can capture all protocols in the most generality. Also, we want to allow the prover to send multiple messages in a row, since each message may have a different oracle representation (for instance, in the Plonk protocol, the prover's first message is a 3-tuple of polynomial commitments.)
\end{remark}

\begin{definition}[Type Signature of a Prover]
    \label{def:prover}
    A prover $\mathcal{P}$ in a reduction consists of the following components:

    \begin{itemize}
        \item \textbf{Prover State}: A family of types $\mathsf{PrvState} : \Fin(n+1) \to \Type$ representing the prover's internal state at each round of the protocol.

        \item \textbf{Input Processing}: A function
        \[ \mathsf{input} : \StmtIn \to \WitIn \to \mathsf{PrvState}(0) \]
        that initializes the prover's state from the input statement and witness.

        \item \textbf{Message Sending}: For each message index $i : \pSpec.\MessageIdx$, a function
        \[ \mathsf{sendMessage}_i : \mathsf{PrvState}(i.\mathsf{val}.\mathsf{castSucc}) \to \OracleComp(\oSpec, \pSpec.\Message(i) \times \mathsf{PrvState}(i.\mathsf{val}.\mathsf{succ})) \]
        that generates the message and updates the prover's state.

        \item \textbf{Challenge Processing}: For each challenge index $i : \pSpec.\ChallengeIdx$, a function
        \[ \mathsf{receiveChallenge}_i : \mathsf{PrvState}(i.\mathsf{val}.\mathsf{castSucc}) \to \pSpec.\Challenge(i) \to \mathsf{PrvState}(i.\mathsf{val}.\mathsf{succ}) \]
        that updates the prover's state upon receiving a challenge.

        \item \textbf{Output Generation}: A function
        \[ \mathsf{output} : \mathsf{PrvState}(\Fin.\mathsf{last}(n)) \to \StmtOut \times \WitOut \]
        that produces the final output statement and witness from the prover's final state.
    \end{itemize}
    \lean{Prover, ProverState, ProverInput, ProverRound, ProverOutput}
\end{definition}

\begin{definition}[Type Signature of an Oracle Prover]
    \label{def:oracle_prover}
    An oracle prover is a prover whose input statement includes the underlying data for oracle statements, and whose output includes oracle statements. Formally, it is a prover with input statement type $\StmtIn \times (\forall i : \iota_{\mathsf{si}}, \OStmtIn(i))$ and output statement type $\StmtOut \times (\forall i : \iota_{\mathsf{so}}, \OStmtOut(i))$, where:
    \begin{itemize}
        \item $\OStmtIn : \iota_{\mathsf{si}} \to \Type$ are the input oracle statement types
        \item $\OStmtOut : \iota_{\mathsf{so}} \to \Type$ are the output oracle statement types
    \end{itemize}
    \lean{OracleProver}
\end{definition}

% Our modeling of oracle reductions only consider \emph{public-coin} verifiers; that is, verifiers who
% only outputs uniformly random challenges drawn from the (finite) types, and uses no other
% randomness. Because of this fixed functionality, we can bake the verifier's behavior in the
% interaction phase directly into the protocol execution semantics.

After the interaction phase, the verifier may then run some verification procedure to check the
validity of the prover's responses. In this procedure, the verifier gets access to the public part
of the context, and oracle access to either the shared oracle, or the oracle inputs.
% This procedure differs depending on whether the verifier has
% full access, or only oracle access, to the prover's messages. Note that there is no difference on
% the prover side whether the protocol is an \emph{interactive oracle reduction (IOR)} or simply an
% \emph{interactive reduction (IR)}.

\begin{definition}[Type Signature of a Verifier]
    \label{def:verifier}
    A verifier $\mathcal{V}$ in a reduction is specified by a single function:
    \[ \mathsf{verify} : \StmtIn \to \FullTranscript(\pSpec) \to \OracleComp(\oSpec, \StmtOut) \]

    This function takes the input statement and the complete transcript of the protocol interaction, and performs an oracle computation (potentially querying the shared oracle $\oSpec$) to produce an output statement.

    The verifier is assumed to be \emph{public-coin}, meaning it only sends uniformly random challenges and uses no other randomness beyond what is provided by the shared oracle.
    \lean{Verifier}
\end{definition}

\begin{definition}[Type Signature of an Oracle Verifier]
    \label{def:oracle_verifier}
    An oracle verifier $\mathcal{V}$ consists of the following components:

    \begin{itemize}
        \item \textbf{Verification Logic}: A function
        \[ \mathsf{verify} : \StmtIn \to \pSpec.\Challenges \to \OracleComp(\oSpec \mathrel{++_\mathsf{o}} ([\OStmtIn]_\mathsf{o} \mathrel{++_\mathsf{o}} [\pSpec.\Message]_\mathsf{o}), \StmtOut) \]
        that takes the input statement and verifier challenges, and performs oracle queries to the shared oracle, input oracle statements, and prover messages to produce an output statement.

        \item \textbf{Output Oracle Embedding}: An injective function
        \[ \mathsf{embed} : \iota_{\mathsf{so}} \hookrightarrow \iota_{\mathsf{si}} \oplus \pSpec.\MessageIdx \]
        that specifies how each output oracle statement is derived from either an input oracle statement or a prover message.

        \item \textbf{Type Compatibility}: A proof term
        \[ \mathsf{hEq} : \forall i : \iota_{\mathsf{so}}, \OStmtOut(i) = \begin{cases}
            \OStmtIn(j) & \text{if } \mathsf{embed}(i) = \mathsf{inl}(j) \\
            \pSpec.\Message(k) & \text{if } \mathsf{embed}(i) = \mathsf{inr}(k)
        \end{cases} \]
        ensuring that output oracle statement types match their sources.
    \end{itemize}

    This design ensures that output oracle statements are always a subset of the available input oracle statements and prover messages.
    \lean{OracleVerifier}
\end{definition}

\begin{definition}[Oracle Verifier to Verifier Conversion]
    \label{def:oracle_verifier_to_verifier}
    An oracle verifier can be converted to a standard verifier through a natural simulation process. The key insight is that while an oracle verifier only has oracle access to certain data (input oracle statements and prover messages), a standard verifier can be given the actual underlying data directly.

    The conversion works as follows: when the oracle verifier needs to make an oracle query to some data, the converted verifier can respond to this query immediately using the actual underlying data it possesses. This is accomplished through the \texttt{OracleInterface} type class, which specifies for each data type how to respond to queries given the underlying data.

    Specifically, given an oracle verifier $\mathcal{V}_{\text{oracle}}$:
    \begin{itemize}
        \item The converted verifier $\mathcal{V}_{\text{oracle}}.\mathsf{toVerifier}$ takes as input both the statement \emph{and} the actual underlying data for all oracle statements
        \item When $\mathcal{V}_{\text{oracle}}$ attempts to query an oracle statement or prover message, the converted verifier uses the corresponding \texttt{OracleInterface} instance to compute the response from the actual data
        \item The output oracle statements are constructed according to the embedding specification, selecting the appropriate subset of input oracle statements and prover messages
    \end{itemize}
    \lean{OracleVerifier.toVerifier}
    \uses{def:oracle_verifier}
\end{definition}

An oracle reduction then consists of a type signature for the interaction, and a pair of prover and
verifier for that type signature.

\begin{definition}[Interactive Reduction]
    \label{def:reduction}
    An interactive reduction for protocol specification $\pSpec : \ProtocolSpec(n)$ and oracle specification $\oSpec$ consists of:
    \begin{itemize}
        \item A \textbf{prover} $\mathcal{P} : \Prover(\pSpec, \oSpec, \StmtIn, \WitIn, \StmtOut, \WitOut)$
        \item A \textbf{verifier} $\mathcal{V} : \Verifier(\pSpec, \oSpec, \StmtIn, \StmtOut)$
    \end{itemize}

    The reduction establishes a relationship between input relations on $(\StmtIn, \WitIn)$ and output relations on $(\StmtOut, \WitOut)$ through the interactive protocol defined by $\pSpec$.
    \lean{Reduction}
    \uses{def:prover, def:verifier}
\end{definition}

\begin{definition}[Interactive Oracle Reduction]
    \label{def:oracle_reduction}
    An interactive oracle reduction for protocol specification $\pSpec : \ProtocolSpec(n)$ with oracle interfaces for all prover messages, and oracle specification $\oSpec$, consists of:
    \begin{itemize}
        \item An \textbf{oracle prover} $\mathcal{P} : \OracleProver(\pSpec, \oSpec, \StmtIn, \WitIn, \StmtOut, \WitOut, \OStmtIn, \OStmtOut)$
        \item An \textbf{oracle verifier} $\mathcal{V} : \OracleVerifier(\pSpec, \oSpec, \StmtIn, \StmtOut, \OStmtIn, \OStmtOut)$
    \end{itemize}

    where:
    \begin{itemize}
        \item $\OStmtIn : \iota_{\mathsf{si}} \to \Type$ are the input oracle statement types with oracle interfaces
        \item $\OStmtOut : \iota_{\mathsf{so}} \to \Type$ are the output oracle statement types
    \end{itemize}

    The oracle reduction allows the verifier to access prover messages and oracle statements only through specified oracle interfaces, enabling more flexible and composable protocol designs.
    \lean{OracleReduction}
    \uses{def:oracle_prover, def:oracle_verifier}
\end{definition}

\subsection{Execution Semantics}\label{sec:execution_semantics}

We now define what it means to execute an oracle reduction. This is essentially achieved by first
executing the prover, interspersed with oracle queries to get the verifier's challenges (these will
be given uniform random probability semantics later on), and then executing the verifier's checks.
Any message exchanged in the protocol will be added to the context. We may also log information
about the execution, such as the log of oracle queries for the shared oracles, for analysis purposes
(i.e. feeding information into the extractor).

\begin{definition}[Prover Execution to Round]
    \label{def:prover_run_to_round}
    The execution of a prover up to round $i : \Fin(n+1)$ is defined inductively:

    \[ \mathsf{Prover}.\mathsf{runToRound}(i, \mathsf{stmt}, \mathsf{wit}) := \]
    \[ \mathsf{Fin}.\mathsf{induction}( \]
    \[ \quad \mathsf{pure}(\langle \mathsf{default}, \mathsf{prover}.\mathsf{input}(\mathsf{stmt}, \mathsf{wit}) \rangle), \]
    \[ \quad \mathsf{prover}.\mathsf{processRound}, \]
    \[ \quad i \]
    \[ ) \]

    where $\mathsf{processRound}$ handles individual rounds by either:
    \begin{itemize}
        \item \textbf{Verifier Challenge} ($\pSpec.\mathsf{dir}(j) = \mathsf{V\_to\_P}$): Query for a challenge and update prover state
        \item \textbf{Prover Message} ($\pSpec.\mathsf{dir}(j) = \mathsf{P\_to\_V}$): Generate message via $\mathsf{sendMessage}$ and update state
    \end{itemize}

    Returns the transcript up to round $i$ and the prover's state after round $i$.
    \lean{Prover.runToRound, Prover.processRound}
    \uses{def:prover, def:protocol_spec, def:transcript}
\end{definition}

\begin{definition}[Complete Prover Execution]
    \label{def:prover_run}
    The complete execution of a prover is defined as:

    \[ \mathsf{Prover}.\mathsf{run}(\mathsf{stmt}, \mathsf{wit}) := \mathsf{do} \; \{ \]
    \[ \quad \langle \mathsf{transcript}, \mathsf{state} \rangle \leftarrow \mathsf{prover}.\mathsf{runToRound}(\Fin.\mathsf{last}(n), \mathsf{stmt}, \mathsf{wit}) \]
    \[ \quad \langle \mathsf{stmtOut}, \mathsf{witOut} \rangle := \mathsf{prover}.\mathsf{output}(\mathsf{state}) \]
    \[ \quad \mathsf{return} \; \langle \mathsf{stmtOut}, \mathsf{witOut}, \mathsf{transcript} \rangle \]
    \[ \} \]

    Returns the output statement, output witness, and complete transcript.
    \lean{Prover.run}
    \uses{def:prover, def:prover_run_to_round}
\end{definition}

\begin{definition}[Verifier Execution]
    \label{def:verifier_run}
    The execution of a verifier is simply the application of its verification function:

    \[ \mathsf{Verifier}.\mathsf{run}(\mathsf{stmt}, \mathsf{transcript}) := \mathsf{verifier}.\mathsf{verify}(\mathsf{stmt}, \mathsf{transcript}) \]

    This takes the input statement and full transcript, and returns the output statement via an oracle computation.
    \lean{Verifier.run}
    \uses{def:verifier}
\end{definition}

\begin{definition}[Oracle Verifier Execution]
    \label{def:oracle_verifier_run}
    The execution of an oracle verifier is defined as:

    \[ \mathsf{OracleVerifier}.\mathsf{run}(\mathsf{stmt}, \mathsf{oStmtIn}, \mathsf{transcript}) := \mathsf{do} \; \{ \]
    \[ \quad \mathsf{f} := \mathsf{simOracle2}(\oSpec, \mathsf{oStmtIn}, \mathsf{transcript}.\mathsf{messages}) \]
    \[ \quad \mathsf{stmtOut} \leftarrow \mathsf{simulateQ}(\mathsf{f}, \mathsf{verifier}.\mathsf{verify}(\mathsf{stmt}, \mathsf{transcript}.\mathsf{challenges})) \]
    \[ \quad \mathsf{return} \; \mathsf{stmtOut} \]
    \[ \} \]

    This simulates the oracle access to input oracle statements and prover messages, then executes the verification logic.
    \lean{OracleVerifier.run}
    \uses{def:oracle_verifier, def:oracle_interface}
\end{definition}

\begin{definition}[Interactive Reduction Execution]
    \label{def:reduction_run}
    The execution of an interactive reduction consists of running the prover followed by the verifier:

    \[ \mathsf{Reduction}.\mathsf{run}(\mathsf{stmt}, \mathsf{wit}) := \mathsf{do} \; \{ \]
    \[ \quad \langle \mathsf{prvStmtOut}, \mathsf{witOut}, \mathsf{transcript} \rangle \leftarrow \mathsf{reduction}.\mathsf{prover}.\mathsf{run}(\mathsf{stmt}, \mathsf{wit}) \]
    \[ \quad \mathsf{stmtOut} \leftarrow \mathsf{reduction}.\mathsf{verifier}.\mathsf{run}(\mathsf{stmt}, \mathsf{transcript}) \]
    \[ \quad \mathsf{return} \; ((\mathsf{prvStmtOut}, \mathsf{witOut}), \mathsf{stmtOut}, \mathsf{transcript}) \]
    \[ \} \]

    Returns both the prover's output (statement and witness) and the verifier's output statement, along with the complete transcript.
    \lean{Reduction.run}
    \uses{def:reduction, def:prover_run, def:verifier_run}
\end{definition}

\begin{definition}[Oracle Reduction Execution]
    \label{def:oracle_reduction_run}
    The execution of an interactive oracle reduction is similar to a standard reduction but includes logging of oracle queries:

    \[ \mathsf{OracleReduction}.\mathsf{run}(\mathsf{stmt}, \mathsf{wit}, \mathsf{oStmt}) := \mathsf{do} \; \{ \]
    \[ \quad \langle \langle \mathsf{prvStmtOut}, \mathsf{witOut}, \mathsf{transcript} \rangle, \mathsf{proveQueryLog} \rangle \leftarrow \]
    \[ \qquad (\mathsf{simulateQ}(\mathsf{loggingOracle}, \mathsf{reduction}.\mathsf{prover}.\mathsf{run}(\langle \mathsf{stmt}, \mathsf{oStmt} \rangle, \mathsf{wit}))).\mathsf{run} \]
    \[ \quad \langle \mathsf{stmtOut}, \mathsf{verifyQueryLog} \rangle \leftarrow \]
    \[ \qquad (\mathsf{simulateQ}(\mathsf{loggingOracle}, \mathsf{reduction}.\mathsf{verifier}.\mathsf{run}(\mathsf{stmt}, \mathsf{oStmt}, \mathsf{transcript}))).\mathsf{run} \]
    \[ \quad \mathsf{return} \; ((\mathsf{prvStmtOut}, \mathsf{witOut}), \mathsf{stmtOut}, \mathsf{transcript}, \mathsf{proveQueryLog}, \mathsf{verifyQueryLog}) \]
    \[ \} \]

    Returns the same outputs as a standard reduction, plus logs of all oracle queries made by both the prover and verifier.
    \lean{OracleReduction.run}
    \uses{def:oracle_reduction, def:prover_run, def:oracle_verifier_run}
\end{definition}


\subsection{Security Properties}\label{sec:security}

We can now define properties of interactive reductions. The two main properties we consider in this
project are completeness and various notions of soundness. We will cover zero-knowledge at a later
stage.

First, for completeness, this is essentially probabilistic Hoare-style conditions on the execution
of the oracle reduction (with the honest prover and verifier). In other words, given a predicate on
the initial context, and a predicate on the final context, we require that if the initial predicate
holds, then the final predicate holds with high probability (except for some \emph{completeness}
error).

\begin{definition}[Completeness]
    \label{def:completeness}
    A reduction satisfies \textbf{completeness} with error $\epsilon \geq 0$ and with respect to
    input relation $R_{\text{in}}$ and output relation $R_{\text{out}}$, if for all valid statement-witness pair
    $(x_{\text{in}}, w_{\text{in}})$ for $R_{\text{in}}$, the execution between the honest prover and the honest verifier
    will result in a tuple $((x_{\text{out}}^P, w_{\text{out}}), x_{\text{out}}^V)$ such that:
    \begin{itemize}
        \item $R_{\text{out}}(x_{\text{out}}^V, w_{\text{out}}) = \text{True}$ (the output statement-witness pair is valid), and
        \item $x_{\text{out}}^P = x_{\text{out}}^V$ (the output statements are the same from both prover and verifier)
    \end{itemize}
    except with probability $\epsilon$.
    \lean{Reduction.completeness}
    \uses{def:reduction, def:reduction_run}
\end{definition}

\begin{definition}[Perfect Completeness]
    \label{def:perfect_completeness}
    A reduction satisfies \textbf{perfect completeness} if it satisfies completeness with error $0$.
    This means that the probability of the reduction outputting a valid statement-witness pair is
    \emph{exactly} 1 (instead of at least $1 - 0$).
    \lean{Reduction.perfectCompleteness}
    \uses{def:completeness}
\end{definition}

Almost all oracle reductions we consider actually satisfy \emph{perfect completeness}, which
simplifies the proof obligation. In particular, this means we only need to show that no matter what challenges are chosen, the verifier will always accept given messages from the honest prover.

\subsubsection{Extractors}

For knowledge soundness, we need to consider different types of extractors that can recover witnesses from malicious provers.

\begin{definition}[Straightline Extractor]
    \label{def:straightline_extractor}
    A \textbf{straightline, deterministic, non-oracle-querying extractor} takes in:
    \begin{itemize}
        \item the output witness $w_{\text{out}}$,
        \item the initial statement $x_{\text{in}}$,
        \item the IOR transcript $\tau$,
        \item the query logs from the prover and verifier
    \end{itemize}
    and returns a corresponding initial witness $w_{\text{in}}$.

    Note that the extractor does not need to take in the output statement, since it can be derived
    via re-running the verifier on the initial statement, the transcript, and the verifier's query
    log.

    This form of extractor suffices for proving knowledge soundness of most hash-based IOPs.
    \lean{Extractor.Straightline}
    \uses{def:transcript}
\end{definition}

\begin{definition}[Round-by-Round Extractor]
    \label{def:rbr_extractor}
    A \textbf{round-by-round extractor} with index $m$ is given:
    \begin{itemize}
        \item the input statement $x_{\text{in}}$,
        \item a partial transcript of length $m$,
        \item the prover's query log
    \end{itemize}
    and returns a witness to the statement.

    Note that the RBR extractor does not need to take in the output statement or witness.
    \lean{Extractor.RoundByRound}
    \uses{def:transcript}
\end{definition}

\begin{definition}[Rewinding Extractor]
    \label{def:rewinding_extractor}
    A \textbf{rewinding extractor} consists of:
    \begin{itemize}
        \item An extractor state type
        \item Simulation oracles for challenges and oracle queries for the prover
        \item A function that runs the extractor with the prover's oracle interface, allowing for calling the prover multiple times
    \end{itemize}
    This allows the extractor to rewind the prover to earlier states and try different challenges.
    \lean{Extractor.Rewinding}
    \uses{def:prover, def:oracle_interface}
\end{definition}

\subsubsection{Adversarial Provers}

% \begin{definition}[Adaptive Prover]
%     \label{def:adaptive_prover}
%     An \textbf{adaptive prover} extends the basic prover type with the ability to choose the input statement adaptively based on oracle access. This models stronger adversaries that can choose their statements after seeing some oracle responses.
%     \lean{Prover.Adaptive}
%     \uses{def:prover}
% \end{definition}

\begin{definition}[State-Restoration Prover]
    \label{def:sr_prover}
    A \textbf{state-restoration prover} is a modified prover that has query access to challenge oracles that can return the $i$-th challenge, for all $i$, given the input statement and the transcript up to that point.

    It takes in the input statement and witness, and outputs a full transcript of interaction,
    along with the output statement and witness.

    This models adversaries in the state-restoration setting where challenges can be queried programmably.
    \lean{Prover.StateRestoration.KnowledgeSoundness}
    \uses{def:transcript}
\end{definition}

\subsubsection{Soundness Definitions}

For soundness, we need to consider different notions. These notions differ in two main aspects:
\begin{itemize}
    \item Whether we consider the plain soundness, or knowledge soundness. The latter relies on the
    notion of an \emph{extractor}.
    \item Whether we consider plain, state-restoration, round-by-round, or rewinding notion of
    soundness.
\end{itemize}

We note that state-restoration knowledge soundness is necessary for the security of the SNARK
protocol obtained from the oracle reduction after composing with a commitment scheme and applying
the Fiat-Shamir transform. It in turn is implied by either round-by-round knowledge soundness, or
special soundness (via rewinding). At the moment, we only care about non-rewinding soundness, so mostly we will care about round-by-round knowledge soundness.

\begin{definition}[Soundness]
    \label{def:soundness}
    A reduction satisfies \textbf{soundness} with error $\epsilon \geq 0$ and with respect to input
    language $L_{\text{in}} \subseteq \text{Statement}_{\text{in}}$ and output language $L_{\text{out}} \subseteq \text{Statement}_{\text{out}}$ if:
    \begin{itemize}
        \item for all (malicious) provers with arbitrary types for witness types,
        \item for all arbitrary input witness,
        \item for all input statement $x_{\text{in}} \notin L_{\text{in}}$,
    \end{itemize}
    the execution between the prover and the honest verifier will result in an output statement
    $x_{\text{out}} \in L_{\text{out}}$ with probability at most $\epsilon$.
    \lean{Verifier.soundness}
    \uses{def:verifier, def:prover_run}
\end{definition}

\begin{definition}[Knowledge Soundness]
    \label{def:knowledge_soundness}
    A reduction satisfies \textbf{(straightline) knowledge soundness} with error $\epsilon \geq 0$ and
    with respect to input relation $R_{\text{in}}$ and output relation $R_{\text{out}}$ if:
    \begin{itemize}
        \item there exists a straightline extractor $E$, such that
        \item for all input statement $x_{\text{in}}$, witness $w_{\text{in}}$, and (malicious) prover,
        \item if the execution with the honest verifier results in a pair $(x_{\text{out}}, w_{\text{out}})$,
        \item and the extractor produces some $w'_{\text{in}}$,
    \end{itemize}
    then the probability that $(x_{\text{in}}, w'_{\text{in}})$ is not valid for $R_{\text{in}}$ and yet $(x_{\text{out}}, w_{\text{out}})$ is valid for $R_{\text{out}}$ is at most $\epsilon$.

    A (straightline) extractor for knowledge soundness is a deterministic algorithm that takes in the output public context after executing the oracle reduction, the side information (i.e. log of oracle queries from the malicious prover) observed during execution, and outputs the witness for the input context.

    Note that since we assume the context is append-only, and we append only the public (or oracle)
    messages obtained during protocol execution, it follows that the witness stays the same throughout
    the execution.
    \lean{Verifier.knowledgeSoundness}
    \uses{def:verifier, def:reduction_run, def:straightline_extractor}
\end{definition}

\subsubsection{Round-by-Round Security}

To define round-by-round (knowledge) soundness, we need to define the notion of a \emph{state function}. This is a (possibly inefficient) function $\mathsf{StateF}$ that, for every challenge sent by the verifier, takes in the transcript of the protocol so far and outputs whether the state is doomed or not. Roughly speaking, the requirement of round-by-round soundness is that, for any (possibly malicious) prover $P$, if the state function outputs that the state is doomed on some partial transcript of the protocol, then the verifier will reject with high probability.

\begin{definition}[State Function]
    \label{def:state_function}
    A \textbf{(deterministic) state function} for a verifier, with respect to input language $L_{\text{in}}$ and
    output language $L_{\text{out}}$, consists of a function that maps partial transcripts to boolean values, satisfying:
    \begin{itemize}
        \item For all input statements not in the language, the state function is false for the empty transcript
        \item If the state function is false for a partial transcript, and the next message is from the
        prover to the verifier, then the state function is also false for the new partial transcript
        regardless of the message
        \item If the state function is false for a full transcript, the verifier will not output a statement
        in the output language
    \end{itemize}
    \lean{Verifier.StateFunction}
\end{definition}

\begin{definition}[Knowledge State Function]
    \label{def:knowledge_state_function}
    A \textbf{knowledge state function} for a verifier, with respect to input relation $R_{\text{in}}$, output
    relation $R_{\text{out}}$, and intermediate witness types, extends the basic state function to track
    witness validity throughout the protocol execution. This is used to define round-by-round knowledge soundness.
    \lean{Verifier.KnowledgeStateFunction}
\end{definition}

\begin{definition}[Round-by-Round Soundness]
    \label{def:round_by_round_soundness}
    A protocol with verifier $\mathcal{V}$ satisfies \textbf{round-by-round soundness} with respect to input language
    $L_{\text{in}}$, output language $L_{\text{out}}$, and error function $\epsilon: \text{ChallengeIdx} \to \mathbb{R}_{\geq 0}$ if:
    \begin{itemize}
        \item there exists a state function for the verifier and the input/output languages, such that
        \item for all initial statements $x_{\text{in}} \notin L_{\text{in}}$,
        \item for all initial witnesses,
        \item for all provers,
        \item for all challenge rounds $i$,
    \end{itemize}
    the probability that:
    \begin{itemize}
        \item the state function is false for the partial transcript output by the prover
        \item the state function is true for the partial transcript appended by next challenge (chosen randomly)
    \end{itemize}
    is at most $\epsilon(i)$.
    \lean{Verifier.rbrSoundness}
    \uses{def:verifier, def:state_function, def:prover_run_to_round}
\end{definition}

\begin{definition}[Round-by-Round Knowledge Soundness]
    \label{def:round_by_round_knowledge_soundness}
    A protocol with verifier $\mathcal{V}$ satisfies \textbf{round-by-round knowledge soundness} with respect to input
    relation $R_{\text{in}}$, output relation $R_{\text{out}}$, and error function $\epsilon: \text{ChallengeIdx} \to \mathbb{R}_{\geq 0}$ if:
    \begin{itemize}
        \item there exists a knowledge state function for the verifier and the languages of the input/output relations,
        \item there exists a round-by-round extractor,
        \item for all initial statements,
        \item for all initial witnesses,
        \item for all provers,
        \item for all challenge rounds $i$,
    \end{itemize}
    the probability that:
    \begin{itemize}
        \item the extracted witness does not satisfy the input relation
        \item the state function is false for the partial transcript output by the prover
        \item the state function is true for the partial transcript appended by next challenge (chosen randomly)
    \end{itemize}
    is at most $\epsilon(i)$.
    \lean{Verifier.rbrKnowledgeSoundness}
    \uses{def:verifier, def:knowledge_state_function, def:rbr_extractor, def:prover_run_to_round}
\end{definition}

% \begin{remark}[Alternative Formulations of RBR Knowledge Soundness]
%     There are different ways to formulate round-by-round knowledge soundness, differing in whether
%     the extractor's failure to produce a valid witness is included as part of the security condition.
%     Some formulations condition on the extractor producing an invalid witness while the state function
%     transitions from false to true, while others may condition on the state function transition
%     regardless of extractor success. The current formalization includes the extractor failure as
%     part of the security condition.
% \end{remark}

% \subsubsection{Extractor Properties}

% These definitions are highly experimental and may change in the future. The goal is to put some conditions on the extractor in order for prove sequential composition preserves knowledge soundness.

% \begin{definition}[Monotone Straightline Extractor]
%     \label{def:monotone_straightline_extractor}
%     An extractor is \textbf{monotone} if its success probability on a given query log is the same as
%     the success probability on any extension of that query log. This property ensures that the extractor's
%     performance does not degrade when given more information.
%     \lean{Verifier.Extractor.Straightline.IsMonotone}
%     \uses{def:straightline_extractor}
% \end{definition}

% \begin{definition}[Monotone RBR Extractor]
%     \label{def:monotone_rbr_extractor}
%     A round-by-round extractor is \textbf{monotone} if its success probability on a given query log
%     is the same as the success probability on any extension of that query log.
%     \lean{Extractor.RoundByRoundOneShot.IsMonotone}
%     \uses{def:rbr_extractor}
% \end{definition}

\subsubsection{Implications Between Security Notions}

We have a lattice of security notions, with knowledge and round-by-round being two strengthenings of soundness.

\begin{theorem}[Knowledge Soundness Implies Soundness]
    \label{thm:knowledge_soundness_implies_soundness}
    Knowledge soundness with knowledge error $\epsilon < 1$ implies soundness with the same
    soundness error $\epsilon$, and for the corresponding input and output languages.
    \lean{Verifier.knowledgeSoundness_implies_soundness}
    \uses{def:knowledge_soundness, def:soundness}
\end{theorem}

\begin{theorem}[RBR Soundness Implies Soundness]
    \label{thm:rbr_soundness_implies_soundness}
    Round-by-round soundness with error function $\epsilon$ implies soundness with error
    $\sum_i \epsilon(i)$, where the sum is over all challenge rounds $i$.
    \lean{Verifier.rbrSoundness_implies_soundness}
    \uses{def:round_by_round_soundness, def:soundness}
\end{theorem}

\begin{theorem}[RBR Knowledge Soundness Implies RBR Soundness]
    \label{thm:rbr_knowledge_soundness_implies_rbr_soundness}
    Round-by-round knowledge soundness with error function $\epsilon$ implies round-by-round
    soundness with the same error function $\epsilon$.
    \lean{Verifier.rbrKnowledgeSoundness_implies_rbrSoundness}
    \uses{def:round_by_round_knowledge_soundness, def:round_by_round_soundness}
\end{theorem}

\begin{theorem}[RBR Knowledge Soundness Implies Knowledge Soundness]
    \label{thm:rbr_knowledge_soundness_implies_knowledge_soundness}
    Round-by-round knowledge soundness with error function $\epsilon$ implies knowledge soundness
    with error $\sum_i \epsilon(i)$, where the sum is over all challenge rounds $i$.
    \lean{Verifier.rbrKnowledgeSoundness_implies_knowledgeSoundness}
    \uses{def:round_by_round_knowledge_soundness, def:knowledge_soundness}
\end{theorem}

\subsubsection{Zero-Knowledge}

\begin{definition}[Simulator]
    \label{def:simulator}
    A \textbf{simulator} consists of:
    \begin{itemize}
        \item Oracle simulation capabilities for the shared oracles
        \item A prover simulation function that takes an input statement and produces a transcript
    \end{itemize}
    The simulator should have programming access to the shared oracles and be able to generate
    transcripts that are indistinguishable from real protocol executions.
    \lean{Reduction.Simulator}
\end{definition}

\begin{remark}[Zero-Knowledge Definition]
    We define honest-verifier zero-knowledge as follows: There exists a simulator such that for all
    (malicious) verifiers, the distributions of transcripts generated by the simulator and the
    interaction between the verifier and the prover are (statistically) indistinguishable.
    A full definition will be provided in future versions.
\end{remark}

\subsubsection{Oracle-Specific Security}

For oracle reductions, the security definitions are analogous to those for standard reductions, but adapted to work with oracle interfaces:

\begin{definition}[Oracle Reduction Completeness]
    \label{def:oracle_reduction_completeness}
    Completeness of an oracle reduction is the same as for non-oracle reductions, but applied to the
    converted reduction where oracle statements are handled through their interfaces.
    \lean{OracleReduction.completeness}
    \uses{def:oracle_reduction, def:completeness, def:oracle_verifier_to_verifier}
\end{definition}

\begin{definition}[Oracle Verifier Soundness]
    \label{def:oracle_verifier_soundness}
    Soundness of an oracle verifier is defined by converting it to a standard verifier and applying
    the standard soundness definition.
    \lean{OracleVerifier.soundness}
    \uses{def:oracle_verifier, def:soundness, def:oracle_verifier_to_verifier}
\end{definition}

\begin{definition}[Oracle Verifier Knowledge Soundness]
    \label{def:oracle_verifier_knowledge_soundness}
    Knowledge soundness of an oracle verifier is defined by converting it to a standard verifier
    and applying the standard knowledge soundness definition.
    \lean{OracleVerifier.knowledgeSoundness}
    \uses{def:oracle_verifier, def:knowledge_soundness, def:oracle_verifier_to_verifier}
\end{definition}

\begin{definition}[Oracle Verifier RBR Soundness]
    \label{def:oracle_verifier_rbr_soundness}
    Round-by-round soundness of an oracle verifier is defined by converting it to a standard verifier
    and applying the standard round-by-round soundness definition.
    \lean{OracleVerifier.rbrSoundness}
    \uses{def:oracle_verifier, def:round_by_round_soundness, def:oracle_verifier_to_verifier}
\end{definition}

\begin{definition}[Oracle Verifier RBR Knowledge Soundness]
    \label{def:oracle_verifier_rbr_knowledge_soundness}
    Round-by-round knowledge soundness of an oracle verifier is defined by converting it to a standard
    verifier and applying the standard round-by-round knowledge soundness definition.
    \lean{OracleVerifier.rbrKnowledgeSoundness}
    \uses{def:oracle_verifier, def:round_by_round_knowledge_soundness, def:oracle_verifier_to_verifier}
\end{definition}

By default, the properties we consider are perfect completeness and (straightline) round-by-round knowledge soundness. We can encapsulate these properties into the following typing judgement:

\[
    \Gamma := (\Psi; \Theta; \varSigma; \rho; \mathcal{O}) \vdash \{\mathcal{R}_1\} \quad \langle\mathcal{P}, \mathcal{V}, \mathcal{E}\rangle \quad \{\!\!\{\mathcal{R}_2; \mathsf{St}; \epsilon\}\!\!\}
\]

\subsubsection{State-Restoration Security}

\begin{definition}[State-Restoration Soundness]
    \label{def:sr_soundness}
    \textbf{State-restoration soundness} is a security notion where the adversarial prover has access to
    challenge oracles that can return the $i$-th challenge for any round $i$, given the input statement
    and the transcript up to that point. This models stronger adversaries in the programmable random
    oracle model or when challenges can be computed deterministically.

    A verifier satisfies state-restoration soundness if for all input statements not in the language,
    for all witnesses, and for all state-restoration provers, the probability that the verifier
    outputs a statement in the output language is bounded by the soundness error.

    \emph{Note: This definition is currently under development in the Lean formalization.}
    % \lean{Verifier.srSoundness}
\end{definition}

\begin{definition}[State-Restoration Knowledge Soundness]
    \label{def:sr_knowledge_soundness}
    \textbf{State-restoration knowledge soundness} extends state-restoration soundness with the
    requirement that there exists a straightline extractor that can recover valid witnesses from
    any state-restoration prover that convinces the verifier.

    \emph{Note: This definition is currently under development in the Lean formalization.}
    % \lean{Verifier.srKnowledgeSoundness}
\end{definition}


% We can now define properties of interactive reductions. The two main properties we consider in this
% project are completeness and various notions of soundness. We will cover zero-knowledge at a later
% stage.

% First, for completeness, this is essentially probabilistic Hoare-style conditions on the execution
% of the oracle reduction (with the honest prover and verifier). In other words, given a predicate on
% the initial context, and a predicate on the final context, we require that if the initial predicate
% holds, then the final predicate holds with high probability (except for some \emph{completeness}
% error).

% \begin{definition}[Completeness]
%     \label{def:completeness}
%     \lean{Reduction.completeness}
%     \uses{def:oracle_reduction}
% \end{definition}

% Almost all oracle reductions we consider actually satisfy \emph{perfect completeness}, which
% simplifies the proof obligation. In particular, this means we only need to show that no matter what challenges are chosen, the verifier will always accept given messages from the honest prover.

% For soundness, we need to consider different notions. These notions differ in two main aspects:
% \begin{itemize}
%     \item Whether we consider the plain soundness, or knowledge soundness. The latter relies on the
%     notion of an \emph{extractor}.
%     \item Whether we consider plain, state-restoration, round-by-round, or rewinding notion of
%     soundness.
% \end{itemize}

% We note that state-restoration knowledge soundness is necessary for the security of the SNARK
% protocol obtained from the oracle reduction after composing with a commitment scheme and applying
% the Fiat-Shamir transform. It in turn is implied by either round-by-round knowledge soundness, or
% special soundness (via rewinding). At the moment, we only care about non-rewinding soundness, so mostly we will care about round-by-round knowledge soundness.

% \begin{definition}[Soundness]
%     \label{def:soundness}
%     \lean{Verifier.soundness}
%     \uses{def:oracle_reduction}
% \end{definition}

% A (straightline) extractor for knowledge soundness is a deterministic algorithm that takes in the output public context after executing the oracle reduction, the side information (i.e. log of oracle queries from the malicious prover) observed during execution, and outputs the witness for the input context.

% Note that since we assume the context is append-only, and we append only the public (or oracle)
% messages obtained during protocol execution, it follows that the witness stays the same throughout
% the execution.

% \begin{definition}[Knowledge Soundness]
%     \label{def:knowledge_soundness}
%     \lean{Verifier.knowledgeSoundness}
%     \uses{def:oracle_reduction}
% \end{definition}

% To define round-by-round (knowledge) soundness, we need to define the notion of a \emph{state function}. This is a (possibly inefficient) function $\mathsf{StateF}$ that, for every challenge sent by the verifier, takes in the transcript of the protocol so far and outputs whether the state is doomed or not. Roughly speaking, the requirement of round-by-round soundness is that, for any (possibly malicious) prover $P$, if the state function outputs that the state is doomed on some partial transcript of the protocol, then the verifier will reject with high probability.

% \begin{definition}[State Function]
%     \label{def:state_function}
%     \lean{Verifier.StateFunction}
% \end{definition}

% \begin{definition}[Round-by-Round Soundness]
%     \label{def:round_by_round_soundness}
%     \lean{Verifier.rbrSoundness}
%     \uses{def:oracle_reduction}
% \end{definition}

% \begin{definition}[Round-by-Round Knowledge Soundness]
%     \label{def:round_by_round_knowledge_soundness}
%     \lean{Verifier.rbrKnowledgeSoundness}
%     \uses{def:oracle_reduction}
% \end{definition}

% \textbf{PL Formalization.} We write our definitions in PL notation in~\Cref{fig:type-defs}. The set of types $\Type$ is the same as Lean's dependent type theory (omitting universe levels); in particular, we care about basic dependent types (Pi and Sigma), finite natural numbers, finite fields, lists, vectors, and polynomials.

% \begin{figure}[t]
%     \[\begin{array}{rcl}
%         % Basic types
%         \mathsf{Type} &::=& \mathsf{Unit} \mid \mathsf{Bool} \mid \mathbb{N} \mid \mathsf{Fin}\; n \mid \mathbb{F}_q \mid \mathsf{List}\;(\alpha : \mathsf{Type}) \mid (i : \iota) \to \alpha\; i \mid (i : \iota) \times \alpha\; i \mid \dots \\[1em]
%         % Protocol message types
%         \mathsf{Dir} &::=& \mathsf{P2V.Pub} \mid \mathsf{P2V.Orac} \mid \mathsf{V2P} \\
%         \mathsf{OI}\; (\mathrm{M} : \Type) &::=& \langle \mathrm{Q}, \mathrm{R}, \mathrm{M} \to \mathrm{Q} \to \mathrm{R} \rangle \\
%         % Protocol type signature
%         \pSpec\; (n : \mathbb{N}) &::=& \mathsf{Fin}\; n \to (d : \mathsf{Dir}) \times (M : \Type) \times (\mathsf{if}\; d = \mathsf{P2V.Orac} \; \mathsf{then} \; \mathsf{OI}(M) \; \mathsf{else} \; \mathsf{Unit}) \\
%         % Oracle type signature
%         \oSpec \; (\iota : \mathsf{Type}) &::=& (i : \iota) \to \mathsf{dom}\; i \times \mathsf{range}\; i \\[1em]
%         % Contexts
%         \varSigma &::=& \emptyset \mid \varSigma \times \Type \\
%         \Omega &::=& \emptyset \mid \Omega \times \langle \mathrm{M} : \Type, \mathsf{OI}(\mathrm{M}) \rangle \\
%         \Psi &::=& \emptyset \mid \Psi \times \Type\\
%     \end{array}\]
%     \[\begin{array}{rcl}
%         \Gamma &::=& (\Psi; \Omega; \varSigma; \rho; \mathcal{O})\\
%         \mathsf{OComp}^{\mathcal{O}}\; (\alpha : \Type) &::=& \mid\; \mathsf{pure}\; (a : \alpha) \\
%         && \mid\; \mathsf{queryBind}\;(i : \iota)\; (q : \mathsf{dom}\; i)\; (k : \mathsf{range}\; i \to \mathsf{OComp}^{\mathcal{O}}\; \alpha) \\
%         && \mid\; \mathsf{fail} \\[1em]
%         \tau_{\mathsf{P}}(\Gamma) &::=& (i : \mathsf{Fin}\; n) \to (h : (\rho \; i).\mathsf{fst} = \mathsf{P2V}) \to \\
%         && \varSigma \to \Omega \to \Psi \to \rho_{[:i]} \to \mathsf{OComp}^{\mathcal{O}}\;\left( (\rho \; i).\mathsf{snd}\right) \\[1em]

%         \tau_{\mathsf{V}}(\Gamma) &::=& \varSigma \to (\rho.\mathsf{Chals}) \to \mathsf{OComp}^{\mathcal{O} :: \OI(\Omega) :: \OI(\rho.\mathsf{Msg.Orac})}\; \mathsf{Unit} \\[1em]
%         \tau_{\mathsf{E}}(\Gamma) &::=& \varSigma \to \Omega \to \rho.\mathsf{Transcript} \to \calO.\mathsf{QueryLog} \to \Psi
%     \end{array}\]
%     \caption{Type definitions for interactive oracle reductions}
%     \label{fig:type-defs}
% \end{figure}

% Using programming language notation, we can express an interactive oracle reduction as a typing judgment:
% \[
%     \Gamma := (\Psi; \Theta; \varSigma; \rho; \mathcal{O}) \vdash \mathcal{P} : \tau_{\mathsf{P}}(\Gamma), \; \mathcal{V} : \tau_{\mathsf{V}}(\Gamma)
% \]
% where:
% \begin{itemize}
%     \item $\Psi$ represents the witness (private) inputs
%     \item $\Theta$ represents the oracle inputs
%     \item $\varSigma$ represents the public inputs (i.e. statements)
%     \item $\mathcal{O} : \oSpec\; \iota$ represents the shared oracle
%     \item $\rho : \pSpec\; n$ represents the protocol type signature
%     \item $\mathcal{P}$ and $\mathcal{V}$ are the prover and verifier, respectively, being of the given types $\tau_{\mathsf{P}}(\Gamma)$ and $\tau_{\mathsf{V}}(\Gamma)$.
% \end{itemize}

% To exhibit valid elements for the prover and verifier types, we will use existing functions in the ambient programming language (e.g. Lean).

% By default, the properties we consider are perfect completeness and (straightline) round-by-round knowledge soundness. We can encapsulate these properties into the following typing judgement:

% \[
%     \Gamma := (\Psi; \Theta; \varSigma; \rho; \mathcal{O}) \vdash \{\mathcal{R}_1\} \quad \langle\mathcal{P}, \mathcal{V}, \mathcal{E}\rangle \quad \{\!\!\{\mathcal{R}_2; \mathsf{St}; \epsilon\}\!\!\}
% \]


\section{Definitions}\label{sec:oracle_reductions_defs}

In this section, we give the basic definitions of a public-coin interactive oracle reduction
(henceforth called an oracle reduction or IOR). We will define its building blocks, and various
security properties.

\subsection{Format}\label{sec:oracle_reductions_defs_format}

An \textbf{(interactive) oracle reduction (IOR)} is an interactive protocol between two parties, a
\emph{prover} $\mathcal{P}$ and a \emph{verifier} $\mathcal{V}$. In ArkLib, IORs are defined in the
following setting:
\begin{enumerate}
    \item We work in an ambient dependent type theory (in our case, Lean).

    \item The protocol flow is fixed and defined by a given \emph{type signature}, which
    describes in each round which party sends a message to the other, and the type of that message.

    \item The prover and verifier has access to some inputs (called the \emph{(oracle) context}) at
    the beginning of the protocol. These inputs are classified as follows:
    \begin{itemize}
        \item \emph{Public inputs} (or \emph{statement}) $\mathbbm{x}$: available to both parties;
        \item \emph{Private inputs} (or \emph{witness}) $\mathbbm{w}$: available only to the prover;
        \item \emph{Oracle inputs} (or \emph{oracle statement}) $\mathbbm{ox}$: the underlying data
        is available to the prover, but it's only exposed as an oracle to the verifier. See~\Cref{def:oracle_interface} for more information.
        \item \emph{Shared oracle} $\mathcal{O}$: the oracle is available to both parties via an
        interface; in most cases, it is either empty, a probabilistic sampling oracle, a random
        oracle, or a group oracle (for the Algebraic Group Model). See~\Cref{sec:vcvio} for more
        information on oracle computations.
    \end{itemize}

    \item The messages sent from the prover may either: 1) be seen directly by the verifier, or 2)
    only available to a verifier through an \emph{oracle interface} (which specifies the type for
    the query and response, and the oracle's behavior given the underlying message).

    Currently, in the oracle reduction setting, we \emph{only} allow messages sent to be available
    through oracle interfaces. In the (non-oracle) reduction setting, all messages are available
    directly. Future extensions may allow for mixed visibility for prover's messages.

    \item $\mathcal{V}$ is assumed to be \emph{public-coin}, meaning that its challenges are chosen
    uniformly at random from the finite type corresponding to that round, and it uses no randomness
    otherwise, except from those coming from the shared oracle.

    \item At the end of the protocol, the prover and verifier outputs a new (oracle) context, which consists of:
    \begin{itemize}
        \item The verifier takes in the input statement and the challenges, performs an \emph{oracle} computation on the input oracle statements and the oracle messages, and outputs a new output statement.

        The verifier also outputs the new oracle statement in an implicit manner, by specifying a
        subset of the input oracle statements \& the oracle messages. Future extensions may allow for more flexibility in specifying output oracle statements (i.e. not just a subset, but a linear combination, or any other function).
        \item The prover takes in some final private state (maintained during protocol execution), and outputs a new output statement, new output oracle statement, and new output witness.
    \end{itemize}
\end{enumerate}

\begin{remark}[Literature Comparison]
In the literature, our definition corresponds to the notion of \emph{functional} IORs. Historically,
(vector) IOPs were the first notion to be introduced by~\cite{IOPs}; these are IORs where the output
statement is true/false, all oracle statements and messages are vectors over some alphabet $\Sigma$,
and the oracle interfaces are for querying specific positions in the vector. More recent works have
considered other oracle interfaces, e.g., polynomial oracles~\cite{Marlin, DARK}, generalized proofs
to reductions~\cite{ARoK, WARP, Arc, fics-facs}, and considered general oracle
interfaces~\cite{WHIR}. Most of the IOP theory has been distilled in the
textbook~\cite{ChiesaYogev2024}.

We have not seen any work that considers our most general setting, of IORs with arbitrary oracle interfaces.
\end{remark}

We now go into more details on these objects, and how they are represented in Lean. Our description will aim to be as close as possible to the Lean code, and hence may differ somewhat from ``mainstream'' mathematical \& cryptographic notation.

\begin{definition}[Oracle Interface]
    \label{def:oracle_interface}
    An oracle interface for an underlying data type $\mathsf{D}$ consists of the following:
    \begin{itemize}
        \item A type $\mathsf{Q}$ for queries to the oracle,
        \item A type $\mathsf{R}$ for responses from the oracle,
        \item A function $\mathsf{oracle} : \mathsf{D} \to \mathsf{Q} \to \mathsf{R}$ that specifies
        the oracle's behavior given the underlying data and a query.
    \end{itemize}
    \lean{OracleInterface}
\end{definition}

See \texttt{OracleInterface.lean} for common instances of $\mathsf{OracleInterface}$.


\begin{definition}[Context]
    \label{def:context}
    In an (oracle) reduction, its \emph{(oracle) context} consists of a statement type, a witness
    type, and (in the oracle case) an indexed list of oracle statement types.

    Currently, we do not abstract out / bundle the context as a separate structure, but rather
    specifies the types explicitly. This may change in the future.
\end{definition}

\begin{definition}[Protocol Specification]
    \label{def:protocol_spec}
    A protocol specification for an $n$-message (oracle) reduction, is an element of the following type:
    \begin{align*}
        \ProtocolSpec\ n &:= \Fin\ n \to \Direction \times \Type.
    \end{align*}
    In the above, $\Direction := \{ \PtoVdir, \VtoPdir \}$ is the type of possible directions of messages, and $\Fin\ n := \{ i : \bbN \quotient i < n \}$ is the type of all natural numbers less than $n$.

    In other words, for each step $i$ of interaction, the protocol specification describes the \emph{direction} of the message sent in that step, i.e., whether it is from the prover or from the verifier. It also describes the \emph{type} of that message.

    In the oracle setting, we also expect an oracle interface for each message from the prover to the verifier.
    \lean{ProtocolSpec}
\end{definition}

We define some supporting definitions for a protocol specification.

\begin{definition}[Protocol Specification Components]
    \label{def:protocol_spec_components}
    Given a protocol spec $\pSpec : \ProtocolSpec\ n$, we define:
    \begin{itemize}
        \item $\pSpec.\Dir\ i := (\pSpec\ i).\mathsf{fst}$ extracts the direction of the $i$-th message.
        \item $\pSpec.\Type\ i := (\pSpec\ i).\mathsf{snd}$ extracts the type of the $i$-th message.
        \item $\pSpec.\MessageIdx := \{i : \Fin\ n \quotient \pSpec.\Dir\ i = \PtoVdir\}$ is the subtype of indices corresponding to prover messages.
        \item $\pSpec.\ChallengeIdx := \{i : \Fin\ n \quotient \pSpec.\Dir\ i = \VtoPdir\}$ is the subtype of indices corresponding to verifier challenges.
        \item $\pSpec.\mathsf{Message}\ i := (i : \pSpec.\MessageIdx) \to \pSpec.\Type\ i.\mathsf{val}$ is an indexed family of message types in the protocol.
        \item $\pSpec.\mathsf{Challenge}\ i := (i : \pSpec.\ChallengeIdx) \to \pSpec.\Type\ i.\mathsf{val}$ is an indexed family of challenge types in the protocol.
    \end{itemize}
    \lean{ProtocolSpec.dir, ProtocolSpec.Type, ProtocolSpec.MessageIdx, ProtocolSpec.ChallengeIdx, ProtocolSpec.Message, ProtocolSpec.Challenge}
    \uses{def:protocol_spec}
\end{definition}

\begin{definition}[Protocol Transcript]
    \label{def:transcript}
        Given protocol specification $\pSpec : \ProtocolSpec\ n$, we define:
    \begin{itemize}
        \item A \emph{transcript} up to round $k : \Fin\ (n + 1)$ is an element of type
                \[ \Transcript\ k\ \pSpec := (i : \Fin\ k) \to \pSpec.\Type\ (\uparrow i : \Fin\ n) \]
        where $\uparrow i : \Fin\ n$ denotes casting $i : \Fin\ k$ to $\Fin\ n$ (valid since $k \leq n + 1$).

        \item A \emph{full transcript} is $\FullTranscript\ \pSpec := (i : \Fin\ n) \to \pSpec.\Type\ i$.

        \item The type of all \emph{messages} from prover to verifier is
        \[ \pSpec.\Messages := \prod_{i : \pSpec.\MessageIdx} \pSpec.\Message\ i \]

        \item The type of all \emph{challenges} from verifier to prover is
        \[ \pSpec.\Challenges := \prod_{i : \pSpec.\ChallengeIdx} \pSpec.\Challenge\ i \]
    \end{itemize}
    \lean{ProtocolSpec.Transcript, ProtocolSpec.Message, ProtocolSpec.Challenge}
    \uses{def:protocol_spec, def:protocol_spec_components}
\end{definition}

% In the interactive protocols we consider, both parties $P$ and $V$ may have access to a shared
% oracle $O$. An interactive protocol becomes an \emph{interactive (oracle) reduction} if its
% execution reduces an input relation $R_{\mathsf{in}}$ to an output relation $R_{\mathsf{out}}$. Here
% a relation is just a function $\mathsf{IsValid}: \mathsf{Statement} \times \mathsf{Witness} \to
% \mathsf{Bool}$, for some types \verb|Statement| and \verb|Witness|. We do not concern ourselves with
% the running time of $\mathsf{IsValid}$ in this project (though future extensions may prove that
% relations can be decided in polynomial time, for a suitable model of computation).

\begin{remark}[Design Decision]
    We do not enforce a particular interaction flow in the definition of an interactive (oracle) reduction. This is done so that we can capture all protocols in the most generality. Also, we want to allow the prover to send multiple messages in a row, since each message may have a different oracle representation (for instance, in the Plonk protocol, the prover's first message is a 3-tuple of polynomial commitments.)
\end{remark}

\begin{definition}[Type Signature of a Prover]
    \label{def:prover}
    A prover $\mathcal{P}$ in a reduction consists of the following components:

    \begin{itemize}
        \item \textbf{Prover State}: A family of types $\mathsf{PrvState} : \Fin(n+1) \to \Type$ representing the prover's internal state at each round of the protocol.

        \item \textbf{Input Processing}: A function
        \[ \mathsf{input} : \StmtIn \to \WitIn \to \mathsf{PrvState}(0) \]
        that initializes the prover's state from the input statement and witness.

        \item \textbf{Message Sending}: For each message index $i : \pSpec.\MessageIdx$, a function
        \[ \mathsf{sendMessage}_i : \mathsf{PrvState}(i.\mathsf{val}.\mathsf{castSucc}) \to \OracleComp(\oSpec, \pSpec.\Message(i) \times \mathsf{PrvState}(i.\mathsf{val}.\mathsf{succ})) \]
        that generates the message and updates the prover's state.

        \item \textbf{Challenge Processing}: For each challenge index $i : \pSpec.\ChallengeIdx$, a function
        \[ \mathsf{receiveChallenge}_i : \mathsf{PrvState}(i.\mathsf{val}.\mathsf{castSucc}) \to \pSpec.\Challenge(i) \to \mathsf{PrvState}(i.\mathsf{val}.\mathsf{succ}) \]
        that updates the prover's state upon receiving a challenge.

        \item \textbf{Output Generation}: A function
        \[ \mathsf{output} : \mathsf{PrvState}(\Fin.\mathsf{last}(n)) \to \StmtOut \times \WitOut \]
        that produces the final output statement and witness from the prover's final state.
    \end{itemize}
    \lean{Prover, ProverState, ProverInput, ProverRound, ProverOutput}
\end{definition}

\begin{definition}[Type Signature of an Oracle Prover]
    \label{def:oracle_prover}
    An oracle prover is a prover whose input statement includes the underlying data for oracle statements, and whose output includes oracle statements. Formally, it is a prover with input statement type $\StmtIn \times (\forall i : \iota_{\mathsf{si}}, \OStmtIn(i))$ and output statement type $\StmtOut \times (\forall i : \iota_{\mathsf{so}}, \OStmtOut(i))$, where:
    \begin{itemize}
        \item $\OStmtIn : \iota_{\mathsf{si}} \to \Type$ are the input oracle statement types
        \item $\OStmtOut : \iota_{\mathsf{so}} \to \Type$ are the output oracle statement types
    \end{itemize}
    \lean{OracleProver}
\end{definition}

% Our modeling of oracle reductions only consider \emph{public-coin} verifiers; that is, verifiers who
% only outputs uniformly random challenges drawn from the (finite) types, and uses no other
% randomness. Because of this fixed functionality, we can bake the verifier's behavior in the
% interaction phase directly into the protocol execution semantics.

After the interaction phase, the verifier may then run some verification procedure to check the
validity of the prover's responses. In this procedure, the verifier gets access to the public part
of the context, and oracle access to either the shared oracle, or the oracle inputs.
% This procedure differs depending on whether the verifier has
% full access, or only oracle access, to the prover's messages. Note that there is no difference on
% the prover side whether the protocol is an \emph{interactive oracle reduction (IOR)} or simply an
% \emph{interactive reduction (IR)}.

\begin{definition}[Type Signature of a Verifier]
    \label{def:verifier}
    A verifier $\mathcal{V}$ in a reduction is specified by a single function:
    \[ \mathsf{verify} : \StmtIn \to \FullTranscript(\pSpec) \to \OracleComp(\oSpec, \StmtOut) \]

    This function takes the input statement and the complete transcript of the protocol interaction, and performs an oracle computation (potentially querying the shared oracle $\oSpec$) to produce an output statement.

    The verifier is assumed to be \emph{public-coin}, meaning it only sends uniformly random challenges and uses no other randomness beyond what is provided by the shared oracle.
    \lean{Verifier}
\end{definition}

\begin{definition}[Type Signature of an Oracle Verifier]
    \label{def:oracle_verifier}
    An oracle verifier $\mathcal{V}$ consists of the following components:

    \begin{itemize}
        \item \textbf{Verification Logic}: A function
        \[ \mathsf{verify} : \StmtIn \to \pSpec.\Challenges \to \OracleComp(\oSpec \mathrel{++_\mathsf{o}} ([\OStmtIn]_\mathsf{o} \mathrel{++_\mathsf{o}} [\pSpec.\Message]_\mathsf{o}), \StmtOut) \]
        that takes the input statement and verifier challenges, and performs oracle queries to the shared oracle, input oracle statements, and prover messages to produce an output statement.

        \item \textbf{Output Oracle Embedding}: An injective function
        \[ \mathsf{embed} : \iota_{\mathsf{so}} \hookrightarrow \iota_{\mathsf{si}} \oplus \pSpec.\MessageIdx \]
        that specifies how each output oracle statement is derived from either an input oracle statement or a prover message.

        \item \textbf{Type Compatibility}: A proof term
        \[ \mathsf{hEq} : \forall i : \iota_{\mathsf{so}}, \OStmtOut(i) = \begin{cases}
            \OStmtIn(j) & \text{if } \mathsf{embed}(i) = \mathsf{inl}(j) \\
            \pSpec.\Message(k) & \text{if } \mathsf{embed}(i) = \mathsf{inr}(k)
        \end{cases} \]
        ensuring that output oracle statement types match their sources.
    \end{itemize}

    This design ensures that output oracle statements are always a subset of the available input oracle statements and prover messages.
    \lean{OracleVerifier}
\end{definition}

\begin{definition}[Oracle Verifier to Verifier Conversion]
    \label{def:oracle_verifier_to_verifier}
    An oracle verifier can be converted to a standard verifier through a natural simulation process. The key insight is that while an oracle verifier only has oracle access to certain data (input oracle statements and prover messages), a standard verifier can be given the actual underlying data directly.

    The conversion works as follows: when the oracle verifier needs to make an oracle query to some data, the converted verifier can respond to this query immediately using the actual underlying data it possesses. This is accomplished through the \texttt{OracleInterface} type class, which specifies for each data type how to respond to queries given the underlying data.

    Specifically, given an oracle verifier $\mathcal{V}_{\text{oracle}}$:
    \begin{itemize}
        \item The converted verifier $\mathcal{V}_{\text{oracle}}.\mathsf{toVerifier}$ takes as input both the statement \emph{and} the actual underlying data for all oracle statements
        \item When $\mathcal{V}_{\text{oracle}}$ attempts to query an oracle statement or prover message, the converted verifier uses the corresponding \texttt{OracleInterface} instance to compute the response from the actual data
        \item The output oracle statements are constructed according to the embedding specification, selecting the appropriate subset of input oracle statements and prover messages
    \end{itemize}
    \lean{OracleVerifier.toVerifier}
    \uses{def:oracle_verifier}
\end{definition}

An oracle reduction then consists of a type signature for the interaction, and a pair of prover and
verifier for that type signature.

\begin{definition}[Interactive Reduction]
    \label{def:reduction}
    An interactive reduction for protocol specification $\pSpec : \ProtocolSpec(n)$ and oracle specification $\oSpec$ consists of:
    \begin{itemize}
        \item A \textbf{prover} $\mathcal{P} : \Prover(\pSpec, \oSpec, \StmtIn, \WitIn, \StmtOut, \WitOut)$
        \item A \textbf{verifier} $\mathcal{V} : \Verifier(\pSpec, \oSpec, \StmtIn, \StmtOut)$
    \end{itemize}

    The reduction establishes a relationship between input relations on $(\StmtIn, \WitIn)$ and output relations on $(\StmtOut, \WitOut)$ through the interactive protocol defined by $\pSpec$.
    \lean{Reduction}
    \uses{def:prover, def:verifier}
\end{definition}

\begin{definition}[Interactive Oracle Reduction]
    \label{def:oracle_reduction}
    An interactive oracle reduction for protocol specification $\pSpec : \ProtocolSpec(n)$ with oracle interfaces for all prover messages, and oracle specification $\oSpec$, consists of:
    \begin{itemize}
        \item An \textbf{oracle prover} $\mathcal{P} : \OracleProver(\pSpec, \oSpec, \StmtIn, \WitIn, \StmtOut, \WitOut, \OStmtIn, \OStmtOut)$
        \item An \textbf{oracle verifier} $\mathcal{V} : \OracleVerifier(\pSpec, \oSpec, \StmtIn, \StmtOut, \OStmtIn, \OStmtOut)$
    \end{itemize}

    where:
    \begin{itemize}
        \item $\OStmtIn : \iota_{\mathsf{si}} \to \Type$ are the input oracle statement types with oracle interfaces
        \item $\OStmtOut : \iota_{\mathsf{so}} \to \Type$ are the output oracle statement types
    \end{itemize}

    The oracle reduction allows the verifier to access prover messages and oracle statements only through specified oracle interfaces, enabling more flexible and composable protocol designs.
    \lean{OracleReduction}
    \uses{def:oracle_prover, def:oracle_verifier}
\end{definition}

\subsection{Execution Semantics}\label{sec:execution_semantics}

We now define what it means to execute an oracle reduction. This is essentially achieved by first
executing the prover, interspersed with oracle queries to get the verifier's challenges (these will
be given uniform random probability semantics later on), and then executing the verifier's checks.
Any message exchanged in the protocol will be added to the context. We may also log information
about the execution, such as the log of oracle queries for the shared oracles, for analysis purposes
(i.e. feeding information into the extractor).

\begin{definition}[Prover Execution to Round]
    \label{def:prover_run_to_round}
    The execution of a prover up to round $i : \Fin(n+1)$ is defined inductively:

    \[ \mathsf{Prover}.\mathsf{runToRound}(i, \mathsf{stmt}, \mathsf{wit}) := \]
    \[ \mathsf{Fin}.\mathsf{induction}( \]
    \[ \quad \mathsf{pure}(\langle \mathsf{default}, \mathsf{prover}.\mathsf{input}(\mathsf{stmt}, \mathsf{wit}) \rangle), \]
    \[ \quad \mathsf{prover}.\mathsf{processRound}, \]
    \[ \quad i \]
    \[ ) \]

    where $\mathsf{processRound}$ handles individual rounds by either:
    \begin{itemize}
        \item \textbf{Verifier Challenge} ($\pSpec.\mathsf{dir}(j) = \mathsf{V\_to\_P}$): Query for a challenge and update prover state
        \item \textbf{Prover Message} ($\pSpec.\mathsf{dir}(j) = \mathsf{P\_to\_V}$): Generate message via $\mathsf{sendMessage}$ and update state
    \end{itemize}

    Returns the transcript up to round $i$ and the prover's state after round $i$.
    \lean{Prover.runToRound, Prover.processRound}
    \uses{def:prover, def:protocol_spec, def:transcript}
\end{definition}

\begin{definition}[Complete Prover Execution]
    \label{def:prover_run}
    The complete execution of a prover is defined as:

    \[ \mathsf{Prover}.\mathsf{run}(\mathsf{stmt}, \mathsf{wit}) := \mathsf{do} \; \{ \]
    \[ \quad \langle \mathsf{transcript}, \mathsf{state} \rangle \leftarrow \mathsf{prover}.\mathsf{runToRound}(\Fin.\mathsf{last}(n), \mathsf{stmt}, \mathsf{wit}) \]
    \[ \quad \langle \mathsf{stmtOut}, \mathsf{witOut} \rangle := \mathsf{prover}.\mathsf{output}(\mathsf{state}) \]
    \[ \quad \mathsf{return} \; \langle \mathsf{stmtOut}, \mathsf{witOut}, \mathsf{transcript} \rangle \]
    \[ \} \]

    Returns the output statement, output witness, and complete transcript.
    \lean{Prover.run}
    \uses{def:prover, def:prover_run_to_round}
\end{definition}

\begin{definition}[Verifier Execution]
    \label{def:verifier_run}
    The execution of a verifier is simply the application of its verification function:

    \[ \mathsf{Verifier}.\mathsf{run}(\mathsf{stmt}, \mathsf{transcript}) := \mathsf{verifier}.\mathsf{verify}(\mathsf{stmt}, \mathsf{transcript}) \]

    This takes the input statement and full transcript, and returns the output statement via an oracle computation.
    \lean{Verifier.run}
    \uses{def:verifier}
\end{definition}

\begin{definition}[Oracle Verifier Execution]
    \label{def:oracle_verifier_run}
    The execution of an oracle verifier is defined as:

    \[ \mathsf{OracleVerifier}.\mathsf{run}(\mathsf{stmt}, \mathsf{oStmtIn}, \mathsf{transcript}) := \mathsf{do} \; \{ \]
    \[ \quad \mathsf{f} := \mathsf{simOracle2}(\oSpec, \mathsf{oStmtIn}, \mathsf{transcript}.\mathsf{messages}) \]
    \[ \quad \mathsf{stmtOut} \leftarrow \mathsf{simulateQ}(\mathsf{f}, \mathsf{verifier}.\mathsf{verify}(\mathsf{stmt}, \mathsf{transcript}.\mathsf{challenges})) \]
    \[ \quad \mathsf{return} \; \mathsf{stmtOut} \]
    \[ \} \]

    This simulates the oracle access to input oracle statements and prover messages, then executes the verification logic.
    \lean{OracleVerifier.run}
    \uses{def:oracle_verifier, def:oracle_interface}
\end{definition}

\begin{definition}[Interactive Reduction Execution]
    \label{def:reduction_run}
    The execution of an interactive reduction consists of running the prover followed by the verifier:

    \[ \mathsf{Reduction}.\mathsf{run}(\mathsf{stmt}, \mathsf{wit}) := \mathsf{do} \; \{ \]
    \[ \quad \langle \mathsf{prvStmtOut}, \mathsf{witOut}, \mathsf{transcript} \rangle \leftarrow \mathsf{reduction}.\mathsf{prover}.\mathsf{run}(\mathsf{stmt}, \mathsf{wit}) \]
    \[ \quad \mathsf{stmtOut} \leftarrow \mathsf{reduction}.\mathsf{verifier}.\mathsf{run}(\mathsf{stmt}, \mathsf{transcript}) \]
    \[ \quad \mathsf{return} \; ((\mathsf{prvStmtOut}, \mathsf{witOut}), \mathsf{stmtOut}, \mathsf{transcript}) \]
    \[ \} \]

    Returns both the prover's output (statement and witness) and the verifier's output statement, along with the complete transcript.
    \lean{Reduction.run}
    \uses{def:reduction, def:prover_run, def:verifier_run}
\end{definition}

\begin{definition}[Oracle Reduction Execution]
    \label{def:oracle_reduction_run}
    The execution of an interactive oracle reduction is similar to a standard reduction but includes logging of oracle queries:

    \[ \mathsf{OracleReduction}.\mathsf{run}(\mathsf{stmt}, \mathsf{wit}, \mathsf{oStmt}) := \mathsf{do} \; \{ \]
    \[ \quad \langle \langle \mathsf{prvStmtOut}, \mathsf{witOut}, \mathsf{transcript} \rangle, \mathsf{proveQueryLog} \rangle \leftarrow \]
    \[ \qquad (\mathsf{simulateQ}(\mathsf{loggingOracle}, \mathsf{reduction}.\mathsf{prover}.\mathsf{run}(\langle \mathsf{stmt}, \mathsf{oStmt} \rangle, \mathsf{wit}))).\mathsf{run} \]
    \[ \quad \langle \mathsf{stmtOut}, \mathsf{verifyQueryLog} \rangle \leftarrow \]
    \[ \qquad (\mathsf{simulateQ}(\mathsf{loggingOracle}, \mathsf{reduction}.\mathsf{verifier}.\mathsf{run}(\mathsf{stmt}, \mathsf{oStmt}, \mathsf{transcript}))).\mathsf{run} \]
    \[ \quad \mathsf{return} \; ((\mathsf{prvStmtOut}, \mathsf{witOut}), \mathsf{stmtOut}, \mathsf{transcript}, \mathsf{proveQueryLog}, \mathsf{verifyQueryLog}) \]
    \[ \} \]

    Returns the same outputs as a standard reduction, plus logs of all oracle queries made by both the prover and verifier.
    \lean{OracleReduction.run}
    \uses{def:oracle_reduction, def:prover_run, def:oracle_verifier_run}
\end{definition}


\subsection{Security Properties}\label{sec:security}

We can now define properties of interactive reductions. The two main properties we consider in this
project are completeness and various notions of soundness. We will cover zero-knowledge at a later
stage.

First, for completeness, this is essentially probabilistic Hoare-style conditions on the execution
of the oracle reduction (with the honest prover and verifier). In other words, given a predicate on
the initial context, and a predicate on the final context, we require that if the initial predicate
holds, then the final predicate holds with high probability (except for some \emph{completeness}
error).

\begin{definition}[Completeness]
    \label{def:completeness}
    A reduction satisfies \textbf{completeness} with error $\epsilon \geq 0$ and with respect to
    input relation $R_{\text{in}}$ and output relation $R_{\text{out}}$, if for all valid statement-witness pair
    $(x_{\text{in}}, w_{\text{in}})$ for $R_{\text{in}}$, the execution between the honest prover and the honest verifier
    will result in a tuple $((x_{\text{out}}^P, w_{\text{out}}), x_{\text{out}}^V)$ such that:
    \begin{itemize}
        \item $R_{\text{out}}(x_{\text{out}}^V, w_{\text{out}}) = \text{True}$ (the output statement-witness pair is valid), and
        \item $x_{\text{out}}^P = x_{\text{out}}^V$ (the output statements are the same from both prover and verifier)
    \end{itemize}
    except with probability $\epsilon$.
    \lean{Reduction.completeness}
    \uses{def:reduction, def:reduction_run}
\end{definition}

\begin{definition}[Perfect Completeness]
    \label{def:perfect_completeness}
    A reduction satisfies \textbf{perfect completeness} if it satisfies completeness with error $0$.
    This means that the probability of the reduction outputting a valid statement-witness pair is
    \emph{exactly} 1 (instead of at least $1 - 0$).
    \lean{Reduction.perfectCompleteness}
    \uses{def:completeness}
\end{definition}

Almost all oracle reductions we consider actually satisfy \emph{perfect completeness}, which
simplifies the proof obligation. In particular, this means we only need to show that no matter what challenges are chosen, the verifier will always accept given messages from the honest prover.

\subsubsection{Extractors}

For knowledge soundness, we need to consider different types of extractors that can recover witnesses from malicious provers.

\begin{definition}[Straightline Extractor]
    \label{def:straightline_extractor}
    A \textbf{straightline, deterministic, non-oracle-querying extractor} takes in:
    \begin{itemize}
        \item the output witness $w_{\text{out}}$,
        \item the initial statement $x_{\text{in}}$,
        \item the IOR transcript $\tau$,
        \item the query logs from the prover and verifier
    \end{itemize}
    and returns a corresponding initial witness $w_{\text{in}}$.

    Note that the extractor does not need to take in the output statement, since it can be derived
    via re-running the verifier on the initial statement, the transcript, and the verifier's query
    log.

    This form of extractor suffices for proving knowledge soundness of most hash-based IOPs.
    \lean{Extractor.Straightline}
    \uses{def:transcript}
\end{definition}

\begin{definition}[Round-by-Round Extractor]
    \label{def:rbr_extractor}
    A \textbf{round-by-round extractor} with index $m$ is given:
    \begin{itemize}
        \item the input statement $x_{\text{in}}$,
        \item a partial transcript of length $m$,
        \item the prover's query log
    \end{itemize}
    and returns a witness to the statement.

    Note that the RBR extractor does not need to take in the output statement or witness.
    \lean{Extractor.RoundByRound}
    \uses{def:transcript}
\end{definition}

\begin{definition}[Rewinding Extractor]
    \label{def:rewinding_extractor}
    A \textbf{rewinding extractor} consists of:
    \begin{itemize}
        \item An extractor state type
        \item Simulation oracles for challenges and oracle queries for the prover
        \item A function that runs the extractor with the prover's oracle interface, allowing for calling the prover multiple times
    \end{itemize}
    This allows the extractor to rewind the prover to earlier states and try different challenges.
    \lean{Extractor.Rewinding}
    \uses{def:prover, def:oracle_interface}
\end{definition}

\subsubsection{Adversarial Provers}

% \begin{definition}[Adaptive Prover]
%     \label{def:adaptive_prover}
%     An \textbf{adaptive prover} extends the basic prover type with the ability to choose the input statement adaptively based on oracle access. This models stronger adversaries that can choose their statements after seeing some oracle responses.
%     \lean{Prover.Adaptive}
%     \uses{def:prover}
% \end{definition}

\begin{definition}[State-Restoration Prover]
    \label{def:sr_prover}
    A \textbf{state-restoration prover} is a modified prover that has query access to challenge oracles that can return the $i$-th challenge, for all $i$, given the input statement and the transcript up to that point.

    It takes in the input statement and witness, and outputs a full transcript of interaction,
    along with the output statement and witness.

    This models adversaries in the state-restoration setting where challenges can be queried programmably.
    \lean{Prover.StateRestoration.KnowledgeSoundness}
    \uses{def:transcript}
\end{definition}

\subsubsection{Soundness Definitions}

For soundness, we need to consider different notions. These notions differ in two main aspects:
\begin{itemize}
    \item Whether we consider the plain soundness, or knowledge soundness. The latter relies on the
    notion of an \emph{extractor}.
    \item Whether we consider plain, state-restoration, round-by-round, or rewinding notion of
    soundness.
\end{itemize}

We note that state-restoration knowledge soundness is necessary for the security of the SNARK
protocol obtained from the oracle reduction after composing with a commitment scheme and applying
the Fiat-Shamir transform. It in turn is implied by either round-by-round knowledge soundness, or
special soundness (via rewinding). At the moment, we only care about non-rewinding soundness, so mostly we will care about round-by-round knowledge soundness.

\begin{definition}[Soundness]
    \label{def:soundness}
    A reduction satisfies \textbf{soundness} with error $\epsilon \geq 0$ and with respect to input
    language $L_{\text{in}} \subseteq \text{Statement}_{\text{in}}$ and output language $L_{\text{out}} \subseteq \text{Statement}_{\text{out}}$ if:
    \begin{itemize}
        \item for all (malicious) provers with arbitrary types for witness types,
        \item for all arbitrary input witness,
        \item for all input statement $x_{\text{in}} \notin L_{\text{in}}$,
    \end{itemize}
    the execution between the prover and the honest verifier will result in an output statement
    $x_{\text{out}} \in L_{\text{out}}$ with probability at most $\epsilon$.
    \lean{Verifier.soundness}
    \uses{def:verifier, def:prover_run}
\end{definition}

\begin{definition}[Knowledge Soundness]
    \label{def:knowledge_soundness}
    A reduction satisfies \textbf{(straightline) knowledge soundness} with error $\epsilon \geq 0$ and
    with respect to input relation $R_{\text{in}}$ and output relation $R_{\text{out}}$ if:
    \begin{itemize}
        \item there exists a straightline extractor $E$, such that
        \item for all input statement $x_{\text{in}}$, witness $w_{\text{in}}$, and (malicious) prover,
        \item if the execution with the honest verifier results in a pair $(x_{\text{out}}, w_{\text{out}})$,
        \item and the extractor produces some $w'_{\text{in}}$,
    \end{itemize}
    then the probability that $(x_{\text{in}}, w'_{\text{in}})$ is not valid for $R_{\text{in}}$ and yet $(x_{\text{out}}, w_{\text{out}})$ is valid for $R_{\text{out}}$ is at most $\epsilon$.

    A (straightline) extractor for knowledge soundness is a deterministic algorithm that takes in the output public context after executing the oracle reduction, the side information (i.e. log of oracle queries from the malicious prover) observed during execution, and outputs the witness for the input context.

    Note that since we assume the context is append-only, and we append only the public (or oracle)
    messages obtained during protocol execution, it follows that the witness stays the same throughout
    the execution.
    \lean{Verifier.knowledgeSoundness}
    \uses{def:verifier, def:reduction_run, def:straightline_extractor}
\end{definition}

\subsubsection{Round-by-Round Security}

To define round-by-round (knowledge) soundness, we need to define the notion of a \emph{state function}. This is a (possibly inefficient) function $\mathsf{StateF}$ that, for every challenge sent by the verifier, takes in the transcript of the protocol so far and outputs whether the state is doomed or not. Roughly speaking, the requirement of round-by-round soundness is that, for any (possibly malicious) prover $P$, if the state function outputs that the state is doomed on some partial transcript of the protocol, then the verifier will reject with high probability.

\begin{definition}[State Function]
    \label{def:state_function}
    A \textbf{(deterministic) state function} for a verifier, with respect to input language $L_{\text{in}}$ and
    output language $L_{\text{out}}$, consists of a function that maps partial transcripts to boolean values, satisfying:
    \begin{itemize}
        \item For all input statements not in the language, the state function is false for the empty transcript
        \item If the state function is false for a partial transcript, and the next message is from the
        prover to the verifier, then the state function is also false for the new partial transcript
        regardless of the message
        \item If the state function is false for a full transcript, the verifier will not output a statement
        in the output language
    \end{itemize}
    \lean{Verifier.StateFunction}
\end{definition}

\begin{definition}[Knowledge State Function]
    \label{def:knowledge_state_function}
    A \textbf{knowledge state function} for a verifier, with respect to input relation $R_{\text{in}}$, output
    relation $R_{\text{out}}$, and intermediate witness types, extends the basic state function to track
    witness validity throughout the protocol execution. This is used to define round-by-round knowledge soundness.
    \lean{Verifier.KnowledgeStateFunction}
\end{definition}

\begin{definition}[Round-by-Round Soundness]
    \label{def:round_by_round_soundness}
    A protocol with verifier $\mathcal{V}$ satisfies \textbf{round-by-round soundness} with respect to input language
    $L_{\text{in}}$, output language $L_{\text{out}}$, and error function $\epsilon: \text{ChallengeIdx} \to \mathbb{R}_{\geq 0}$ if:
    \begin{itemize}
        \item there exists a state function for the verifier and the input/output languages, such that
        \item for all initial statements $x_{\text{in}} \notin L_{\text{in}}$,
        \item for all initial witnesses,
        \item for all provers,
        \item for all challenge rounds $i$,
    \end{itemize}
    the probability that:
    \begin{itemize}
        \item the state function is false for the partial transcript output by the prover
        \item the state function is true for the partial transcript appended by next challenge (chosen randomly)
    \end{itemize}
    is at most $\epsilon(i)$.
    \lean{Verifier.rbrSoundness}
    \uses{def:verifier, def:state_function, def:prover_run_to_round}
\end{definition}

\begin{definition}[Round-by-Round Knowledge Soundness]
    \label{def:round_by_round_knowledge_soundness}
    A protocol with verifier $\mathcal{V}$ satisfies \textbf{round-by-round knowledge soundness} with respect to input
    relation $R_{\text{in}}$, output relation $R_{\text{out}}$, and error function $\epsilon: \text{ChallengeIdx} \to \mathbb{R}_{\geq 0}$ if:
    \begin{itemize}
        \item there exists a knowledge state function for the verifier and the languages of the input/output relations,
        \item there exists a round-by-round extractor,
        \item for all initial statements,
        \item for all initial witnesses,
        \item for all provers,
        \item for all challenge rounds $i$,
    \end{itemize}
    the probability that:
    \begin{itemize}
        \item the extracted witness does not satisfy the input relation
        \item the state function is false for the partial transcript output by the prover
        \item the state function is true for the partial transcript appended by next challenge (chosen randomly)
    \end{itemize}
    is at most $\epsilon(i)$.
    \lean{Verifier.rbrKnowledgeSoundness}
    \uses{def:verifier, def:knowledge_state_function, def:rbr_extractor, def:prover_run_to_round}
\end{definition}

% \begin{remark}[Alternative Formulations of RBR Knowledge Soundness]
%     There are different ways to formulate round-by-round knowledge soundness, differing in whether
%     the extractor's failure to produce a valid witness is included as part of the security condition.
%     Some formulations condition on the extractor producing an invalid witness while the state function
%     transitions from false to true, while others may condition on the state function transition
%     regardless of extractor success. The current formalization includes the extractor failure as
%     part of the security condition.
% \end{remark}

% \subsubsection{Extractor Properties}

% These definitions are highly experimental and may change in the future. The goal is to put some conditions on the extractor in order for prove sequential composition preserves knowledge soundness.

% \begin{definition}[Monotone Straightline Extractor]
%     \label{def:monotone_straightline_extractor}
%     An extractor is \textbf{monotone} if its success probability on a given query log is the same as
%     the success probability on any extension of that query log. This property ensures that the extractor's
%     performance does not degrade when given more information.
%     \lean{Verifier.Extractor.Straightline.IsMonotone}
%     \uses{def:straightline_extractor}
% \end{definition}

% \begin{definition}[Monotone RBR Extractor]
%     \label{def:monotone_rbr_extractor}
%     A round-by-round extractor is \textbf{monotone} if its success probability on a given query log
%     is the same as the success probability on any extension of that query log.
%     \lean{Extractor.RoundByRoundOneShot.IsMonotone}
%     \uses{def:rbr_extractor}
% \end{definition}

\subsubsection{Implications Between Security Notions}

We have a lattice of security notions, with knowledge and round-by-round being two strengthenings of soundness.

\begin{theorem}[Knowledge Soundness Implies Soundness]
    \label{thm:knowledge_soundness_implies_soundness}
    Knowledge soundness with knowledge error $\epsilon < 1$ implies soundness with the same
    soundness error $\epsilon$, and for the corresponding input and output languages.
    \lean{Verifier.knowledgeSoundness_implies_soundness}
    \uses{def:knowledge_soundness, def:soundness}
\end{theorem}

\begin{theorem}[RBR Soundness Implies Soundness]
    \label{thm:rbr_soundness_implies_soundness}
    Round-by-round soundness with error function $\epsilon$ implies soundness with error
    $\sum_i \epsilon(i)$, where the sum is over all challenge rounds $i$.
    \lean{Verifier.rbrSoundness_implies_soundness}
    \uses{def:round_by_round_soundness, def:soundness}
\end{theorem}

\begin{theorem}[RBR Knowledge Soundness Implies RBR Soundness]
    \label{thm:rbr_knowledge_soundness_implies_rbr_soundness}
    Round-by-round knowledge soundness with error function $\epsilon$ implies round-by-round
    soundness with the same error function $\epsilon$.
    \lean{Verifier.rbrKnowledgeSoundness_implies_rbrSoundness}
    \uses{def:round_by_round_knowledge_soundness, def:round_by_round_soundness}
\end{theorem}

\begin{theorem}[RBR Knowledge Soundness Implies Knowledge Soundness]
    \label{thm:rbr_knowledge_soundness_implies_knowledge_soundness}
    Round-by-round knowledge soundness with error function $\epsilon$ implies knowledge soundness
    with error $\sum_i \epsilon(i)$, where the sum is over all challenge rounds $i$.
    \lean{Verifier.rbrKnowledgeSoundness_implies_knowledgeSoundness}
    \uses{def:round_by_round_knowledge_soundness, def:knowledge_soundness}
\end{theorem}

\subsubsection{Zero-Knowledge}

\begin{definition}[Simulator]
    \label{def:simulator}
    A \textbf{simulator} consists of:
    \begin{itemize}
        \item Oracle simulation capabilities for the shared oracles
        \item A prover simulation function that takes an input statement and produces a transcript
    \end{itemize}
    The simulator should have programming access to the shared oracles and be able to generate
    transcripts that are indistinguishable from real protocol executions.
    \lean{Reduction.Simulator}
\end{definition}

\begin{remark}[Zero-Knowledge Definition]
    We define honest-verifier zero-knowledge as follows: There exists a simulator such that for all
    (malicious) verifiers, the distributions of transcripts generated by the simulator and the
    interaction between the verifier and the prover are (statistically) indistinguishable.
    A full definition will be provided in future versions.
\end{remark}

\subsubsection{Oracle-Specific Security}

For oracle reductions, the security definitions are analogous to those for standard reductions, but adapted to work with oracle interfaces:

\begin{definition}[Oracle Reduction Completeness]
    \label{def:oracle_reduction_completeness}
    Completeness of an oracle reduction is the same as for non-oracle reductions, but applied to the
    converted reduction where oracle statements are handled through their interfaces.
    \lean{OracleReduction.completeness}
    \uses{def:oracle_reduction, def:completeness, def:oracle_verifier_to_verifier}
\end{definition}

\begin{definition}[Oracle Verifier Soundness]
    \label{def:oracle_verifier_soundness}
    Soundness of an oracle verifier is defined by converting it to a standard verifier and applying
    the standard soundness definition.
    \lean{OracleVerifier.soundness}
    \uses{def:oracle_verifier, def:soundness, def:oracle_verifier_to_verifier}
\end{definition}

\begin{definition}[Oracle Verifier Knowledge Soundness]
    \label{def:oracle_verifier_knowledge_soundness}
    Knowledge soundness of an oracle verifier is defined by converting it to a standard verifier
    and applying the standard knowledge soundness definition.
    \lean{OracleVerifier.knowledgeSoundness}
    \uses{def:oracle_verifier, def:knowledge_soundness, def:oracle_verifier_to_verifier}
\end{definition}

\begin{definition}[Oracle Verifier RBR Soundness]
    \label{def:oracle_verifier_rbr_soundness}
    Round-by-round soundness of an oracle verifier is defined by converting it to a standard verifier
    and applying the standard round-by-round soundness definition.
    \lean{OracleVerifier.rbrSoundness}
    \uses{def:oracle_verifier, def:round_by_round_soundness, def:oracle_verifier_to_verifier}
\end{definition}

\begin{definition}[Oracle Verifier RBR Knowledge Soundness]
    \label{def:oracle_verifier_rbr_knowledge_soundness}
    Round-by-round knowledge soundness of an oracle verifier is defined by converting it to a standard
    verifier and applying the standard round-by-round knowledge soundness definition.
    \lean{OracleVerifier.rbrKnowledgeSoundness}
    \uses{def:oracle_verifier, def:round_by_round_knowledge_soundness, def:oracle_verifier_to_verifier}
\end{definition}

By default, the properties we consider are perfect completeness and (straightline) round-by-round knowledge soundness. We can encapsulate these properties into the following typing judgement:

\[
    \Gamma := (\Psi; \Theta; \varSigma; \rho; \mathcal{O}) \vdash \{\mathcal{R}_1\} \quad \langle\mathcal{P}, \mathcal{V}, \mathcal{E}\rangle \quad \{\!\!\{\mathcal{R}_2; \mathsf{St}; \epsilon\}\!\!\}
\]

\subsubsection{State-Restoration Security}

\begin{definition}[State-Restoration Soundness]
    \label{def:sr_soundness}
    \textbf{State-restoration soundness} is a security notion where the adversarial prover has access to
    challenge oracles that can return the $i$-th challenge for any round $i$, given the input statement
    and the transcript up to that point. This models stronger adversaries in the programmable random
    oracle model or when challenges can be computed deterministically.

    A verifier satisfies state-restoration soundness if for all input statements not in the language,
    for all witnesses, and for all state-restoration provers, the probability that the verifier
    outputs a statement in the output language is bounded by the soundness error.

    \emph{Note: This definition is currently under development in the Lean formalization.}
    % \lean{Verifier.srSoundness}
\end{definition}

\begin{definition}[State-Restoration Knowledge Soundness]
    \label{def:sr_knowledge_soundness}
    \textbf{State-restoration knowledge soundness} extends state-restoration soundness with the
    requirement that there exists a straightline extractor that can recover valid witnesses from
    any state-restoration prover that convinces the verifier.

    \emph{Note: This definition is currently under development in the Lean formalization.}
    % \lean{Verifier.srKnowledgeSoundness}
\end{definition}


% We can now define properties of interactive reductions. The two main properties we consider in this
% project are completeness and various notions of soundness. We will cover zero-knowledge at a later
% stage.

% First, for completeness, this is essentially probabilistic Hoare-style conditions on the execution
% of the oracle reduction (with the honest prover and verifier). In other words, given a predicate on
% the initial context, and a predicate on the final context, we require that if the initial predicate
% holds, then the final predicate holds with high probability (except for some \emph{completeness}
% error).

% \begin{definition}[Completeness]
%     \label{def:completeness}
%     \lean{Reduction.completeness}
%     \uses{def:oracle_reduction}
% \end{definition}

% Almost all oracle reductions we consider actually satisfy \emph{perfect completeness}, which
% simplifies the proof obligation. In particular, this means we only need to show that no matter what challenges are chosen, the verifier will always accept given messages from the honest prover.

% For soundness, we need to consider different notions. These notions differ in two main aspects:
% \begin{itemize}
%     \item Whether we consider the plain soundness, or knowledge soundness. The latter relies on the
%     notion of an \emph{extractor}.
%     \item Whether we consider plain, state-restoration, round-by-round, or rewinding notion of
%     soundness.
% \end{itemize}

% We note that state-restoration knowledge soundness is necessary for the security of the SNARK
% protocol obtained from the oracle reduction after composing with a commitment scheme and applying
% the Fiat-Shamir transform. It in turn is implied by either round-by-round knowledge soundness, or
% special soundness (via rewinding). At the moment, we only care about non-rewinding soundness, so mostly we will care about round-by-round knowledge soundness.

% \begin{definition}[Soundness]
%     \label{def:soundness}
%     \lean{Verifier.soundness}
%     \uses{def:oracle_reduction}
% \end{definition}

% A (straightline) extractor for knowledge soundness is a deterministic algorithm that takes in the output public context after executing the oracle reduction, the side information (i.e. log of oracle queries from the malicious prover) observed during execution, and outputs the witness for the input context.

% Note that since we assume the context is append-only, and we append only the public (or oracle)
% messages obtained during protocol execution, it follows that the witness stays the same throughout
% the execution.

% \begin{definition}[Knowledge Soundness]
%     \label{def:knowledge_soundness}
%     \lean{Verifier.knowledgeSoundness}
%     \uses{def:oracle_reduction}
% \end{definition}

% To define round-by-round (knowledge) soundness, we need to define the notion of a \emph{state function}. This is a (possibly inefficient) function $\mathsf{StateF}$ that, for every challenge sent by the verifier, takes in the transcript of the protocol so far and outputs whether the state is doomed or not. Roughly speaking, the requirement of round-by-round soundness is that, for any (possibly malicious) prover $P$, if the state function outputs that the state is doomed on some partial transcript of the protocol, then the verifier will reject with high probability.

% \begin{definition}[State Function]
%     \label{def:state_function}
%     \lean{Verifier.StateFunction}
% \end{definition}

% \begin{definition}[Round-by-Round Soundness]
%     \label{def:round_by_round_soundness}
%     \lean{Verifier.rbrSoundness}
%     \uses{def:oracle_reduction}
% \end{definition}

% \begin{definition}[Round-by-Round Knowledge Soundness]
%     \label{def:round_by_round_knowledge_soundness}
%     \lean{Verifier.rbrKnowledgeSoundness}
%     \uses{def:oracle_reduction}
% \end{definition}

% \textbf{PL Formalization.} We write our definitions in PL notation in~\Cref{fig:type-defs}. The set of types $\Type$ is the same as Lean's dependent type theory (omitting universe levels); in particular, we care about basic dependent types (Pi and Sigma), finite natural numbers, finite fields, lists, vectors, and polynomials.

% \begin{figure}[t]
%     \[\begin{array}{rcl}
%         % Basic types
%         \mathsf{Type} &::=& \mathsf{Unit} \mid \mathsf{Bool} \mid \mathbb{N} \mid \mathsf{Fin}\; n \mid \mathbb{F}_q \mid \mathsf{List}\;(\alpha : \mathsf{Type}) \mid (i : \iota) \to \alpha\; i \mid (i : \iota) \times \alpha\; i \mid \dots \\[1em]
%         % Protocol message types
%         \mathsf{Dir} &::=& \mathsf{P2V.Pub} \mid \mathsf{P2V.Orac} \mid \mathsf{V2P} \\
%         \mathsf{OI}\; (\mathrm{M} : \Type) &::=& \langle \mathrm{Q}, \mathrm{R}, \mathrm{M} \to \mathrm{Q} \to \mathrm{R} \rangle \\
%         % Protocol type signature
%         \pSpec\; (n : \mathbb{N}) &::=& \mathsf{Fin}\; n \to (d : \mathsf{Dir}) \times (M : \Type) \times (\mathsf{if}\; d = \mathsf{P2V.Orac} \; \mathsf{then} \; \mathsf{OI}(M) \; \mathsf{else} \; \mathsf{Unit}) \\
%         % Oracle type signature
%         \oSpec \; (\iota : \mathsf{Type}) &::=& (i : \iota) \to \mathsf{dom}\; i \times \mathsf{range}\; i \\[1em]
%         % Contexts
%         \varSigma &::=& \emptyset \mid \varSigma \times \Type \\
%         \Omega &::=& \emptyset \mid \Omega \times \langle \mathrm{M} : \Type, \mathsf{OI}(\mathrm{M}) \rangle \\
%         \Psi &::=& \emptyset \mid \Psi \times \Type\\
%     \end{array}\]
%     \[\begin{array}{rcl}
%         \Gamma &::=& (\Psi; \Omega; \varSigma; \rho; \mathcal{O})\\
%         \mathsf{OComp}^{\mathcal{O}}\; (\alpha : \Type) &::=& \mid\; \mathsf{pure}\; (a : \alpha) \\
%         && \mid\; \mathsf{queryBind}\;(i : \iota)\; (q : \mathsf{dom}\; i)\; (k : \mathsf{range}\; i \to \mathsf{OComp}^{\mathcal{O}}\; \alpha) \\
%         && \mid\; \mathsf{fail} \\[1em]
%         \tau_{\mathsf{P}}(\Gamma) &::=& (i : \mathsf{Fin}\; n) \to (h : (\rho \; i).\mathsf{fst} = \mathsf{P2V}) \to \\
%         && \varSigma \to \Omega \to \Psi \to \rho_{[:i]} \to \mathsf{OComp}^{\mathcal{O}}\;\left( (\rho \; i).\mathsf{snd}\right) \\[1em]

%         \tau_{\mathsf{V}}(\Gamma) &::=& \varSigma \to (\rho.\mathsf{Chals}) \to \mathsf{OComp}^{\mathcal{O} :: \OI(\Omega) :: \OI(\rho.\mathsf{Msg.Orac})}\; \mathsf{Unit} \\[1em]
%         \tau_{\mathsf{E}}(\Gamma) &::=& \varSigma \to \Omega \to \rho.\mathsf{Transcript} \to \calO.\mathsf{QueryLog} \to \Psi
%     \end{array}\]
%     \caption{Type definitions for interactive oracle reductions}
%     \label{fig:type-defs}
% \end{figure}

% Using programming language notation, we can express an interactive oracle reduction as a typing judgment:
% \[
%     \Gamma := (\Psi; \Theta; \varSigma; \rho; \mathcal{O}) \vdash \mathcal{P} : \tau_{\mathsf{P}}(\Gamma), \; \mathcal{V} : \tau_{\mathsf{V}}(\Gamma)
% \]
% where:
% \begin{itemize}
%     \item $\Psi$ represents the witness (private) inputs
%     \item $\Theta$ represents the oracle inputs
%     \item $\varSigma$ represents the public inputs (i.e. statements)
%     \item $\mathcal{O} : \oSpec\; \iota$ represents the shared oracle
%     \item $\rho : \pSpec\; n$ represents the protocol type signature
%     \item $\mathcal{P}$ and $\mathcal{V}$ are the prover and verifier, respectively, being of the given types $\tau_{\mathsf{P}}(\Gamma)$ and $\tau_{\mathsf{V}}(\Gamma)$.
% \end{itemize}

% To exhibit valid elements for the prover and verifier types, we will use existing functions in the ambient programming language (e.g. Lean).

% By default, the properties we consider are perfect completeness and (straightline) round-by-round knowledge soundness. We can encapsulate these properties into the following typing judgement:

% \[
%     \Gamma := (\Psi; \Theta; \varSigma; \rho; \mathcal{O}) \vdash \{\mathcal{R}_1\} \quad \langle\mathcal{P}, \mathcal{V}, \mathcal{E}\rangle \quad \{\!\!\{\mathcal{R}_2; \mathsf{St}; \epsilon\}\!\!\}
% \]


\chapter{References}\label{chap:references}
