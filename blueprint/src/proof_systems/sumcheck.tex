\section{The Sum-Check Protocol}\label{sec:sumcheck}

In this section, we describe the sum-check protocol~\cite{sumcheck-protocol} in a modular manner, as a running example for our approach to specifying and proving properties of oracle reductions (based on a program logic approach).

The sum-check protocol, as described in the original paper and many expositions thereafter, is a
protocol to reduce the claim that \[ \sum_{x \in \{0, 1\}^n} P(x) = c, \] where $P$ is an
$n$-variate polynomial of certain individual degree bounds, and $c$ is some field element, to the
claim that \[ P(r) = v, \] for some claimed value $v$ (derived from the protocol transcript), where
$r$ is a vector of random challenges from the verifier sent during the protocol.

In our language, the initial context of the sum-check protocol is the pair $(P, c)$, where $P$ is an
oracle input and $c$ is public. The protocol proceeds in $n$ rounds of interaction, where in each
round $i$ the prover sends a univariate polynomial $s_i$ of bounded degree and the verifier sends a
challenge $r_i \gets \mathbb{F}$. The honest prover would compute \[ s_i(X) = \sum_{x \in \{0,
1\}^{n - i - 1}} P(r_1, \ldots, r_{i - 1}, X, x), \] and the honest verifier would check that
$s_i(0) + s_i(1) = s_{i - 1}(r_{i - 1})$, with the convention that $s_0(r_0) = c$.

\begin{theorem}
    The sum-check protocol is complete.
    \label{thm:sumcheck_protocol_complete}
    \uses{def:completeness}
\end{theorem}

We now proceed to break down this protocol into individual messages, and then specify the predicates that should hold before and after each message is exchanged.

First, it is clear that we can consider each round in isolation. In fact, each round can be seen as an instantiation of the following simpler "virtual" protocol:

\begin{definition}
    \label{def:virtual_sumcheck_round_protocol}
    \begin{enumerate}
        \item In this protocol, the context is a pair $(p, d)$, where $p$ is now a \emph{univariate} polynomial of bounded degree. The predicate / relation is that $p(0) + p(1) = d$.
        \item The prover first sends a univariate polynomial $s$ of the same bounded degree as $p$. In
        the honest case, it would just send $p$ itself.
        \item The verifier samples and sends a random challenge $r \gets \mathbb{F}$.
        \item The verifier checks that $s(0) + s(1) = d$. The predicate on the resulting output context
        is that $p(r) = s(r)$.
    \end{enumerate}
\end{definition}

The reason why this simpler protocol is related to a sum-check round is that we can \emph{emulate} the simpler protocol using variables in the context at the time:
\begin{itemize}
    \item The univariate polynomial $p$ is instantiated as $\sum_{x \in \{0, 1\}^{n - i - 1}} P(r_1, \ldots, r_{i - 1}, X, x)$.
    \item The scalar $d$ is instantiated as $c$ if $i = 0$, and as $s_{i - 1}(r_{i - 1})$ otherwise.
\end{itemize}

It is "clear" that the simpler protocol is perfectly complete. It is sound (and since there is no
witness, also knowledge sound) since by the Schwartz-Zippel Lemma, the probability that $p \ne s$
and yet $p(r) = s(r)$ for a random challenge $r$ is at most the degree of $p$ over the size of the
field.

\begin{theorem}
    The virtual sum-check round protocol is sound.
    \label{thm:virtual_sumcheck_round_protocol_sound}
    \uses{def:soundness, def:virtual_sumcheck_round_protocol, thm:schwartz_zippel}
\end{theorem}

Note that there is no witness, so knowledge soundness follows trivially from soundness. 

\begin{theorem}
    The virtual sum-check round protocol is knowledge sound.
    \label{thm:virtual_sumcheck_round_protocol_knowledge_sound}
    \uses{def:knowledge_soundness, thm:virtual_sumcheck_round_protocol_sound}
\end{theorem}

Moreover, we
can define the following state function for the simpler protocol:
\begin{enumerate}
    \item The initial state function is the same as the predicate on the initial context, namely that
    $p(0) + p(1) = d$.
    \item The state function after the prover sends $s$ is the predicate that $p(0) + p(1) = d$ and
    $s(0) + s(1) = d$. Essentially, we add in the verifier's check.
    \item The state function for the output context (after the verifier sends $r$) is the predicate that $s(0) + s(1) = d$ and $p(r) = s(r)$.
\end{enumerate}
Seen in this light, it should be clear that the simpler protocol satisfies round-by-round soundness.

\begin{theorem}
    The virtual sum-check round protocol is round-by-round soundness.
    \label{thm:virtual_sumcheck_round_protocol_rbr_sound}
    \uses{def:round_by_round_soundness, thm:virtual_sumcheck_round_protocol_knowledge_sound}
\end{theorem}

In fact, we can break down this simpler protocol even more: consider the two sub-protocols that each
consists of a single message. Then the intermediate state function is the same as the predicate on
the intermediate context, and is given in a "strongest post-condition" style where it incorporates
the verifier's check along with the initial predicate. We can also view the final state function as
a form of "canonical" post-condition, that is implied by the previous predicate except with small
probability.
